\addcontentsline{toc}{chapter}{پیوست}
\chapter*{پیوست‌ها}\label{peyvast}

% پیوست A: کدهای پیاده‌سازی مدل MAGNET
\section{پیوست A: کدهای پیاده‌سازی مدل MAGNET}

\subsection{کد معماری مدل MAGNET}
این بخش کد اصلی معماری مدل MAGNET را ارائه می‌دهد که در فصل 3 به‌صورت شبه‌کد توصیف شد. این کد با استفاده از PyTorch پیاده‌سازی شده است.

\begin{LTR}
\begin{lstlisting}[language=Python, caption={کد معماری مدل MAGNET}, label={lst:magnet_architecture}, basicstyle=\scriptsize\ttfamily]
import torch
import torch.nn as nn

class MAGNET(nn.Module):
    def __init__(self, embedding_dim=64, lstm_num_layers=1, dropout=0.2):
        super(MAGNET, self).__init__()        
        self.embedding_dim = embedding_dim
        # Transformation layers for different data modalities
        self.tab_to_emb = nn.Linear(430, embedding_dim)
        self.graph_to_emb = nn.Linear(embedding_dim, embedding_dim)
        self.sequence_processor = nn.LSTM(embedding_dim, embedding_dim, 
                                         num_layers=lstm_num_layers, batch_first=True)
        # Fusion and classification layers
        self.fusion_layer = nn.Linear(3 * embedding_dim, embedding_dim)
        self.classifier = nn.Linear(embedding_dim, 1)
        self.dropout_layer = nn.Dropout(dropout)
    def forward(self, tab_data, graph_data, seq_data):
        # tab_data: (batch_size, 430)
        # graph_data: (batch_size, embedding_dim)
        # seq_data: (batch_size, seq_len, embedding_dim)
        # Transform different data types to embedding vectors
        tab_emb = torch.relu(self.tab_to_emb(tab_data))
        graph_emb = torch.relu(self.graph_to_emb(graph_data))
        
        # Process sequential data with LSTM
        lstm_out, (hn, cn) = self.sequence_processor(seq_data)
        # Use the last output vector from LSTM for each sample in batch
        seq_emb = lstm_out[:, -1, :]
        # Concatenate embedding vectors
        combined_embeddings = torch.cat((tab_emb, graph_emb, seq_emb), dim=-1)
        # Apply fusion layer and activation
        fused_representation = self.fusion_layer(combined_embeddings)
        fused_representation = torch.relu(fused_representation)
        fused_representation = self.dropout_layer(fused_representation)
        # Final classification
        output = torch.sigmoid(self.classifier(fused_representation))
        return output
\end{lstlisting}
\end{LTR}

\subsection{کد بهینه‌سازی با PIRATES}
این بخش قسمت اصلی کد بهینه‌سازی PIRATES را نشان می‌دهد که برای تنظیم ابرپارامترها استفاده شد.

\begin{LTR}
\begin{lstlisting}[language=Python, caption={کد بهینه‌سازی با PIRATES}, label={lst:pirates_optimization}, basicstyle=\scriptsize\ttfamily]
import numpy as np

class Pirates():
    def __init__(self, func, fmax=(), fmin=(), hr=0.2, ms=3, max_r=1, 
                 num_ships=5, dimensions=2, max_iter=10, max_wind=1, c={},
                 top_ships=10, sailing_radius=0.3, plundering_radius=0.1):        
        # Main algorithm parameters
        self.num_ships = num_ships
        self.num_top_ships = top_ships
        self.max_iter = max_iter
        # Objective function parameters
        self.func_obj = func
        self.cost_func = self.func_obj.func
        self.fmin = fmin
        self.fmax = fmax
        self.dimensions = dimensions
        # Weight parameters
        default_c = {
            'leader': 0.5,
            'private_map': 0.5,
            'map': 0.5,
            'top_ships': 0.5
        }
        self.c = {**default_c, **c}
        # Movement parameters
        self.sailing_radius = sailing_radius
        self.plundering_radius = plundering_radius
        # Leader and map variables
        self.leader_index = None
        self.hr = 1 - hr
        self.r = None
        self.max_r = max_r
        self.ms = ms
        self.map = None
        # Problem type
        self.problem = 'min'
        # Chart variables
        self.bsf_position = None
        self.bsf_list = []
        # Initialization
        self.random_init()
        self.iter = 0
    def search(self):
        """
        Run optimization algorithm and return best results
        Returns:
        --------
        tuple
            (best position, best cost, best metrics)
        """
        # Run algorithm
        self.start()
        # Get results
        result = self.cal_costs()
        if result is not None:
            best_cost, best_metrics = result
        else:
            best_cost = self.costs[self.leader_index]
            best_metrics = {'f1': 0.0, 'accuracy': 0.0, 
                          'precision': 0.0, 'recall': 0.0}
        return self.ships[self.leader_index], best_cost, best_metrics
\end{lstlisting}
\end{LTR}

% پیوست B: داده‌های خام و پیش‌پردازش
\section{پیوست \lr{B}: داده‌های خام و پیش‌پردازش}

\subsection{نمونه داده‌های خام DREBIN}
این جدول نمونه‌ای از داده‌های خام دیتاست DREBIN را نشان می‌دهد که برای آموزش مدل استفاده شد.

\begin{table}[ht]
    \centering
    \caption{نمونه‌ای از داده‌های خام دیتاست DREBIN}
    \label{tab:raw_data_sample}
    \begin{tabular}{|l|c|c|c|}
        \hline
        \textbf{شناسه نمونه} & \textbf{تعداد مجوزها} & \textbf{فراخوانی‌های API} & \textbf{برچسب} \\ 
        \hline
        001 & 15 & \lr{["read\_contacts", "send\_sms"]} & 1 \\ 
        \hline
        002 & 8 & \lr{["get\_accounts"]} & 0 \\ 
        \hline
        003 & 12 & \lr{["read\_phone\_state", "write\_external\_storage"]} & 1 \\ 
        \hline
    \end{tabular}
\end{table}

\subsection{توضیحات پیش‌پردازش}
داده‌ها پیش‌پردازش شدند تا برای مدل مناسب شوند:

\begin{itemize}
    \item تنظیم ابعاد ویژگی‌ها از (16, 32) به (16, 430)
    \item نرمال‌سازی با استفاده از استانداردسازی \lr{z-score}
    \item تبدیل داده‌های متنی به بردارهای باینری
\end{itemize}

% پیوست C: جزئیات سخت‌افزاری و نرم‌افزاری
\section{پیوست \lr{C}: جزئیات سخت‌افزاری و نرم‌افزاری}

\subsection{مشخصات سخت‌افزاری}
آزمایش‌ها با استفاده از زیرساخت زیر اجرا شدند:
\begin{itemize}
    \item \lr{GPU}: \lr{NVIDIA RTX 3090} با 24 گیگابایت \lr{VRAM}
    \item \lr{CPU}: \lr{Intel Xeon E5-2690 v4} با 32 هسته
    \item \lr{RAM}: 128 گیگابایت
\end{itemize}

\subsection{مشخصات نرم‌افزاری}
محیط نرم‌افزاری شامل موارد زیر بود:
\begin{itemize}
    \item زبان برنامه‌نویسی: \lr{Python 3.8.5}
    \item کتابخانه‌ها: 
    \begin{itemize}
        \item \lr{PyTorch 1.9.0}
        \item \lr{PyTorch Geometric 1.7.0}
        \item \lr{Optuna 2.10.0}
    \end{itemize}
    \item سیستم‌عامل: \lr{Ubuntu 20.04 LTS}
\end{itemize}

% پیوست D: نتایج اضافی و ماتریس‌های کامل
\section{پیوست \lr{D}: نتایج اضافی و ماتریس‌های کامل}

\subsection{ماتریس درهم‌ریختگی کامل}
این جدول ماتریس درهم‌ریختگی را برای مجموعه تست با 1,451 نمونه نشان می‌دهد.

\begin{table}[ht]
    \centering
    \caption{ماتریس درهم‌ریختگی برای مجموعه تست}
    \label{tab:confusion_matrix}
    \begin{tabular}{|l|c|c|}
        \hline
        \textbf{پیش‌بینی/واقعیت} & \textbf{کلاس 0} & \textbf{کلاس 1} \\ 
        \hline
        \multicolumn{1}{|c|}{\textbf{کلاس 0}} & \lr{304 (TN)} & \lr{23 (FP)} \\ 
        \hline
        \multicolumn{1}{|c|}{\textbf{کلاس 1}} & \lr{17 (FN)} & \lr{1107 (TP)} \\ 
        \hline
    \end{tabular}
\end{table}

\subsection{گزارش طبقه‌بندی برای هر دسته}
این جدول نتایج هر دسته در اعتبارسنجی متقاطع 5-تایی را نشان می‌دهد.

\begin{table}[ht]
    \centering
    \caption{گزارش طبقه‌بندی برای هر دسته در اعتبارسنجی متقاطع}
    \label{tab:fold_results}
    \begin{tabular}{|l|c|c|c|c|}
        \hline
        \textbf{دسته} & \textbf{\lr{F1 Score}} & \textbf{دقت} & \textbf{\lr{AUC}} & \textbf{زیان} \\ 
        \hline
        دسته 1 & \lr{0.9858} & \lr{0.9785} & \lr{0.9950} & \lr{0.0786} \\ 
        \hline
        دسته 2 & \lr{0.9846} & \lr{0.9763} & \lr{0.9955} & \lr{0.0735} \\ 
        \hline
        دسته 3 & \lr{0.9839} & \lr{0.9752} & \lr{0.9945} & \lr{0.0839} \\ 
        \hline
        دسته 4 & \lr{0.9742} & \lr{0.9601} & \lr{0.9861} & \lr{0.1199} \\ 
        \hline
        دسته 5 & \lr{0.9808} & \lr{0.9709} & \lr{0.9946} & \lr{0.0864} \\ 
        \hline
    \end{tabular}
\end{table}

