% !TeX root=SBUKThesis-main.tex
\clearpage
\thispagestyle{empty}
\chapter{نتیجه‌گیری و پیشنهادات آتی}\label{chap6}

\section{نتیجه‌گیری}
در این پژوهش، یک مدل چندوجهی مبتنی بر ترنسفورمر به نام MAGNET برای تشخیص بدافزار اندروید پیشنهاد شد که از سه ماژول اصلی تشکیل شده است: EnhancedTabTransformer برای پردازش ویژگی‌های جدولی، GraphTransformer برای تحلیل گراف فراخوانی، و SequenceTransformer برای پردازش توالی‌های API. این مدل با هدف بهبود دقت تشخیص بدافزار در محیط‌های اندروید طراحی شد و از الگوریتم‌های بهینه‌سازی Adam و CosineAnnealingWarmRestarts برای تنظیم پارامترها استفاده کرد. بهینه‌سازی ابرپارامترها با دو روش مختلف انجام شد: بهینه‌سازی دستی\LTRfootnote{PIRATES با ۴۷۶ آزمایش} و روش Optuna\LTRfootnote{با ۱۳ آزمایش} \cite{Optuna2019}. دیتاست DREBIN \cite{Drebin} با \lr{6,092} نمونه (\lr{4,641} نمونه برای آموزش و \lr{1,451} نمونه برای تست) برای ارزیابی مدل به کار گرفته شد.

نتایج اعتبارسنجی متقاطع \lr{5}-تایی (جدول \ref{tab:cv_results}) نشان داد که پیکربندی بهینه‌سازی دستی با \lr{3} اپوک به میانگین \lr{F1 Score} \lr{0.9818}، دقت \lr{0.9722}، و \lr{AUC} \lr{0.9932} دست یافت. این نتایج در هر دسته به ترتیب برای \lr{F1 Score} شامل \lr{0.9858}، \lr{0.9846}، \lr{0.9839}، \lr{0.9742}، و \lr{0.9808} بود، که نشان‌دهنده پایداری مدل در دسته‌های مختلف است. همچنین، پیکربندی Optuna با \lr{10} اپوک به \lr{F1 Score} \lr{0.9825}، دقت \lr{0.9730}، و \lr{AUC} \lr{0.9935} رسید، که بهبود جزئی را نسبت به روش PIRATES نشان می‌دهد. این بهبود به دلیل افزایش \lr{num\_epochs} از \lr{3} به \lr{10}، افزایش \lr{embedding\_dim} از \lr{32} به \lr{64}، و کاهش \lr{learning\_rate} از \lr{0.00215} به \lr{0.0019} بود.

در بخش مقایسه با مدل‌های پایه (جدول \ref{tab:comparison_with_literature})، MAGNET با دقت \lr{97.24\%}، \lr{F1 Score} \lr{0.9823}، و \lr{AUC} \lr{0.9932} عملکرد برتری نسبت به روش‌های پایه داشت. به طور خاص، MAGNET نسبت به روش چندوجهی \cite{Alsaleh2023} با دقت \lr{89.2\%} بهبود \lr{8.04\%} در دقت نشان داد و نسبت به روش مبتنی بر ترنسفورمر \cite{TransformerMalware} با دقت \lr{95.8\%}، بهبود \lr{1.44\%} داشت. اگرچه این تفاوت با روش مبتنی بر ترنسفورمر اندک بود، اما MAGNET با ارائه \lr{F1 Score} و \lr{AUC} بالاتر، تعادل بهتری بین دقت و جامعیت ایجاد کرد.

علاوه بر این، مقایسه با مدل‌های یادگیری ماشین کلاسیک و پیشرفته (جدول \ref{tab:baseline_comparison}) نشان داد که MAGNET نسبت به \lr{SVM} (\lr{F1 Score} \lr{0.995} روی دیتاست \lr{CICAndMal2017}، اما \lr{AUC} \lr{0.985})، \lr{Random Forest} (\lr{F1 Score} \lr{0.945} روی \lr{Malgenome})، \lr{XGBoost} (\lr{F1 Score} \lr{0.957} روی \lr{Malgenome})، \lr{ANN} (\lr{F1 Score} \lr{0.962} روی \lr{DREBIN})، \lr{CNN} (\lr{F1 Score} \lr{0.965} روی \lr{VX-Heaven})، و \lr{LSTM} (\lr{F1 Score} \lr{0.882} روی \lr{CICAndMal2017}) عملکرد بهتری دارد. این برتری نشان‌دهنده توانایی MAGNET در ترکیب داده‌های چندوجهی (جدولی، گراف، و توالی) و ارائه یک مدل پایدار و کارآمد است.

تحلیل حساسیت پارامترها (بخش تحلیل حساسیت پارامترهای الگوریتم) نشان داد که \lr{num\_epochs}، \lr{embedding\_dim}، و \lr{learning\_rate} تأثیر قابل‌توجهی بر عملکرد دارند. افزایش \lr{num\_epochs} از \lr{3} به \lr{10} و \lr{embedding\_dim} از \lr{32} به \lr{64}، همراه با کاهش \lr{learning\_rate}، بهبودهای جزئی اما معناداری را ایجاد کرد. با این حال، تغییرات \lr{dropout}\LTRfootnote{از ۰.۲۰۲۹ به ۰.۲} تأثیر محدودی داشت، که نشان‌دهنده پایداری مدل نسبت به این پارامتر است.

در مجموع، مدل MAGNET با استفاده از معماری چندوجهی و بهینه‌سازی دقیق، توانست عملکرد برتری نسبت به روش‌های موجود ارائه دهد و راه‌حلی مؤثر برای تشخیص بدافزار اندروید فراهم آورد. با این حال، محدودیت‌هایی مانند عدم تعادل کلاس‌ها در دیتاست DREBIN و استفاده محدود از داده‌های پویا (مانند الگوهای زمان‌بندی API) شناسایی شد که می‌تواند در تحقیقات آینده بهبود یابد.

\section{پیشنهادات آتی}
\subsection{پژوهش‌های تکمیلی}
برای بهبود عملکرد مدل MAGNET، پیشنهادهای زیر ارائه می‌شود:
\begin{itemize}
    \item \textbf{افزایش تعداد لایه‌های ترنسفورمر}: با توجه به نتایج بهینه‌سازی که \lr{num\_layers = 1} را بهینه یافتند، پیشنهاد می‌شود تأثیر افزایش تعداد لایه‌ها\LTRfootnote{num\_layers به ۲ یا ۳} بررسی شود. این تغییر می‌تواند ظرفیت مدل را برای دیتاست‌های بزرگ‌تر و پیچیده‌تر (مانند \lr{CICAndMal2017} یا \lr{AndroZoo} \cite{AndroZoo}) افزایش دهد.
    \item \textbf{تحلیل داده‌های پویا}: در این پژوهش، داده‌های پویا (مانند الگوهای زمان‌بندی فراخوانی \lr{API} یا فعالیت شبکه) به‌صورت محدود استفاده شدند. پیشنهاد می‌شود مدل MAGNET با داده‌های پویا آزمایش شود تا توانایی آن در تشخیص بدافزارهای پیشرفته‌تر (مانند بدافزارهای روز صفر) ارزیابی شود.
    \item \textbf{مقاومت در برابر حملات گریز}: بررسی مقاومت مدل در برابر حملات گریز (\lr{Adversarial Attacks}) پیشنهاد می‌شود. این شامل تزریق نویز به ویژگی‌های ورودی (مانند گراف فراخوانی یا توالی \lr{API}) و ارزیابی پایداری مدل است.
    \item \textbf{بهینه‌سازی پیشرفته‌تر}: استفاده از روش‌های بهینه‌سازی پیشرفته‌تر مانند \lr{Bayesian Optimization} یا \lr{Genetic Algorithms} برای تنظیم پارامترها می‌تواند جایگزین یا مکمل \lr{Optuna} \cite{Optuna2019} شود و پیکربندی‌های بهتری ارائه دهد.
\end{itemize}

\subsection{پیشنهادات اجرایی}
\begin{itemize}
    \item \textbf{پیاده‌سازی در سیستم‌های امنیتی واقعی}: پیشنهاد می‌شود مدل MAGNET به‌عنوان بخشی از یک سیستم امنیتی واقعی برای اندروید پیاده‌سازی شود. این سیستم می‌تواند با ادغام داده‌های پویا (مانند فعالیت شبکه، دسترسی به فایل‌ها، و الگوهای زمان‌بندی) دقت تشخیص را در محیط‌های عملیاتی افزایش دهد. به عنوان مثال، این مدل می‌تواند به‌عنوان افزونه‌ای برای Google Play Protect توسعه یابد و در زمان واقعی از کاربران در برابر بدافزارها محافظت کند.
    \item \textbf{ادغام با فناوری‌های ابری}: برای بهبود مقیاس‌پذیری، پیشنهاد می‌شود مدل MAGNET در یک پلتفرم ابری پیاده‌سازی شود تا بتواند داده‌های حجیم و متنوع را پردازش کرده و به‌روزرسانی‌های مداوم را دریافت کند.
    \item \textbf{ارزیابی در دستگاه‌های واقعی}: آزمایش مدل روی دستگاه‌های اندرویدی واقعی (به جای شبیه‌سازی) می‌تواند اثربخشی آن را در شرایط عملیاتی واقعی نشان دهد. این شامل تست مدل روی دستگاه‌هایی با منابع محدود (مانند حافظه و \lr{CPU}) است.
\end{itemize}

\subsection{تولید داده‌های جدید}
\begin{itemize}
    \item \textbf{تعادل کلاس‌ها}: دیتاست DREBIN با عدم تعادل کلاس‌ها مواجه است. پیشنهاد می‌شود دیتاستی با تعادل بیشتر جمع‌آوری شود، به طوری که تعداد نمونه‌های کلاس \lr{0} به حداقل \lr{1,000} نمونه افزایش یابد تا به \lr{1,124} نمونه کلاس \lr{1} نزدیک شود. این کار می‌تواند تأثیر سوگیری کلاس را کاهش دهد.
    \item \textbf{افزودن ویژگی‌های جدید}: افزودن ویژگی‌های جدید مانند الگوهای رفتاری کاربران (مانند الگوهای استفاده از اپلیکیشن‌ها)، داده‌های شبکه (مانند ترافیک ورودی و خروجی)، و داده‌های سنسور (مانند شتاب‌سنج) می‌تواند دقت مدل را بهبود بخشد.
    \item \textbf{جمع‌آوری داده‌های پویا}: جمع‌آوری داده‌های پویا از محیط‌های واقعی (مانند رفتار بدافزار در زمان اجرا) و ایجاد دیتاستی جامع‌تر برای آزمایش مدل MAGNET پیشنهاد می‌شود.
\end{itemize}

\subsection{تحلیل‌های آینده}
\begin{itemize}
    \item \textbf{تحلیل مصرف منابع}: بررسی مصرف منابع مدل MAGNET (مانند زمان اجرا، حافظه، و مصرف باتری) در دستگاه‌های اندرویدی پیشنهاد می‌شود تا کارایی آن در شرایط عملیاتی ارزیابی شود.
    \item \textbf{مقایسه با روش‌های ترکیبی}: مقایسه مدل MAGNET با روش‌های ترکیبی (مانند ترکیب ترنسفورمر با مدل‌های مبتنی بر گرادیان مانند \lr{XGBoost}) می‌تواند دیدگاه‌های جدیدی برای بهبود عملکرد ارائه دهد.
    \item \textbf{ارزیابی با دیتاست‌های متنوع}: آزمایش مدل روی دیتاست‌های متنوع‌تر (مانند \lr{CICMalDroid} \cite{CICMalDroid} یا \lr{VirusShare} ) می‌تواند توانایی تعمیم‌پذیری مدل را بهتر نشان دهد.
\end{itemize}

در نهایت، مدل MAGNET با ارائه یک رویکرد چندوجهی و کارآمد، پتانسیل بالایی برای تشخیص بدافزار اندروید نشان داد. این پژوهش می‌تواند پایه‌ای برای توسعه سیستم‌های امنیتی پیشرفته‌تر و تحقیقات آینده در حوزه امنیت سایبری فراهم آورد.
