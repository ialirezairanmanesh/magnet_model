%% !TEX root=SBUKThesis-main.tex
%% !TEX TS-program = XeLaTeX
%% Last version 1395.11.7
% نمونه رساله دکتری دانشگاه شهید باهنر کرمان (SBUK)
% فرید صابری موحد، گروه ریاضی دانشگاه تحصیلات تکمیلی صنعتی و فناوری پیشرفته
%نظرات اصلاحی همه عزیزان را در ارتباط با قالب پایان نامه طراحی شده به دیده منت می‌نهیم و با کمال سپاس و قدردانی مورد استفاده قرار خواهیم داد (fdsaberi@gmail.com).
%%%%%%%%%%%%%%%%%%%%%%%%%%%%%%%%%%%%%%%%
% تقدیم به آقای دکتر سید احمد موسوی که در زمینه رشد و ترویج نرم افزار لاتک در استان کرمان زحمات بسیار بسیار زیادی را کشیدند.
% با تشکر از تمام اعضای فعال گروه پارسی‌لاتک http://www.parsilatex.com بویژه آقای دکتر وفا خلیقی، که زحمات بسیاری در زمینه رشد و توسعه علم حروف چینی لاتک در ایران کشیده اند.
%%%%%%%%%%%%%%%%%%%%%%%%%%%%%%%%%%%%%%%%%%
\documentclass[a4paper,oneside,12pt]{report}
%در صورت xنیاز به فراخوانی بسته‌های اضافی، آن‌ها را در فایل commands.tex و قبل از
%بسته زی‌پرشین فراخوانی کنید.
% !TeX root=SBUKThesis-main.tex
\usepackage{natbib}	
\usepackage{longtable}
\usepackage{setspace}
%بسته های برای تایپ متون ریاضی و محیط های شماره دار
\usepackage{amsmath}
\usepackage{amsthm}
\usepackage{amssymb}
\usepackage{mathtools}
\usepackage[shortlabels]{enumitem}
%%%%%%%%%%%%%%%%%%%%%%%%%%%%%%%%%%%%%%%%%%%%%%%%%%%
% بسته‌ای برای تنطیم حاشیه‌های بالا، پایین، چپ و راست صفحه
\usepackage[top=30mm, bottom=30mm, left=30mm, right=30mm]{geometry}
\usepackage[hyphens]{url}
%%%%%%%%%%%%%%%%%%%%%%%%%%%%%%%%%%%%%%%%%%%%%%%%%%%
% بسته ای برای رسم انواع مختلف نمودارها
%\usepackage{tikz}
%%%%%%%%%%%%%%%%%%%%%%%%%%%%%%%%%%%%%%%%%%%%%%%%%%%
% بسته‌‌ای برای ظاهر شدن شکل‌ها و تصاویر متن
\usepackage{graphicx}
\usepackage{svg}
%%%%%%%%%%%%%%%%%%%%%%%%%%%%%%%%%%%%%%%%%%%%%%%%%%%
% بسته‌ای برای رسم کادر
\usepackage{framed}
%%%%%%%%%%%%%%%%%%%%%%%%%%%%%%%%%%%%%%%%%%%%%%%%%%%
%برای ایجاد فرورفتگی در ابتدای پاراگراف‌ها
\usepackage{indentfirst}
%%%%%%%%%%%%%%%%%%%%%%%%%%%%%%%%%%%%%%%%%%%%%%%%%%%
%بسته ای برای مدیریت و طراحی ساختار فهرست مطالب، فهرست جداول، فهرست شکل ها و ...
\usepackage{tocloft}
%%%%%%%%%%%%%%%%%%%%%%%%%%%%%%%%%%%%%%%%%%%%%%%%%%%
% برای مدیریت بر ساختار عناوین، هدر و ... مربوط به فصل، بخش و ...
\usepackage{emptypage}
\usepackage{titlesec}
%%%%%%%%%%%%%%%%%%%%%%%%%%%%%%%%%%%%%%%%%%%%%%%%%%%
% مورد استفاده در تعاریف با استفاده از newcommand
\usepackage{xspace}
%%%%%%%%%%%%%%%%%%%%%%%%%%%%%%%%%%%%%%%%%%%%%%%%%%%
%برای حروف چینی محیط شعر پارسی
\usepackage{bidipoem}
%%%%%%%%%%%%%%%%%%%%%%%%%%%%%%%%%%%%%%%%%%%%%%%%%%%
%%%% بسته‌ و دستوراتی برای ایجاد لینک‌های رنگی با امکان جهش
\usepackage[colorlinks=true,linkcolor=Blue,urlcolor=Blue,citecolor=Blue]{hyperref}
%%%%%%%%%%%%%%%%%%%%%%%%%%%%%%%%%%%%%%%%%%%%%%%%%%%
%بسته های مربوط به ایجاد الگوریتم
\usepackage{algorithm,algcompatible}
%%%%%%%%%%%%%%%%%%%%%%%%%%%%%%%%%%%%%%%%%%%%%%%%%%%
%بسته هایی برای مدیریت کپشن در محیط های شناور مانند شکل و جدول و ...
\usepackage[textfont={small},labelfont={rm,small},format=hang,labelsep=quad,justification={centering},aboveskip=1pt,belowskip=1pt]{caption}
\usepackage{subcaption}
%%%%%%%%%%%%%%%%%%%%%%%%%%%%%%%%%%%%%%%%%%%%%%%%%%%
% بسته‌ای برای ظاهر شدن نمایه و مراجع در فهرست
\usepackage[nottoc,notlof,notlot]{tocbibind}
\usepackage{appendix}
%%%%%%%%%%%%%%%%%%%%%%%%%%%%%%%%%%%%%%%%%%%%%%%%%%%
% بسته مورد نیاز برای نوشتن کدهای برنامه نویسی در نوشتار و رنگ
\usepackage{listings}
\usepackage{color}
\definecolor{Blue}{rgb}{0,0,.55}
%%تنظیمات مورد نظر خروجی بسته listings
\lstset{
	tabsize=6,%4
	rulecolor=,
	language=matlab,
	basicstyle=\normalfont,%\scriptsize,
	aboveskip={1.5\baselineskip},
	columns=fixed,
	showstringspaces=false,
	extendedchars=true,
	breaklines=true,
	prebreak = \raisebox{0ex}[0ex][0ex]{\ensuremath{\hookleftarrow}},
	showtabs=false,
	showspaces=false,
	showstringspaces=false,
	identifierstyle=\ttfamily,
	keywordstyle=\color[rgb]{0,0,1},
	commentstyle=\color[rgb]{0.133,0.545,0.133},
	stringstyle=\color[rgb]{0.627,0.126,0.941},
	numbers=left,
numbersep=6pt,
	numberstyle=\footnotesize,%\normalfont,%\tiny,
	frame=l,
tabsize=1
}
%%%%%%%%%%%%%%%%%%
\usepackage{lscape}
\usepackage{array}
\usepackage{courier}
\usepackage{multirow}
\usepackage{array}
\usepackage{float}
\usepackage{hyperref}
\usepackage[table]{xcolor}
\definecolor{mybluecolor}{HTML}{80C4E9}
\usepackage[most, breakable]{tcolorbox}
%%%%%%%%%%%%%%%%%%



% فراخوانی بسته زی‌پرشین و تعریف قلم فارسی و انگلیسی
\usepackage[extrafootnotefeatures]{xepersian}
\settextfont[Scale=1]{XB Zar}
\setlatintextfont[Scale=0.91]{Times New Roman}
% تعریف قلم‌های فارسی و انگلیسی اضافی برای استفاده در بعضی از قسمت‌های متن
\defpersianfont\nastaliq[Scale=1]{IranNastaliq}
\defpersianfont\titr[Scale=1.3]{XB Zar}
% برای نوشتن انگلیسی اعداد در فرمولها
%\DefaultMathsDigits
% برای شکستن فرمولهای داخل محیط align
\allowdisplaybreaks
%%%%%%%%%%%%%%%%%%%%%%%%%%%%%%%%%%%%%%%%%%%%%%%%%%%%%%%%%%%%%%%%%%%%%%%%%%%%%
%جهت داشتن زیرنویس به صورت شماره گذاری شده، صفحه به صفحه
\usepackage{perpage}
\MakePerPage{footnote}
%%%%%%%%%%%%%%%%%%%%%%%%%%%%%%%%%%%%%%%%%%%%%%%%%%%%%%%%%%%%%%%%%%%%%%%%%%%%%
\SepMark{-}
% تعریف و نحوه ظاهر شدن عنوان قضیه‌ها، تعریف‌ها، مثال‌ها و ...
\newtheorem{theorem}{قضیه}[section]
\newtheorem{lemma}[theorem]{لم}
\newtheorem{proposition}[theorem]{گزاره}
\newtheorem{corollary}[theorem]{نتیجه}
\newtheorem{hokm}[theorem]{حکم}
\theoremstyle{definition}
\newtheorem{remark}[theorem]{ملاحظه}
\newtheorem{example}[theorem]{مثال}
\newtheorem{definition}[theorem]{تعریف}
\newtheorem{problem}[theorem]{مسأله}
\newtheorem{taz}{تذکر}
\newtheorem{nok}{نکته}
%%%%%%%%%%%%%%%%%%%%%%%%%%%%%%%%%%%%%%%%%%%%%%%%%%%%%%%%%%%%%%%%%%%%%%%%%%%%%%%
%بکارگیری چند تعریف ساده
\newcommand{\rr}{\mathbb{R}}
\newcommand{\kk}{\mathcal{K}}
\renewcommand{\vec}{\mathrm{vec}}
\newcommand{\seq}[1]{\langle#1\rangle}
\DeclareMathOperator{\tr}{trace}
\DeclareMathOperator{\Log}{Log}
%%%%%%%%%%%%%%%%%%%%%%%%%%%%%%
\renewcommand{\bibname}{منابع و مآخذ}
\newcommand{\danesh}{دانشگاه شهید باهنر کرمان\xspace}
%%%%%%%%%%%%%%%%%%%%%%%%%%%%%%%%%
%برای مدیریت فاصله ها، مثل فاصله خطوط
\renewcommand{\baselinestretch}{1.6}
% فولدر شامل تصاویر را برای تک شناسایی می کند.
\graphicspath{{images/}}
% دستوری برای تعریف واژه‌نامه انگلیسی به فارسی
\newcommand\persiangloss[2]{#1\dotfill\lr{#2}\\}
% دستوری برای تعریف واژه‌نامه فارسی به انگلیسی
\newcommand\englishgloss[2]{#2\dotfill\lr{#1}\\}
%%%%%%%%%%%%%%%%%%%%%%%%%%%%%%%%%%%%%%%%%
% زیاد کردن عمق شماره‌گذاری‌ها در متن و فهرست مطالب
\setcounter{secnumdepth}{4}
\setcounter{tocdepth}{4}
%%%%%%%%%%%%%%%%%%%%%%%%%%%%%%%%%%%%%%%%%
\makeatletter
\newcommand*{\@thechapapp}{\@tartibi\c@chapter}
\bidi@appto\appendix{\gdef\@thechapapp{\@harfi\c@chapter}}
\bidi@patchcmd{\Hy@org@chapter}{%
	\addcontentsline{toc}{chapter}%
	{\protect\numberline{\thechapter}#1}%
}{%
\addcontentsline{toc}{chapter}%
{\protect\numberline{\@chapapp~\@thechapapp:}#1}%
}{\typeout{We succeded in redefining \string\@chapter}}
{\typeout{We failed in redefining \string\@chapter}}
\makeatletter
\setlength\cftchapnumwidth{6.5em}
\setlength\cftsecnumwidth{4em}
\setlength\cftsubsecnumwidth{5em}
\setlength\cftsubsubsecnumwidth{6em}
%%%%%%%%%%%%%%%%%%%%%%%%%%%%%%%%%%%%%%%%%%%%%%%%%%%%%%%%%%%%%%%%%%%%%%%%%
%دستوراتی برای وسط چین شدن عبارات فهرست مطالب، فهرست جداول، فهرست شکلها
\renewcommand{\contentsname}{\hfill فهرست مطالب \hfill\hfill}
\renewcommand{\listtablename}{\hfill فهرست جداول \hfill}
\renewcommand{\listfigurename}{\hfill فهرست اشکال \hfill}
%%%%%%%%%%%%%%%%%%%%%%%%%%%%%%%%%%%%%%%%%%%%%%%%%%%%%%%%%%%%%%%%%%%%%
% تعیین اندازه عبارات "فهرست مطالب"، "فهرست تصاویر" و "فهرست جداول"
\renewcommand{\cfttoctitlefont}{\fontsize{15pt}{15pt}\selectfont\bfseries} %% فهرست مطالب
\renewcommand{\cftaftertoctitle}{}

\renewcommand{\cftloftitlefont}{\fontsize{15pt}{15pt}\selectfont\bfseries} %% فهرست اشکال
\renewcommand{\cftafterloftitle}{}

\renewcommand{\cftlottitlefont}{\fontsize{15pt}{15pt}\selectfont\bfseries} %% فهرست جداول
\renewcommand{\cftafterlottitle}{}
%فهرست الگوریتم ها
\renewcommand{\listalgorithmname}{\vspace{-2.7cm}\fontsize{15}{16}\selectfont\bfseries\centering فهرست الگوریتم‌ها}
\renewcommand{\thealgorithm}{\arabic{chapter}\@SepMark\arabic{algorithm}}
%%%%%%%%%%%%%%%%%%%%%%%%%%%%%%%
\addtocontents{toc}{\vspace{-1.5cm}\textbf{عنوان}~\hfill\textbf{صفحه}\vskip 1mm
\hrule height 1.5pt\vspace{.25cm}}

\addtocontents{lot}{\vspace{-1.5cm}\textbf{عنوان}~\hfill\textbf{صفحه}\vskip 1mm
\hrule height 1.5pt\vspace{.25cm}}

\addtocontents{lof}{\vspace{-1.5cm}\textbf{عنوان}~\hfill\textbf{صفحه}\vskip 1mm
\hrule height 1.5pt\vspace{.25cm}}

\addtocontents{loa}{\vspace{-1.5cm}\textbf{عنوان}~\hfill\textbf{صفحه}\vskip 1mm
\hrule height 1.5pt\vspace{.25cm}}
%%%%%%%%%%%%%%%%%%%%%%%%%%%%%%%%%%%%%%%%%
 \setlength{\cftbeforetoctitleskip}{-.68cm} %% فهرست مطالب
\setlength{\cftbeforelottitleskip}{-.68cm} %% فهرست جداول
\setlength{\cftbeforeloftitleskip}{-.68cm} %% فهرست اشکال

\makeatletter
\def\@myabjad#1{\ifcase#1\or الف \or ب \or ج \or د \or ه \or و \or ز \or ح \or ط \or ی \or ک \or ل \or م \or ن \or س \or ع \or ف \or ص \or ق \or ر \or ش \or ت \or ث\else\@ctrerr\fi}
\def\myabjad#1{\expandafter\@myabjad\csname c@#1\endcsname}
\makeatother
\usepackage{threeparttable}
%%%%%%%%%%%%%%%%%%%%%%%%%%%%5
%\PersianMathsDigits
\usepackage[backend=biber,style=numeric,sorting=none]{biblatex}
\addbibresource{SBUKThesis-bibliography.bib}

% تنظیمات زی‌پرشین
% \usepackage[extrafootnotefeatures]{xepersian}
% \settextfont[Scale=1.2]{XB Zar}
% \setdigitfont[Scale=1.2]{XB Zar}

\begin{document}
\pagenumbering{myabjad}
%در صورتی که می خواهید دو صفحه خالی در ابتدای پایان نامه باشد، دو خط زیر فعال شود
%%% !TeX root=GUATThesis-main.tex
%% !TEX TS-program = XeLaTeX
%\newpage\null\thispagestyle{empty}\newpage
%\cleardoublepage
% !TeX root=SBUKThesis-main.tex
\thispagestyle{empty}
\begin{center}
\vspace*{8cm}
{\fontsize{24}{28}\selectfont\bfseries
بسم الله الرحمن الرحیم
}
\end{center}
\newpage
%صفحه بسم الله الرحمن الرحیم
% !TeX root=SBUKThesis-main.tex
\setlength{\parindent}{0pt}
\begin{center}
\includegraphics[height=3cm]{logo} \\
{\fontsize{14}{15}\selectfont \textbf{دانشکده فنی و مهندسی }} \\
{\fontsize{14}{15}\selectfont\textbf{بخش مهندسی کامپیوتر }} \\
\vskip.7cm

 {\fontsize{14}{15}\selectfont\bfseries
پایان نامه تحصیلی برای دریافت درجه کارشناسی ارشد
\\[3mm]
رشته مهندسی کامپیوتر گرایش هوش مصنوعی
}
\vskip 1.5cm
\hrule height 1.5pt
\vskip .3mm
\hrule height 1.15pt
\par
\begin{center}
\fontsize{16}{17}\selectfont\bfseries
تشخیص قدرتمند بدافزار‌های اندروید با استفاده از شبکه‌های عصبی ترنسفورمر
\end{center}
\par
\hrule height 1.5pt
\vskip .3mm
\hrule height 1.15pt
\par\vskip 2cm
{\fontsize{16}{17}\selectfont\bfseries
مؤلف:}
\\
{\fontsize{14}{15}\selectfont\bfseries
علیرضا ایرانمنش}
\par \vskip 1cm
{\fontsize{16}{17}\selectfont\bfseries
	 استاد راهنما:}
\\
{\fontsize{14}{15}\selectfont\bfseries
	دکتر حمید میروزیری
}
% \par \vskip 1cm
% {\fontsize{16}{17}\selectfont\bfseries
% 	استاد مشاور:}
% \\
% {\fontsize{14}{15}\selectfont\bfseries
% 	دکتر سحر وحدتی
% }
\par\vskip 1cm
{\fontsize{16}{17}\selectfont\bfseries}

{\fontsize{14}{15}\selectfont\bfseries }
\par\vskip 1cm
{\fontsize{14}{15}\selectfont\bfseries
اردیبهشت 1404}
\end{center} %صفحه جلد فارسی
%% !TeX root=SBUKThesis-main.tex
\begin{center}

\begin{figure}[!ht]
	\centering
	\includegraphics[width=170mm]{images/sign}
\end{figure}
\end{center}
\setlength{\parindent}{0pt}%صفحه امضا تیم داوری
{\nastaliq\bfseries \fontsize{10}{11}\selectfont
\begin{center}
به نام خدا
	\vskip 2mm
منشور اخلاق پژوهش\\
	\vskip 2mm
با استعانت از خدای سبحان و اعتقاد راسخ به این که عالم محضر خداست و او همواره ناظر بر اعمال ماست و به منظور انجام شایسته‌ی پژوهش‌های اصیل، تولید دانش جدید و بهسازی زندگانی بشر، ما دانشجویان و اعضای هیأت علمی دانشگاه‌ها و پژوهشگاه‌های کشور:
\end{center}
\begin{itemize}
\item[$\square$]
تمام تلاش خود را برای کشف حقیقت و فقط حقیقت به کار خواهیم بست و از هر گونه جعل و تحریف در فعالیت های علمی پرهیز می‌کنیم.
\item[$\square$]
حقوق پژوهشگران، پژوهیدگان (انسان، حیوان، نبات و اشیاء)، سازمان‌ها و سایر صاحبان حق را به رسمیت می شناسیم و در حفظ آن می‌کوشیم.
\item[$\square$]
به مالکیت مادي و معنوي آثار پژوهشی ارج می نهیم، برای انجام پژوهشی اصیل اهتمام ورزیده و از سرقت علمی و ارجاع نامناسب اجتناب می کنیم.
\item[$\square$]
ضمن پایبندی به انصاف و اجتباب از هر گونه تبعیض و تعصب در کلیه فعالیت‌های پژوهشی، رهیافتی نقادانه اتخاذ خواهیم کرد.
\item[$\square$]
ضمن امانت‌داری، از منابع و امکانات اقتصادی، انسانی و فنی موجود، استفاده بهره‌ورانه خواهیم کرد.
\item[$\square$]
از انتشار غیراخلاقی نتایج پژوهش، نظیر انتشار موازی، همپوشان و چندگانه (تکه‌ای) پرهیز می‌کنیم.
\item[$\square$]
اصل محرمانه بودن و رازداری را محور تمام فعالیت‌های پژوهشی خود قرار می‌دهیم.
\item[$\square$]
در همه فعالیت‌های پژوهشی به منافع ملی توجه کرده و برای تحقق آن می‌کوشیم.
\item[$\square$]
خویش را ملزم به رعایت کلیه هنجارهای علمی رشته خود، قوانین و مقررات، سیاست‌های حرفه‌ای، سازمانی، دولتی و راهبردهای ملی در همه مراحل پژوهش می‌دانیم.
\item[$\square$]
رعایت اصول اخلاق در پژوهش را اقدامی فرهنگی می‌دانیم و به منظور بالندگی این فرهنگ، به ترویج و اشاعه آن در جامعه اهتمام می‌ورزیم.
\end{itemize}
}%منشور اخلاق پژوهش
\begin{center}
\includegraphics[width=2cm]{logo}
\vskip -3mm
{\bfseries \fontsize{10}{11}\selectfont
	تعهدنامه}
\end{center}
{\fontsize{11}{12}\selectfont
اینجانب علیرضا ایرانمنش به شماره دانشجویی ۴۰۱۱۵۵۰۱۵ دانشجوی مقطع کارشناسی‌ ارشد رشته مهندسی کامپیوتر-هوش مصنوعی دانشکده فنی مهندسی دانشگاه شهید باهنر کرمان نویسنده پایان‌نامه با عنوان «تشخیص قدرتمند بدافزار‌های اندروید با استفاده از شبکه‌های عصبی ترنسفورمر» تحت راهنمایی دکتر حمید میروزیری تأیید می‌کنم که این پایان‌نامه نتیجه پژوهش اینجانب می‌باشد و در عین حال که موضوع آن تکراری نیست، در صورت استفاده از منابع دیگران، نشانی دقیق و مشخصات کامل آن درج شده است. همچنین موارد زیر را نیز تعهد می‌کنم:  \
1- برای انتشار تمام یا قسمتی از داده\/ها یا دستاوردهای ‌‌‌ خود در مجامع و رسانه\/های علمی اعم از همایش\/ها و مجلات داخلی و خارجی به صورت مقاله، کتاب، ثبت اختراع و .... به صورت مکتوب یا غیرمکتوب، با کسب مجوز از دانشگاه شهید باهنر کرمان و استاد(ان) راهنما اقدام نمایم. \\
2- از درج اسامی افراد خارج از کمیته پایان\/نامه در جمع نویسندگان مقاله\/های مستخرج از پایان‌نامه، بدون مجوز استاد(ان) راهنما اجتناب نمایم و اسامی افراد کمیته پابان نامه را در جمع نویسندگان مقاله درج نمایم.\\
3- از درج نشانی یا وابستگی کاری \lr{(affiliation)} نویسندگان سازمان\/های دیگر (غیر از دانشگاه شهید باهنر کرمان) در مقاله\/های مستخرج از پایان\/نامه بدون تأیید استاد(دان) راهنما اجتناب نمایم
\footnote {
	تنها آدرس مورد قبول برای دانشگاه به این صورت می باشد:
\begin{LTR}
	Shahid Bahonar University of Kerman, Kerman, Iran.
\end{LTR}
	\begin{RTL}
نام و آدرس واحدهای دانشگاه در تولیدات علمی محققان دانشگاه به تشخیص بخش و دانشکده به شرح زیر می باشد:
\end{RTL}
\begin{LTR}
	Department of Computer, Faculty of Engineering Shahid Bahonar University of Kerman, Kerman, Iran.
\end{LTR}
آدرس صحیح جهت درج در مقالات و سایر تولیدات علمی فارسی: 
\begin{RTL}
گروه (بخش) کامپیوتر، دانشکده فنی مهندسی، دانشگاه شهید باهنر کرمان، کرمان، ایران.
\end{RTL}
 }.\\
4- کلیه ضوابط و اصول اخلاقی مربوط به استفاده از موجودات زنده یا بافتهای آنها را برای انجام پایان‌نامه رعایت نمایم.\\
5- در صورت اثبات تخلف (در هر زمان) مدرک تحصیلی صادر شده توسط دانشگاه شهید باهنر کرمان از درجه اعتبار ساقط و اینجانب هیچ\/گونه ادعایی نخواهم داشت.\\
کلیه حقوق مادی و معنوی این اثر (مقالات مستخرج، برنامه های رایانه ای، نرم افزارها و تجهیزات ساخته شده) مطابق با آیین\/نامه مالکیت فکری، متعلق به دانشگاه شهید باهنر کرمان است و بدون اخذ اجازه کتبی از دانشگاه قابل واگذاری به شخص ثالث نیست. همچنین استفاده از اطلاعات و نتایج این پایان‌نامه بدون ذکر مرجع مجاز نمی باشد. چنانچه مبادرت به عملی خلاف این تعهدنامه محرز گردد، دانشگاه شهید باهنر کرمان در هر زمان و به هر نحو مقتضی حق هرگونه اقدام قانونی را در استیفای حقوق خود دارد.
\begin{figure}[!ht]
	\vspace{-5mm}
	\includegraphics[width=13mm]{images/sign2}
	\vspace{-2mm}
	\parbox[l]{.6\linewidth}{تاریخ و امضا:
		
		علیرضا ایرانمنش
		
		اردیبهشت 1404}
\end{figure}

}%صفحه تایید پایان نامه
% !TeX root=SBUKThesis-main.tex
\chapter*{\vspace{-2.38cm}\fontsize{15}{16}\selectfont تقدیم به:}
«این مجموعه را باکمال افتخار و احترام تقدیم می‌کنم به:\\
روح بلند بنیانگذار دانشگاه، مرحوم 
\textbf{افضلی‌پور}
 و همسر گرامیشان 
\textbf{ بانو فاخره صبا}.\\
آنان که عاشقانه سوختند تا گرمابخش وجود ما و روشنگر راهمان باشند...\\
به پاس تعبیر عظیم و انسانی‌شان از کلمه ایثار.\\
به پاس عاطفه سرشار و گرمای امیدبخش وجودشان که در این سردترین روزگاران بهترین پشتیبان ماست.\\
به پاس قلب بزرگشان، و به پاس محبت‌های بی دریغشان که هرگز فروکش نمی‌کند»







%صفحه تقدیم به
\chapter*{\vspace{-2.38cm}\fontsize{15}{16}\selectfont تشکر و قدردانی:}

با سپاس از خداوند بزرگ که به من توانایی و انگیزه برای پیمودن این مسیر علمی را عطا نمود. این پایان‌نامه حاصل تلاش و کوشش‌های فراوان است و بدون حمایت و راهنمایی‌های ارزشمند افراد بسیاری به ثمر نمی‌نشست.

به مصداق شعر «به یاد کسی که در این راه بود، به یاد کسی که در این راه رفت»، شایسته می‌دانم مراتب سپاس و قدردانی صمیمانه خود را تقدیم نمایم به استاد فرهیخته و فرزانه، جناب آقای دکتر حمید میروزیری، که با دانش و تجربه‌ی خود همواره راهنمای من بودند و با صبر و شکیبایی به سوالات و ابهاماتم پاسخ دادند، صمیمانه تشکر می‌کنم.

همچنین از دوست عزیزم، محمدحسین شبانی، که با حمایت‌های بی‌دریغ و تشویق‌های همیشگی‌اش، انگیزه و انرژی مضاعفی به من بخشید، قدردانی می‌نمایم.

در پایان، از خانواده‌ی عزیزم که با عشق و محبت بی‌پایان خود همواره پشتیبان من بودند و در تمامی مراحل این مسیر پرچالش، همراه و همدل من بودند، بی‌نهایت سپاسگزارم.

%صفحه تشکر و قدردانی 
% !TeX root=SBUKThesis-main.tex
\chapter*{\vspace{-2.38cm}\fontsize{15}{16}\selectfont چکیده:}
\vspace{-1.5cm}\setlength{\parindent}{20pt}
تشخیص بدافزارهای اندرویدی با افزایش روزافزون تهدیدات سایبری، یکی از چالش‌های اصلی در حوزه امنیت اطلاعات به شمار می‌رود. روش‌های سنتی، به‌ویژه آن‌هایی که صرفاً بر تحلیل ویژگی‌های تک‌وجهی تکیه دارند، اغلب با محدودیت‌هایی نظیر ناتوانی در پردازش داده‌های پیچیده چندوجهی و تعمیم‌پذیری ضعیف در برابر تهدیدات جدید مواجه‌اند. این کاستی‌ها، نیاز به توسعه رویکردهای نوین و کارآمد را بیش از پیش آشکار می‌سازد. این پژوهش مدلی چندوجهی با عنوان تبدیل‌گر چندوجهی مبتنی بر جاسازی گراف دینامیک با توجه پویا (MAGNET) توسعه داد که از ترکیب داده‌های جدولی، گراف و ترتیبی، نظیر توالی فراخوانی‌های ،API برای شناسایی بدافزارهای اندرویدی بهره برد. هدف اصلی، بهبود دقت و پایداری تشخیص با استفاده از معماری پیشرفته مبتنی بر یادگیری عمیق و ترنسفورمر بود. روش تحقیق شامل بهینه‌سازی هایپرپارامترها با الگوریتم‌های پیشرفته مانند PIRATES و Optuna، آموزش مدل با مجموعه داده‌ای شامل 4641 نمونه آموزشی و 1451 نمونه آزمایشی، و اعتبارسنجی متقاطع 5-تایی شد. ویژگی‌های مورد استفاده شامل ویژگی‌های ایستا مانند مجوزها، فراخوانی‌های API، مقاصد، و نام‌های مؤلفه و ویژگی‌های پویا مانند فعالیت شبکه و دسترسی به فایل‌ها بود. داده‌ها به صورت بردارهای عددی باینری یا نرمال‌سازی شده بودند. ابعاد ویژگی‌ها پس از پیش‌پردازش به 430 ویژگی تنظیم شد. ابزارهای مورد استفاده شامل کتابخانه‌های یادگیری عمیق مانند PyTorch، تکنیک‌های پیش‌پردازش داده‌ها نظیر استانداردسازی و نرمال‌سازی، و ساختارهای داده‌ای گرافی بودند. مواد اولیه شامل داده‌های واقعی از رفتار اپلیکیشن‌های اندرویدی، شامل ویژگی‌های ایستا و پویا، بود که با دقت آماده‌سازی شدند. نتایج نشان داد که مدل پیشنهادی عملکردی برجسته با دقت بالا، پایداری قابل‌توجه و قابلیت تعمیم‌پذیری خوب ارائه کرد و نسبت به روش‌های پیشین بهبود قابل‌ملاحظه‌ای داشت. این دستاوردها پتانسیل کاربرد مدل در سیستم‌های امنیتی واقعی را برجسته ساخت. پیشنهاد می‌شود تحقیقات آینده بر افزایش حجم داده‌ها، ادغام روش‌های خودنظارتی پیشرفته، آزمایش مدل در محیط‌های متنوع‌تر و بهینه‌سازی زمان اجرای آن متمرکز شوند تا کارایی مدل در سناریوهای پیچیده‌تر و واقعی‌تر ارتقا یابد. همچنین، بررسی تأثیر ترکیب داده‌های جدیدتر و توسعه الگوریتم‌های مقاوم در برابر حملات مخرب می‌تواند مسیرهای نوینی برای تحقیقات بعدی گشوده کند. این پژوهش، گامی مؤثر در راستای ارتقای سیستم‌های تشخیص خودکار بدافزارها برداشت و پایه‌ای محکم برای توسعه راه‌حل‌های امنیتی پیشرفته‌تر فراهم آورد.

\par\vspace{.5cm}\setlength{\parindent}{0pt}
{\bf
واژگان کلیدی: تشخیص بدافزار، یادگیری عمیق، داده‌های چندوجهی، امنیت اندروید، ترنسفورمر.

} 

 
%صفحه چکیده فارسی
%%%%%%%%%%%%%%%%%%%%%%%%%%%
\tableofcontents%فهرست منابع
\cleardoublepage
\setlength{\cftfignumwidth}{2.5em}
%فهرست جداول
{
	\let\oldnumberline\numberline%
	\renewcommand{\numberline}{\tablename~\oldnumberline}%
\listoftables%
}
\cleardoublepage
%\listoffigures%
{
	\let\oldnumberline\numberline%
	\renewcommand{\numberline}{\figurename~\oldnumberline}%
	\listoffigures%
}
\cleardoublepage
%فهرست الگوریتم ها
{
	\let\oldnumberline\numberline%
	\renewcommand{\numberline}{الگوریتم~\oldnumberline}%
\listofalgorithms%
}
\cleardoublepage
%\chapter*{\vspace{-2.38cm}\centering\bfseries\fontsize{15}{16}\selectfont فهرست علایم 	اختصاری
\vspace{0.75cm}\hrule height 1.5pt \vspace{-.75cm}}
\persiangloss{ میدان اعداد حقیقی و مختلط}{$\mathbb{R}$, $\mathbb{C}$}
\persiangloss{مجموعه بردارهای ستونی و حقیقی $n\times 1$}{$\mathbb{R}^n$}%فهرست علائم اختصاری
\cleardoublepage
%\chapter*{\vspace{-2.38cm}\centering\bfseries\fontsize{15}{16}\selectfont فهرست کلمات اختصاری
\vspace{0.75cm}\hrule height 1.5pt \vspace{-.75cm}}

\persiangloss{ماشین یادگیری سریع}{ELM}
\persiangloss{روش باقیمانده کمینه سراسری}{Gl-MR}
%فهرست کلمات اختصاری
\cleardoublepage
%%%%%%%%%%%%%%%%%%%%%%%%%%%%%%%%
\pagenumbering{arabic}
\titleformat{\chapter}[display]
{\vspace{5cm}\filcenter}
{\filcenter \LARGE\bfseries
\chaptertitlename\ \justwords{\thechapter}\!:}
{1ex}
{\filcenter\fontsize{24}{25}\selectfont\bfseries\thispagestyle{empty}}
[\vfill\clearpage]
\def\justwords#1{\ifcase#1 \or اول \or دوم \or سوم \or چهارم \or پنجم \or ششم \or هفتم etc.\fi}
%%%%%%%%%%
 \titleformat*{\section}{\bfseries\fontsize{13}{14}\selectfont}
\titleformat*{\subsection}{\bfseries\fontsize{12}{13}\selectfont}
\titleformat*{\subsubsection}{\bfseries\fontsize{11}{12}\selectfont}
%%%%%%%%%%%%%%%%%%%%%%%%%%%%%%%%
\setlength{\parindent}{20pt}
@inbook{ID,
	author = {author},
	title = {title},
	booktitle = {booktitle},
	date = {date},
	OPTbookauthor = {bookauthor},
	OPTeditor = {editor},
	OPTeditora = {editora},
	OPTeditorb = {editorb},
	OPTeditorc = {editorc},
	OPTtranslator = {translator},
	OPTannotator = {annotator},
	OPTcommentator = {commentator},
	OPTintroduction = {introduction},
	OPTforeword = {foreword},
	OPTafterword = {afterword},
	OPTsubtitle = {subtitle},
	OPTtitleaddon = {titleaddon},
	OPTmaintitle = {maintitle},
	OPTmainsubtitle = {mainsubtitle},
	OPTmaintitleaddon = {maintitleaddon},
	OPTbooksubtitle = {booksubtitle},
	OPTbooktitleaddon = {booktitleaddon},
	OPTlanguage = {language},
	OPToriglanguage = {origlanguage},
	OPTvolume = {volume},
	OPTpart = {part},
	OPTedition = {edition},
	OPTvolumes = {volumes},
	OPTseries = {series},
	OPTnumber = {number},
	OPTnote = {note},
	OPTpublisher = {publisher},
	OPTlocation = {location},
	OPTisbn = {isbn},
	OPTchapter = {chapter},
	OPTpages = {pages},
	OPTaddendum = {addendum},
	OPTpubstate = {pubstate},
	OPTdoi = {doi},
	OPTeprint = {eprint},
	OPTeprintclass = {eprintclass},
	OPTeprinttype = {eprinttype},
	OPTurl = {url},
	OPTurldate = {urldate},
}
% !TeX root=SBUKThesis-main.tex
\clearpage
\thispagestyle{empty}

% Make footnotes left-aligned
\renewcommand{\@makefntext}[1]{%
  \parindent 1em%
  \noindent
  \@makefnmark#1}

\chapter{کلیات پژوهش}\label{chap1}

% \section*{فهرست مطالب}
% \begin{enumerate}
%     \item مقدمه و بیان مسئله
%     \begin{enumerate}
%         \item روش‌های تشخیص بدافزار
%         \item مجموعه داده‌های مربوطه
%     \end{enumerate}
%     \item ضرورت تحقیق و اهداف
%     \item سازماندهی پایان‌نامه
% \end{enumerate}

\section{مقدمه و بیان مسئله}\label{intro}
در سال‌های اخیر، گسترش تلفن‌های همراه و به‌ویژه سیستم‌عامل اندروید\LTRfootnote{\lr{Android}}، موجب افزایش وابستگی کاربران به این ابزارها شده است. این دستگاه‌ها نه تنها در زندگی روزمره، بلکه در حوزه‌های تجاری و نظامی نیز نقش مهمی ایفا می‌کنند. با این حال، محبوبیت و فراگیری اندروید، آن را به هدفی جذاب برای حملات بدافزاری\LTRfootnote{\lr{Malware}} تبدیل کرده است. عرضه نرم‌افزارهای غیرمعتبر و تهدیداتی مانند ویروس‌ها و بدافزارها، امنیت کاربران را به خطر انداخته است. مطالعات اخیر نشان می‌دهد که بیش از \lr{70} درصد دستگاه‌های هوشمند از سیستم‌عامل اندروید استفاده می‌کنند و این امر باعث شده است که این پلتفرم به هدف اصلی حملات امنیتی تبدیل شود \cite{AndroidSecurity}. با وجود پیشرفت‌های قابل توجه در روش‌های تشخیص بدافزار، همچنان چالش‌های جدی در شناسایی بدافزارهای جدید و پیچیده وجود دارد.

در ابتدا، روش‌های سنتی مبتنی بر تحلیل مجوزها\LTRfootnote{\lr{Permissions}} و بازکردن فایل‌ها مورد استفاده قرار می‌گرفتند که به دلیل دقت پایین و ضعف در شناسایی بدافزارهای پیچیده، محدودیت‌هایی داشتند. پژوهش‌های اخیر نشان داده‌اند که روش‌های مبتنی بر یادگیری ماشین\LTRfootnote{\lr{Machine Learning}} و یادگیری عمیق\LTRfootnote{\lr{Deep Learning}} می‌توانند عملکرد بهتری در تشخیص بدافزارها داشته باشند \cite{DeepLearningMalware}. با این حال، همچنان چالش‌های مهمی در زمینه تفسیرپذیری مدل‌ها\LTRfootnote{\lr{Model Interpretability}} و قابلیت تعمیم‌پذیری\LTRfootnote{\lr{Generalization}} وجود دارد. این چالش‌ها به ویژه در مواجهه با بدافزارهای جدید و ناشناخته (\rl{Zero-Day})\LTRfootnote{\lr{Zero-Day}} بیشتر خود را نشان می‌دهند.

مدل \lr{MAGNET}\LTRfootnote{\lr{MAGNET(Multi-Modal Analysis for Graph and Network Threat Detection)}} که در این پژوهش معرفی شده است، با بهره‌گیری از معماری ترنسفورمر\LTRfootnote{\lr{Transformer}} چندوجهی و ترکیب داده‌های جدولی، گراف و توالی، تلاش می‌کند تا این چالش‌ها را برطرف کند. این مدل با استفاده از مکانیزم‌های توجه پویا\LTRfootnote{\lr{Attention Mechanism}} و تحلیل همزمان داده‌های مختلف، قادر به تشخیص دقیق‌تر بدافزارها خواهد بود. نتایج نشان می‌دهد که این رویکرد با دقت \lr{97.24\% ± 0.5\%}، معیار \lr{F1 = 0.9823 ± 0.002}، و معیار \lr{AUC = 0.9932 ± 0.003} عملکرد بهتری نسبت به مدل‌های پایه مانند \lr{SVM} (دقت \lr{90.6\%})، \lr{CNN} (دقت \lr{92.8\%}) و \lr{LSTM} (دقت \lr{91.5\%}) دارد.

\subsection{روش‌های تشخیص بدافزار}
تشخیص بدافزارهای اندرویدی به دو روش کلی پویا\LTRfootnote{\lr{Dynamic Analysis}} و ایستا\LTRfootnote{\lr{Static Analysis}} انجام می‌شود. در روش پویا، رفتار اپلیکیشن در زمان اجرا مانند مصرف باتری، پردازنده یا ترافیک شبکه بررسی می‌شود تا الگوهای غیرعادی شناسایی گردد. این روش به‌تنهایی کافی نیست و ممکن است برخی تهدیدات پنهان را نادیده بگیرد. روش ایستا با تحلیل ساختار و کد اپلیکیشن، مانند بررسی فراخوانی‌های \lr{API}\LTRfootnote{\lr{API}} و مجوزها، اطلاعات ارزشمندی ارائه می‌دهد که می‌تواند در تشخیص دقیق‌تر کمک کند. پژوهش‌های اخیر نشان داده‌اند که ترکیب این دو روش می‌تواند نتایج بهتری در تشخیص بدافزارها ارائه دهد \cite{AndroidMalwareSurvey}.

\subsection{مجموعه داده‌های مربوطه}
در حوزه تشخیص بدافزار اندروید، مجموعه داده‌های متنوعی برای ارزیابی عملکرد مدل‌ها مورد استفاده قرار گرفته‌اند. از جمله این مجموعه داده‌ها می‌توان به موارد زیر اشاره کرد:
\begin{itemize}
    \item مجموعه داده‌های \lr{Drebin} \cite{Drebin} و \lr{AndroZoo} \cite{AndroZoo} که شامل نمونه‌های گسترده‌ای از بدافزارها و برنامه‌های سالم اندرویدی هستند.
    \item مجموعه داده‌های \lr{CICMalDroid} \cite{CICMalDroid} و \lr{VirusShare} که شامل نمونه‌های جدید و به‌روز از بدافزارها می‌باشند.
    \item مجموعه داده‌های خصوصی و صنعتی که توسط شرکت‌های امنیتی و مراکز تحقیقاتی گردآوری شده‌اند.
\end{itemize}
این مجموعه داده‌ها به عنوان شاخص‌های استاندارد، امکان ارزیابی دقیق و جامع عملکرد الگوریتم‌های تشخیص بدافزار را فراهم می‌کنند و نقش مهمی در اثبات قابلیت تعمیم و کارایی روش‌های پیشنهادی دارند.

\section{ضرورت تحقیق و اهداف}\label{import}
پلتفرم اندروید به دلیل محبوبیت گسترده و سهم عظیمش از بازار جهانی، به هدف اصلی بدافزارها و حملات امنیتی تبدیل شده است. این سیستم‌عامل، که بیش از \lr{70} درصد دستگاه‌های هوشمند را پشتیبانی می‌کند، به دلیل ساختار باز و دسترسی‌پذیری بالا، با تهدیدات پیشرفته‌ای مواجه است. بدافزارهای اندرویدی، از جمله تروجان‌ها\LTRfootnote{\lr{Trojan}}، جاسوس‌افزارها\LTRfootnote{\lr{Spyware}} و باج‌افزارها\LTRfootnote{\lr{Ransomware}}، با روش‌های پیچیده‌ای طراحی شده‌اند و پیشرفت‌های چشمگیری داشته‌اند. این تهدیدات، از سرقت اطلاعات حساس گرفته تا ایجاد اختلال در عملکرد دستگاه‌ها، چالش‌های امنیتی جدی ایجاد کرده‌اند. از این رو، نیاز به سیستمی قدرتمند و کارآمد برای تشخیص بدافزارهای اندرویدی بیش از پیش احساس می‌شود. هدف اصلی این پژوهش، تمرکز بر شناسایی بدافزارهای ناشناخته و نادیده (\rl{Zero-Day})\LTRfootnote{\lr{Zero-Day}} است که تا کنون شناسایی نشده‌اند و می‌توانند تهدیداتی پنهان برای کاربران ایجاد کنند.

با توجه به چالش‌ها و نیازهای مطرح شده، اهداف اصلی این تحقیق بدین شرح می‌باشد:
\begin{itemize}
    \item توسعه یک مدل چندوجهی پیشرفته با نام \lr{MAGNET} که قادر به تحلیل همزمان داده‌های جدولی، گرافی و ترتیبی باشد.
    \item بهبود دقت تشخیص بدافزارهای اندرویدی با استفاده از معماری ترنسفورمر و مکانیزم‌های توجه پویا\LTRfootnote{\lr{Attention Mechanism}}.
    \item کاهش نرخ خطای تشخیص و افزایش قابلیت تعمیم‌پذیری مدل در مواجهه با بدافزارهای جدید.
    \item بهینه‌سازی مصرف منابع محاسباتی و افزایش سرعت تشخیص با استفاده از الگوریتم‌های پیشرفته.
    \item ایجاد یک چارچوب استاندارد برای ارزیابی و مقایسه روش‌های مختلف تشخیص بدافزار.
\end{itemize}

\section{سازماندهی پایان نامه}\label{organiz}
در این پایان‌نامه، ساختار مطالب به گونه‌ای تدوین شده که مسیر پژوهش از مبانی نظری و معرفی مسئله تا ارائه نتایج تجربی به صورت پیوسته و منطقی دنبال شود. به عبارت دیگر، هدف از سازماندهی مطالب این است که خواننده بتواند به راحتی با مباحث پایه، چالش‌ها، روش‌های موجود و نوآوری‌های پیشنهادی آشنا شود و در نهایت به درک جامع از دستاوردهای تحقیق دست یابد. ساختار کلی پایان‌نامه به شرح زیر است:
\begin{itemize}
    \item فصل \lr{2} – پیشینه تحقیق و مفاهیم پایه:
    
    در این فصل، ابتدا به بررسی کلی امنیت اندروید و اهمیت تشخیص بدافزار پرداخته می‌شود. سپس، چالش‌ها و محدودیت‌های روش‌های سنتی بیان شده و مسئله تحقیق به تفصیل معرفی می‌شود. هدف این فصل ایجاد زمینه نظری مناسب برای درک اهمیت تشخیص خودکار بدافزارهاست.
    
    در ادامه به بررسی جامع مطالعات پیشین در حوزه تشخیص بدافزار اندروید پرداخته می‌شود. در این بخش، رویکردهای مختلف از جمله روش‌های مبتنی بر یادگیری ماشین و یادگیری عمیق مورد تحلیل قرار می‌گیرند. نقاط قوت و ضعف هر یک از این رویکردها همراه با چالش‌های موجود در هر کدام به تفصیل بررسی می‌شود.
    
    \item فصل \lr{3} – روش پیشنهادی (\lr{MAGNET}):
    
    در این فصل، مدل پیشنهادی \lr{MAGNET} به صورت کامل تشریح می‌شود. ابتدا معماری کلی مدل و اجزای اصلی آن معرفی می‌شوند. سپس، جزئیات پیاده‌سازی و الگوریتم‌های بهینه‌سازی مورد استفاده توضیح داده می‌شود. در نهایت، نوآوری‌های اصلی این روش نسبت به سایر روش‌ها برجسته می‌شود.
    
    \item فصل \lr{4} – نتایج و بحث:
    
    این فصل به ارائه نتایج آزمایش‌های انجام شده بر روی چندین مجموعه داده معتبر اختصاص دارد. عملکرد مدل \lr{MAGNET} از نظر دقت، کارایی و صرفه‌جویی در منابع محاسباتی مورد مقایسه قرار گرفته و نتایج به دست آمده تحلیل می‌شوند.
    
    \item فصل \lr{5} – نتیجه‌گیری و پیشنهادات آتی:
    
    در فصل نهایی، یافته‌های اصلی تحقیق به طور خلاصه ارائه شده و به نتیجه‌گیری کلی از دستاوردهای پژوهش پرداخته می‌شود. در این بخش، چالش‌های باقی‌مانده، محدودیت‌های تحقیق و نیز پیشنهاداتی جهت تحقیقات آتی و بهبود رویکرد ارائه می‌شود.
\end{itemize}


%فصل اول
\cleardoublepage
% !TeX root=SBUKThesis-main.tex
\clearpage
\thispagestyle{empty}
\chapter{پیشینه تحقیق و مفاهیم پایه}\label{chap2}
\section{مقدمه}
مدل‌های زبانی بزرگ بر اساس داده‌های متنی گسترده آموزش دیده و قابلیت درک و تولید زبان طبیعی را دارند. عملکرد این مدل ها بر پایه پیش‌بینی کلمات بعدی در یک جمله یا به عبارتی تکمیل متون بر اساس ورودی داده‌شده است.  

در سال‌های اخیر، مدل‌های زبانی بزرگ رشد چشمگیری داشته\/اند و به دنبال این رشد، ایده استفاده از این مدل ها در تمامی زمینه‌های پردازش زبان طبیعی مورد توجه قرار گرفته است. قابلیت یادگیری مبتنی بر محتوا
\LTRfootnote{In-context Learning}
، که باعث میشود مدل‌ها به ورودی‌های دریافتی دقت ویژه‌ای کنند، کیفیت متن تولید شده جدید را به متن ورودی وابسته میکند. فلذا مهندسی اعلان یکی از اجزای کلیدی در بهبود عملکرد مدل‌های زبانی  است. با طراحی دقیق اعلان‌ها، می‌توان ورودی‌های مدل را به گونه‌ای تنظیم کرد که پاسخ‌های تولیدی با اهداف و نیازهای خاص همخوانی بیشتری داشته باشند. این فرایند نه تنها به بهبود کیفیت و دقت خروجی‌های مدل کمک می‌کند، بلکه در کنترل و هدایت رفتار آن در مواجهه با وظایف مختلف نقش حیاتی دارد. استفاده از مهندسی اعلان زمینه‌ساز تطبیق بهتر مدل با شرایط متغیر و کاهش ابهامات در تولید جواب است.

مسئله اصلی این تحقیق، چالش‌های موجود در طراحی دستی اعلان‌هاست که به دلیل پیچیدگی و زمان‌بر بودن، نیازمند راهکارهایی خودکار می‌باشد. در این راستا، هدف این تحقیق ارائه روشی مبتنی بر الگوریتم جستجو در فضای اعلان‌ها برای تولید خودکار اعلان‌های بهینه است. این رویکرد می‌تواند باعث بهبود دقت و کارایی مدل‌های زبانی شده و از بروز خطاهای ناشی از طراحی دستی جلوگیری کند.

\section{بررسی مدل‌های زبانی بزرگ}
مدل‌های زبانی بزرگ سیستم‌های پیشرفته‌ای هستند که با بهره‌گیری از تکنیک‌های یادگیری عمیق، توانایی پردازش و تولید زبان طبیعی را به سطحی بالا رسانده‌اند. این مدل‌ها با تحلیل حجم عظیمی از داده‌های متنی، قادر به درک مفاهیم، استخراج اطلاعات و تولید متونی دقیق و معنادار می‌باشند. از کاربردهای آن‌ها می‌توان به ترجمه، خلاصه‌سازی، پاسخگویی به سوالات و حتی تولید محتوا در حوزه‌های مختلف اشاره کرد.

تحولات اخیر در این حوزه، مسیر توسعه و بهبود این مدل‌ها را هموار ساخته است. شناخت تاریخچه و معماری این مدل‌ها نقش مهمی در درک عملکرد و پتانسیل‌های آن‌ها دارد که در ادامه به آن می\/پردازیم.

\subsection{تاریخچه و تکامل مدل‌ها}

تکامل مدل‌های زبانی، مسیر پیچیده‌ای از رویکردهای اولیه‌ی مبتنی بر قواعد نمادین تا استفاده از شبکه‌های عصبی پیشرفته و معماری‌های نوین مانند ترنسفورمر را در بر می‌گیرد. در ادامه به تفصیل به بررسی مراحل مختلف این تکامل پرداخته می‌شود.

\textbf{دوران اولیه: رویکردهای نمادین و قواعد دست‌نویس}

\noindent در دهه‌های ۱۹۵۰ و ۱۹۶۰، اولین تلاش‌ها برای پردازش زبان طبیعی به وسیله‌ی روش‌های نمادین انجام شد. پژوهشگران در آن زمان سعی می‌کردند ساختارهای دستوری و قوانین زبان را به صورت صریح و دستی تعریف کنند. این رویکردها با وجود تلاش‌های ارزشمند، به دلیل محدودیت‌های محاسباتی و عدم وجود داده‌های کافی، نتوانستند به دقت و کارایی مورد انتظار دست یابند.

\textbf{ورود به عصر آماری}

\noindent با گذر زمان و ورود به دهه‌های ۱۹۶۰ و ۱۹۷۰، رویکردهای آماری جایگزین بخش‌هایی از روش‌های نمادین شدند. در این دوران، مدل‌های \lr{n-gram} که بر مبنای احتمال وقوع یک کلمه با توجه به کلمات قبلی محاسبه می‌شدند، به عنوان اولین قدم‌های موفق در مدلسازی زبان مطرح شدند. اگرچه این مدل‌ها ساده بودند، اما توانستند برخی از پیچیدگی‌های اولیه‌ی پردازش زبان را کاهش دهند.

\textbf{ظهور یادگیری ماشین و شبکه‌های عصبی}

\noindent در دهه‌های ۱۹۸۰ و ۱۹۹۰، با پیشرفت‌های چشمگیر در فناوری‌های محاسباتی و افزایش دسترسی به داده‌های متنی، روش‌های یادگیری ماشین وارد عرصه شدند. الگوریتم‌های یادگیری نظارت‌شده و غیرنظارتی به منظور تشخیص الگوهای زبانی به کار گرفته شدند. با این حال، محدودیت‌های موجود همچنان مانع از دستیابی به درک عمیق‌تر و تولید متن‌های طبیعی به سطح امروزی می‌شدند.

\textbf{عصر شبکه‌های عصبی عمیق}

\noindent ورود به قرن ۲۱ و به‌ویژه دهه ۲۰۱۰، با ظهور شبکه‌های عصبی عمیق مانند شبکه‌های عصبی بازگشتی
\LTRfootnote{Recursive Neural Networks (RNNs)}
 و شبکه‌های حافظه بلندمدت
\LTRfootnote{Long short-term memory (LSTM)}
  همراه بود. این مدل‌ها توانستند وابستگی‌های زمانی و روابط بلندمدت موجود در متن را بهتر مدل‌سازی کنند. با این حال، چالش‌هایی همچنان در زمینه بهبود کیفیت و کارایی تولید متن وجود داشت.

\textbf{انقلاب ترنسفورمر و ظهور مدل‌های بزرگ}

\noindent نقطه عطف مهم در تکامل مدل‌های زبانی، معرفی معماری ترنسفورمر 
\cite{attention}
 بود. این معماری با بهره‌گیری از مکانیزم توجه
\LTRfootnote{Attention Mechanism}
  توانست وابستگی‌های بین کلمات را به صورت موازی و با کارایی بالا پردازش کند. ویژگی‌های کلیدی ترنسفورمر شامل پردازش موازی داده‌ها، درک بهتر وابستگی‌های طولانی‌مدت در متن و افزایش سرعت و بهبود کارایی مدل‌های زبانی است.

\textbf{توسعه مدل‌های پیشرفته مانند GPT و BERT}

\noindent با معرفی ترنسفورمر، مدل‌های بزرگی نظیر GPT و BERT توسعه یافتند:
\begin{itemize}
	\item GPT
	\LTRfootnote{Generative Pre-trained Transformer}
	: این مدل ها با افزایش تعداد پارامترها (به عنوان مثال، GPT-3 با ۱۷۵ میلیارد پارامتر) توانسته‌اند وظایفی مانند ترجمه، خلاصه‌سازی و پاسخ به سؤالات را با دقت بسیار بالا انجام دهند.
	\item BERT
	\LTRfootnote{Bidirectional Encoder Representations from Transformers}
	: این مدل با تمرکز بر درک بهتر معنایی کلمات در متن، در وظایف مختلف پردازش زبان عملکرد قابل‌توجهی از خود نشان داده است.
\end{itemize}

\textbf{تکنیک‌های بهبود عملکرد: تنظیم دقیق و یادگیری انتقالی}

\noindent علاوه بر افزایش تعداد پارامترها و مقیاس داده‌های آموزشی، تکنیک‌هایی مانند تنظیم دقیق
\LTRfootnote{Fine-tuning}
 و یادگیری انتقالی
\LTRfootnote{Transfer Learning}
  نقش مهمی در بهبود عملکرد مدل‌های زبانی داشته‌اند. این تکنیک‌ها امکان تطبیق مدل‌های پیش‌آموزش داده شده با وظایف خاص را فراهم می‌آورند که باعث بهبود کیفیت و دقت در کاربردهای متنوع می‌شود.


پیشرفت‌های حاصل از توسعه مدل‌های زبانی بزرگ، مرزهای جدیدی در تعامل انسان و ماشین ایجاد کرده است. سیستم‌های هوشمند مبتنی بر این مدل‌ها قادرند که به صورت طبیعی و انسانی با کاربران تعامل کنند، همچنین در زمینه‌های مختلفی از جمله خدمات مشتری، ترجمه ماشینی، تحلیل متون و تولید محتوا به کار گرفته شوندو وظایف پیچیده زبانی را با دقت و سرعت بالا انجام دهند.


تکامل مدل‌های زبانی از روش‌های نمادین اولیه به سوی استفاده از شبکه‌های عصبی عمیق و معماری‌های پیشرفته مانند ترنسفورمر، نشان‌دهنده یک مسیر پرفراز و نشیب اما پر از نوآوری است. این پیشرفت‌ها بهبود قابل‌توجهی در درک و تولید زبان انسانی ایجاد کرده و نقش مهمی در توسعه فناوری‌های هوش مصنوعی و تعامل انسان-ماشین داشته‌اند.


\subsection{معماری‌ها و کاربردهای اصلی}
در ادامه، به تفصیل به بررسی معماری‌های کلیدی این مدل‌ها می‌پردازیم.
\begin{enumerate}
	\item \textbf{معماری ترنسفورمر}
	\LTRfootnote{Transformer}\textbf{:}
	معماری ترنسفورمر پایه و اساس بسیاری از مدل‌های زبانی بزرگ است. این معماری با استفاده از مکانیزم توجه
	\LTRfootnote{Attention Mechanism}
	، امکان پردازش موازی داده‌ها و درک وابستگی‌های طولانی‌مدت در متن را فراهم می‌کند. برخلاف مدل‌های پیشین که بر پایه شبکه‌های عصبی بازگشتی
	\LTRfootnote{Recursive Neural Networks}
	 بودند، ترنسفورمرها با حذف وابستگی‌های ترتیبی، کارایی و سرعت پردازش را بهبود بخشیدند. دیاگرام این معماری در شکل \ref{fig_transformer} نمایش داده شده است.
	 \begin{figure}[!t]
	 	\centering
	 	\includegraphics[width=140mm]{images/transformer}
	 	\caption{دیاگرام معماری ترنسفورمر }
	 	\label{fig_transformer}
	 \end{figure}
	 
	\item \textbf{مکانیزم توجه:}
	مکانیزم توجه به مدل‌ها اجازه می‌دهد تا به بخش‌های مختلف ورودی با وزن‌های متفاوت نگاه کنند و وابستگی‌های معنایی را بهتر درک نمایند. این مکانیزم به‌ویژه در ترجمه ماشینی و خلاصه‌سازی متون کاربرد دارد.
	
	\item \textbf{معماری‌های خودبازگشتی}
	\LTRfootnote{Autoregressive}
	\textbf{ و خودرمزگذار}
	 \LTRfootnote{Autoencoder}\textbf{:}
	\begin{itemize}
		\item 
	مدل‌های خودبازگشتی مانند سری GPT متن را به‌صورت ترتیبی تولید می‌کنند و هر کلمه را بر اساس کلمات قبلی پیش‌بینی می‌نمایند.
		\item 
	مدل‌های خودرمزگذار مانند BERT با استفاده از ماسک‌کردن کلمات در ورودی، سعی در درک زمینه و پیش‌بینی کلمات ماسک‌شده دارند.
	\end{itemize}
\end{enumerate}

\subsection{مدل بزرگ زبانی Mistral}


مدل بزرگ زبانی Mistral \cite{mistral} از جمله دستاوردهای جدید در حوزه پردازش زبان طبیعی به شمار می‌آید که با بهره‌گیری از معماری ترنسفورمر، عملکرد بالایی در وظایف متنوع زبانی نشان داده است. در ادامه به بررسی جامع این مدل می‌پردازیم.

مدل Mistral بر پایه معماری ترنسفورمر طراحی شده است. همانطور که گفته شد، این معماری با استفاده از مکانیزم توجه قادر است وابستگی‌های طولانی‌مدت در متون را به‌صورت موازی پردازش کند. از ویژگی‌های برجسته این مدل می‌توان به استفاده از تعداد پارامترهای بالا
\footnote{در این پژوهش از مدل‌های 7 میلیارد پارامتری استفاده شده است}
 اشاره کرد که موجب بهبود دقت و کیفیت تولید متن می‌شود.

برای رسیدن به عملکرد مطلوب، مدل Mistral با استفاده از مجموعه‌های داده گسترده و متنوع آموزش داده شده است. علاوه بر این، بهره‌گیری از تکنیک‌های تنظیم دقیق
\LTRfootnote{Fine-tuning}
 و بهینه‌سازی پیشرفته، موجب شده تا مدل بتواند در وظایف خاص، همچون ترجمه، خلاصه‌سازی و پاسخ به پرسش، عملکرد بهتری از خود نشان دهد.


مدل Mistral در حوزه‌های مختلف پردازش زبان طبیعی کاربرد دارد. از وظایف مهم آن می\/توان به توانایی تولید متونی با کیفیت بالا و طبیعی، ارائه ترجمه‌های دقیق و روان بین زبان‌ها، استخراج اطلاعات کلیدی و ارائه خلاصه‌های مفید از متون طولانی و همچنین درک دقیق سوالات و ارائه پاسخ‌های مرتبط و دقیق اشاره کرد.


با وجود این توانایی ها، مدل Mistral دارای نوآوری‌ها و مزایای متعددی نیز است که آن را از سایر مدل‌های زبانی متمایز می‌کند. این نوآوری ها شامل بهره‌گیری از معماری ترنسفورمر جهت پردازش موازی و بهبود سرعت محاسبات است، همچنین استفاده از تعداد پارامترهای بالا به افزایش دقت و کیفیت خروجی‌ها منجر می‌شود و قابلیت تنظیم دقیق برای تطبیق با وظایف خاص و کاربردهای صنعتی و پژوهشی را فراهم می آورد و به دنبال آن، بهبود چشمگیری در درک وابستگی‌های زبانی و تولید متون طبیعی حاصل می\/شود.


با وجود دستاوردهای قابل توجه، مدل Mistral همچنان با چالش‌هایی همچون مصرف بالای منابع محاسباتی و نیاز به داده‌های آموزشی گسترده مواجه است. پژوهش‌های آتی در زمینه بهبود کارایی، کاهش هزینه‌های محاسباتی و افزایش دقت در کاربردهای خاص، افق‌های روشن‌تری را برای این مدل ترسیم می‌کند.



\section{مهندسی اعلان}
مهندسی اعلان 
\LTRfootnote{Prompt Engineering}
 به فرآیند طراحی و بهینه‌سازی ورودی‌های متنی اطلاق می‌شود که به مدل‌های زبانی بزرگ ارائه می‌گردد تا خروجی‌های مطلوب و دقیقی تولید کنند. این ورودی‌ها می‌توانند شامل دستورات، سؤالات یا داده‌های زمینه‌ای باشند که به مدل کمک می‌کنند تا پاسخ‌های خود را در چارچوب معنایی و ساختاری مشخصی ارائه دهد.
می\/دانیم که مهندسی اعلان تأثیر مستقیمی بر کارایی و دقت مدل‌های زبانی بزرگ دارد از این رو با تدوین اعلان‌های دقیق و متناسب با وظیفه موردنظر، می‌توان رفتار مدل را به‌گونه‌ای هدایت کرد که خروجی‌های مرتبط‌تر و با کیفیت‌تری تولید کند. این امر به‌ویژه در شرایطی که داده‌های آموزشی محدود یا ناموجود هستند، اهمیت بیشتری پیدا می‌کند.
از مزایای مهندسی اعلان میتوان به موارد زیر اشاره کرد :
\begin{itemize}
	\item 
	با استفاده از اعلان‌های دقیق و مناسب می‌توان نتایج مدل را به سمت پاسخ‌های موردنظر هدایت کرد و کنترل بیشتری بر خروجی مدل داشت.
	\item 
	مهندسی اعلان می‌تواند به کاهش سوگیری‌های موجود در مدل‌های زبانی کمک کند و نتایج منصفانه‌تری ارائه دهد.
	\item 
	با تدوین اعلان‌های مؤثر، می‌توان زمان و منابع موردنیاز برای رسیدن به نتایج مطلوب را کاهش داد و کارایی را افزایش داد.
\end{itemize}

در نتیجه، مهندسی اعلان به‌عنوان ابزاری قدرتمند برای بهبود عملکرد مدل‌های زبانی بزرگ محسوب می‌شود و نقش کلیدی در توسعه و بهره‌برداری مؤثر از این مدل‌ها ایفا می‌کند.

\section{روش‌های دستی در مهندسی اعلان}
مهندسی اعلان دستی به فرآیند طراحی و بهینه‌سازی دستی ورودی‌ها (اعلان‌ها) برای هدایت بهتر مدل‌های زبانی بزرگ در تولید پاسخ‌های مطلوب گفته می‌شود. برخلاف روش‌های خودکار که از الگوریتم‌های یادگیری ماشین برای بهینه‌سازی اعلان‌ها استفاده می‌کنند، روش‌های دستی بر دانش زبانی، شهود انسانی و آزمایش‌های مکرر متکی هستند. این تکنیک‌ها در کاربردهای واقعی که نیاز به کنترل دقیق بر خروجی مدل دارند، مانند پاسخ‌گویی به سوالات، خلاصه‌سازی متون و انجام وظایف استدلالی، به‌کار گرفته می‌شوند.  

مهندسی اعلان دستی برای افزایش اثربخشی مدل‌های زبانی ضروری است، به‌ویژه در مواردی که تنظیم و آموزش مجدد مدل امکان‌پذیر نیست. از آنجایی که مدل‌های زبانی پاسخ‌های خود را بر اساس ورودی‌ها تولید می‌کنند، حتی تغییرات جزئی در ساختار یا نحوه بیان اعلان‌ها می‌تواند تأثیر قابل‌توجهی بر عملکرد آن‌ها داشته باشد. اعلان‌های طراحی‌شده به‌صورت بهینه می‌توانند دقت مدل را افزایش داده، توانایی استدلال آن را بهبود بخشند و میزان سوگیری در پاسخ‌ها را کاهش دهند، در نتیجه پاسخ‌های دقیق‌تر و متناسب‌تری ارائه کنند.

از ویژگی‌های کلیدی مهندسی اعلان دستی میتوان به موارد زیر اشاره کرد :
\begin{enumerate}
	\item طراحی مبتنی بر دانش انسانی 
	\begin{itemize}
		\item برخلاف روش‌های خودکار که از الگوریتم‌های بهینه‌سازی استفاده می‌کنند، مهندسی اعلان دستی بر شهود انسانی و دانش زبانی تکیه دارد.
		\item درک صحیح از زبان و زمینه موردنظر نقش مهمی در طراحی اعلان‌هایی دارد که مدل را به تولید خروجی‌های مطلوب هدایت می‌کنند. 
	\end{itemize}
	
	\item بهینه‌سازی تدریجی و تکرارشونده
	\begin{itemize}
		\item طراحی اعلان‌های مؤثر نیازمند آزمایش‌های مداوم و اصلاحات متوالی است. 
		\item تنظیمات و تغییرات مداوم در نحوه بیان اعلان به شناسایی بهترین ساختار و سبک ورودی کمک می‌کند. 
	\end{itemize}
	
	\item انعطاف‌پذیری در کاربردهای مختلف
	\begin{itemize}
		\item مهندسی اعلان دستی امکان شخصی‌سازی ورودی‌ها را برای وظایف متنوعی مانند تولید محتوا، برنامه‌نویسی و استدلال منطقی فراهم می‌کند.  
		\item استراتژی‌های خاصی را می‌توان برای هر کاربرد به‌کار گرفت، مانند ارائه دستورالعمل‌های گام‌به‌گام، اضافه کردن نشانه‌های متنی یا استفاده از نمونه‌های مشابه.
	\end{itemize}
	
	\item کنترل و تفسیرپذیری بهتر
	\begin{itemize}
		\item از آنجا که اعلان‌های دستی توسط انسان طراحی می‌شوند، کنترل بیشتری بر رفتار مدل فراهم می‌کنند.  
		\item این روش امکان درک بهتر نحوه پاسخ‌گویی مدل به ورودی‌های مختلف را فراهم کرده و به عیب‌یابی و بهبود عملکرد کمک می‌کند.
	\end{itemize}
\end{enumerate}

با وجود مزایای فراوان، این روش با چالش‌هایی همراه است:  
\begin{itemize}
	\item فرآیند زمان‌بر: طراحی و بهینه‌سازی دستی اعلان‌ها نیاز به صرف زمان زیادی دارد.  
	\item مقیاس‌پذیری پایین: برخلاف روش‌های خودکار، اعلان‌های دستی به‌راحتی برای مدل‌ها یا وظایف دیگر تعمیم نمی‌یابند.  
	\item ماهیت مبتنی بر آزمون و خطا: یافتن اعلان بهینه اغلب نیازمند آزمایش‌های متعدد است که همیشه نتیجه‌ای ثابت و پایدار را تضمین نمی‌کند.  
\end{itemize}

در نهایت مهندسی اعلان دستی به‌عنوان رویکردی بنیادین برای بهینه‌سازی تعاملات با مدل‌های زبانی شناخته می‌شود. در بخش‌های بعدی، روش‌های مختلف مهندسی اعلان دستی را بررسی خواهیم کرد و تأثیر هر یک را بر توانایی‌های استدلالی و کیفیت پاسخ‌دهی مدل می‌سنجیم.
%بررسی کاربرد و مزایا و معایب این روش‌ها
\subsection{یادگیری درون متنی}

یادگیری درون‌متنی
\LTRfootnote{In-Context Learning}
 از قابلیت‌های برجسته مدل‌های زبانی بزرگ است که به آن‌ها اجازه می‌دهد بدون نیاز به بازآموزی
 \LTRfootnote{Training}
  یا تنظیم مجدد وزن‌ها، تنها با دریافت چند نمونه در ورودی، وظایف جدید را تشخیص دهند و به درستی انجام دهند. این روش به مدل کمک می‌کند با تحلیل مثال‌های ارائه‌شده در ورودی، پاسخ‌هایی متناسب با همان زمینه تولید کند.
از مزایای مهم این روش، افزایش انعطاف‌پذیری در مواجهه با وظایف و موضوعات جدید است که باعث می\/شود مدل‌های زبانی به سرعت خود را با زمینه‌های مختلف تطبیق دهند. با حذف نیاز به بازآموزی برای هر وظیفه جدید، این روش می‌تواند بهینه‌سازی قابل توجهی در مصرف منابع و زمان ایجاد کند.
از سوی دیگر، این قابلیت نقش مهمی در بهبود کیفیت تعاملات میان انسان و ماشین ایفا می‌کند. مدل‌ها با درک بهتر زمینه و مثال‌های ارائه‌شده، پاسخ‌هایی طبیعی‌تر و دقیق‌تر تولید می‌کنند که باعث افزایش رضایت کاربران می‌شود.

با وجود مزایای فوق، یادگیری درون‌متنی با چالش‌هایی نیز همراه است. عملکرد مدل‌ها به شدت به کیفیت و تنوع داده‌ها و مثال‌های ورودی وابسته است. مثال‌های ناقص یا ناسازگار می‌توانند منجر به تولید پاسخ‌های نادرست شوند. همچنین، در برخی موقعیت‌های پیچیده یا مبهم، مدل ممکن است در درک دقیق زمینه دچار مشکل شود. 


از روش های یادگیری درون متنی میتوان به یادگیری بدون نمونه و یادگیری با نمونه های کم اشاره کرد. در یادگیری بدون نمونه 
\cite{ZSL}
\LTRfootnote{Zero-Shot Learning (ZSL)}
  مدل‌ها وظیفه دارند اشیاء یا مفاهیمی را که در طول آموزش با آن‌ها روبرو نشده‌اند شناسایی و دسته‌بندی کنند. برخلاف یادگیری نظارت‌شده سنتی که نیاز به مثال‌های برچسب‌خورده برای هر کلاس دارد، یادگیری بدون نمونه به مدل‌ها این امکان را می‌دهد که پیش‌بینی‌هایی در مورد دسته‌های دیده نشده در زمان آموزش با استفاده از اطلاعات کمکی انجام دهند.  
برای شناسایی کلاس‌های دیده نشده در زمان آموزش، مدل‌های یادگیری بدون نمونه روابطی بین کلاس‌های مشاهده‌شده (دیده‌شده) و کلاس‌های مشاهده‌نشده (دیده‌ نشده) برقرار می‌کنند. این ارتباط معمولاً از طریق ویژگی‌های مشترک یا اطلاعات معنایی که این دو را به هم پیوند می‌دهند، تسهیل می‌شود.  
از طرفی وجود اطلاعات کمکی برای یادگیری بدون نمونه حیاتی است و  داده‌های توصیفی در مورد کلاس‌های دیده‌شده و دیده نشده ارائه می‌دهد. این اطلاعات می‌تواند شامل ویژگی‌هایی مانند توضیحات متنی، جاسازی‌های معنایی یا سایر داده‌های مرتبط باشد که به مدل کمک می‌کند تا کلاس‌ها را درک کرده و از هم تمایز دهد. 

چگونگی عملکرد یادگیری بدون نمونه بدین صورت است که در غیاب مثال‌های برچسب‌خورده برای کلاس‌های دیده نشده، برای پر کردن شکاف به اطلاعات کمکی اتکا می\/کنند. به عنوان مثال، در طبقه‌بندی تصاویر، مدلی که بر اساس گونه‌های خاص حیوانات آموزش دیده باشد، می‌تواند از توضیحات متنی گونه‌های دیده نشده برای شناسایی صحیح آن‌ها استفاده کند. با درک اینکه یک گورخر مشابه یک اسب است اما با خطوط راه‌راه، مدل می‌تواند گورخرها را بدون دیدن آن‌ها در طول آموزش شناسایی کند (تصویر \ref{fig_zil}).

\begin{figure}[!t]
	\centering
	\includegraphics[width=100mm]{images/ZeroShot}
	\caption{مثالی از یادگیری بدون نمونه در یادگیری درون متنی}
	\label{fig_zil}
\end{figure}

یادگیری بدون نمونه در حوزه‌های مختلفی کاربرد داشته است، از جمله:
\begin{itemize}
	\item 
	طبقه‌بندی تصویر: شناسایی اشیاء یا گونه‌هایی که در داده‌های آموزشی وجود ندارند.
	\item 
	بخش‌بندی معنایی: بخش‌بندی اشیاء دیده نشده در تصاویر بر اساس ویژگی‌های یادگرفته‌شده.
	\item 
	پردازش زبان طبیعی: انجام کارهایی مانند طبقه‌بندی متن یا شناسایی موجودیت‌ها بدون مثال‌های صریح.
	\item 
	زیست‌شناسی محاسباتی: پیش‌بینی عملکرد ژن‌ها یا پروتئین‌هایی که فاقد نشانه‌های آزمایشی هستند.  
\end{itemize}

از چالش‌های مهم یادگیری بدون نمونه می\/توان به احتمال اشتباه کردن مدل در دسته‌بندی کلاس‌های دیده نشده اشاره کرد، به ویژه زمانی که این کلاس‌ها ویژگی‌هایی مشابه با کلاس‌های دیده‌شده دارند. تحقیقات جاری به دنبال افزایش کارایی و دقت مدل‌های یادگیری بدون نمونه است و روش‌هایی مانند تولید داده‌های مصنوعی برای کلاس‌های دیده نشده و بهبود کیفیت اطلاعات کمکی را بررسی می‌کنند.

در طرف دیگر ماجرا، برای یادگیری با نمونه های کم \cite{FSL} مدل‌ها طوری طراحی می‌شوند که بتوانند بر اساس تعداد بسیار محدودی از نمونه‌های آموزشی، یادگیری کرده و پیش‌بینی‌های دقیقی انجام دهند. این رویکرد به‌ویژه در موقعیت‌هایی مفید است که جمع‌آوری مجموعه‌داده‌های بزرگ عملی یا مقرون‌به‌صرفه نیست.

در یادگیری با نمونه های کم، مدل‌ها در مرحله استنتاج تنها تعداد کمی نمونه برچسب‌خورده (که اغلب به آن مجموعه پشتیبان
\LTRfootnote{support set}
 گفته می‌شود) برای هر کلاس جدید دریافت می‌کنند. این تعداد محدود از نمونه‌ها به مدل امکان می‌دهد تا به سرعت خود را تطبیق داده و برای کلاس‌های نادیده‌شده پیش‌بینی انجام دهد.

یک تنظیم رایج در یادگیری با نمونه های کم شامل الگوی N-way K-shot است که در آن 'N' نشان‌دهنده تعداد کلاس‌های جدید و 'K' نشان‌دهنده تعداد نمونه‌های برچسب‌خورده موجود برای هر کلاس است. برای مثال، در یک سناریوی 5-way 1-shot، پنج کلاس جدید وجود دارد که هر کدام تنها یک نمونه برچسب‌خورده دارند.

مدل‌های یادگیری با نمونه های کم معمولاً از تکنیک‌های فرا-یادگیری 
\LTRfootnote{meta Learning}
 استفاده می‌کنند که به آن "یادگیری برای یادگیری
 \LTRfootnote{Learning to Learn}
 " نیز گفته می‌شود. در این چارچوب، مدل در طیف وسیعی از وظایف آموزش می‌بیند تا یک استراتژی برای تطبیق سریع با وظایف جدید با حداقل داده‌ها را یاد بگیرد. در زمان استنتاج، مدل با استفاده از همان تعداد اندک نمونه‌ها، پارامترهای خود را به‌طور مؤثری تنظیم می‌کند و می‌تواند دسته‌بندی‌های جدید را شناسایی و طبقه‌بندی کند.

از کاربردهای یادگیری با نمونه‌های کم می\/توان به موارد زیر اشاره کرد:
\begin{itemize}
	\item 
	یادگیری با نمونه‌های کم به مدل‌ها امکان می‌دهد تصاویر را با استفاده از تنها چند نمونه برچسب‌خورده در دسته‌های جدید طبقه‌بندی کنند که این موضوع به‌ویژه در حوزه‌هایی مانند تصویربرداری پزشکی که برچسب‌گذاری داده‌ها بسیار زمان‌بر است، اهمیت دارد.
	\item 
	در حوزه پردازش زبان طبیعی، یادگیری با نمونه‌های کم می‌تواند در وظایفی مانند طبقه‌بندی متن و تحلیل احساسات به کار رود و به مدل‌ها کمک کند تا با حداقل داده متنی، موضوعات جدید را پردازش و درک کنند.
	\item 
	ربات‌ها می‌توانند با استفاده از یادگیری با نمونه‌های کم وظایف جدید در زمینه دست‌کاری اشیاء یا تطبیق با محیط‌های تازه را یاد بگیرند، که این موضوع نیاز به آموزش مجدد گسترده را کاهش داده و امکان استقرار سریع در محیط‌های پویا را فراهم می‌آورد.
\end{itemize}

یکی از چالش‌های اصلی در یادگیری با نمونه‌های کم این است که اطمینان حاصل شود مدل‌ها خود را با داده‌های محدودبه‌خوبی تعمیم می‌دهند و دچار بیش‌برازش
\LTRfootnote{overfitting}
 نمی‌شوند. پژوهشگران در حال بررسی روش‌های مختلفی از جمله افزایش داده 
 \LTRfootnote{data augmentation}
 ، یادگیری انتقالی
 \LTRfootnote{transfer learning}
  و الگوریتم‌های پیشرفته فرا-یادگیری
  \LTRfootnote{meta Learning}
   برای بهبود عملکرد مدل‌های یادگیری با نمونه‌های کم هستند.

برای فهم شفاف تر روش یادگیری درون متنی میتوان از یک چهارچوب ریاضیاتی
\cite{beysian}
 کمک گرفت. در این چارچوب ریاضیاتی می\/توان یادگیری درون‌متنی را به‌عنوان یک استنتاج بیزی ضمنی
\LTRfootnote{implicit Bayesian inference}
 تفسیر کرد. در چارچوب آن‌ها، در طی پیش‌آموزش
 \LTRfootnote{pre-training}
 ، مدل‌های زبانی بزرگ با استنتاج مفاهیم پنهان در اعلان، که روابط معنایی و نحوی مختلفی را در متن در بر می‌گیرد، یاد می‌گیرند که توکن‌های بعدی را پیش‌بینی کنند. در زمان استنتاج
 \LTRfootnote{inference time}
 ، زمانی که مدلی با یک اعلان شامل مثال‌های ورودی-خروجی مواجه می‌شود، یک مفهوم پنهان مشترک بین این مثال‌ها را شناسایی می‌کند. این فرآیند شناسایی با استنتاج بیزی هم‌راستا است، جایی که مدل بر اساس داده‌های مشاهده‌شده، باورهای خود را به‌روزرسانی می‌کند تا پیش‌بینی انجام دهد.

\begin{equation}\label{eq_icl}
	P(\text{خروجی} \mid \text{اعلان}) = \int P(\text{خروجی} \mid \text{مفهوم}) \cdot P(\text{اعلان} \mid \text{مفهوم}) \cdot P( \text{مفهوم}) \, d \text{مفهوم}
\end{equation}



همانطور که در معادله \ref{eq_icl} مشاهده می\/شود، از منظر بیزی، یادگیری درون متنی با پیدا کردن احتمال خروجی به شرط اعلان برابر است که انجام استنتاج توسط مدل برای یافتن یک مفهوم پنهان را در بر می\/گیرد که این مفهوم با وظیفه مورد نظر همخوانی دارد. با دریافت یک اعلان، مدل توزیع پسین
\LTRfootnote{posterior}
 را بر روی مفاهیم پنهان ممکن استنتاج می‌کند و مفهومی را انتخاب می‌کند که به بهترین نحو مثال‌های ارائه‌شده را توضیح می‌دهد. این فرآیند مشابه به‌روزرسانی بیزی است که در آن باورهای قبلی در پرتو شواهد جدید تنظیم می‌شوند تا پیش‌بینی‌های آگاهانه‌تری انجام گیرد.

\subsection{روش زنجیره تفکر}
روش زنجیره تفکر
\LTRfootnote{Chain-of-Thought} \cite{CoT}
 تکنیکی است که با هدف تقویت توانایی استدلال مدل‌های زبانی بزرگ طراحی شده و آن‌ها را راهنمایی می‌کند تا هنگام حل مسائل پیچیده، گام‌های میانی استدلالی تولید کنند. این رویکرد مدل‌ها را تشویق می‌کند تا وظایف را به بخش‌های متوالی و مرحله‌به‌مرحله تقسیم کرده و در نتیجه به خروجی‌هایی دقیق‌تر و قابل تفسیرتر برسند.

این روش که توسط پژوهشگران گوگل معرفی شد، شامل ارائه نمونه‌هایی به مدل‌ها است که هم شامل مسئله و هم شامل راه‌حل گام‌به‌گام و دقیق هستند. به عنوان مثال، وقتی یک مسئله کلامی ریاضی به مدل داده می‌شود، مدل با استفاده از زنجیره تفکر تشویق می‌شود که محاسبات و مراحل منطقی منتهی به پاسخ نهایی را به صورت شفاف بیان کند. این روش نشان داده که عملکرد مدل‌ها را در وظایف نیازمند به محاسبات ریاضی، استدلال مبتنی بر عقل سلیم و استدلال نمادین به شکل چشمگیری بهبود می‌دهد. به طور خاص، یک مدل با ۵۴۰ میلیارد پارامتر با استفاده از CoT موفق شد به دقتی فراتر از حد استاندارد در دیتاست GSM8K برای مسائل ریاضی دست پیدا کند و حتی بهتر از نسخه‌های تنظیم شده GPT-3 عمل کند.

اثربخشی روش زنجیره تفکر به‌ویژه در مدل‌های بزرگ‌تر بیشتر مشهود است. مدل‌هایی با بیش از ۱۰۰ میلیارد پارامتر در مواجهه با زنجیره تفکر توانایی‌های نوظهوری در زمینه استدلال چند مرحله‌ای از خود نشان می‌دهند و می‌توانند مسائل چند گامی را به شکل موثرتری حل کنند. این تکنیک نه تنها دقت را افزایش می‌دهد، بلکه شفافیت فرآیند استدلال مدل را نیز بهبود می‌بخشد؛ چرا که هر گام به صورت صریح ارائه می‌شود.

در شکل \ref{fig_cot} یک نمونه از تولید جواب با استفاده از روش زنجیره تفکر آورده شده است. در این مثال به عنوان ورودی یک سوال و جواب نمونه به همراه راه حل قدم به قدم مسئله نیز آورده شده است تا مدل به سمت تولید راه حل سوق داده شود.
\begin{figure}[!t]
	\centering
	\includegraphics[width=140mm]{images/Cot}
	\caption{مثالی از روش زنجیره تفکر برای حل یک سوال از دیتاست \lr{GSM8K}}
	\label{fig_cot}
\end{figure}

\subsection{روش استدلال بدون دیدن نمونه آموزشی} 
در مقاله‌ی استدلال بدون دیدن نمونه آموزشی
\LTRfootnote{Large Language Models are Zero-Shot Reasoners} \cite{LLMzeroshot}
، پتانسیل مدل های بزرگ زبانی را برای انجام استدلال  از طریق تغییرات ساده در اعلان‌ها و بدون افزودن نمونه ای از مثال های حل شده، بررسی شده است. با اضافه کردن عبارت «بیایید مرحله به مرحله فکر کنیم» به انتهای یک اعلان، مدل‌هایی مانند GPT-3 و PaLM
\cite{palm2}
به‌طور قابل توجهی عملکرد بهتری در دیتاست‌های مختلف استدلالی ارائه می‌دهند. 
این روش بر توانایی ذاتی مدل‌ها در پردازش و تولید متن شبیه به انسان تکیه دارد و آن‌ها را تشویق می‌کند تا دنباله‌ای منطقی از تفکرات را که منجر به پاسخ نهایی می‌شود، تولید کنند. سادگی و کارآمدی این تکنیک، توانایی‌های استدلالی نهفته در مدل های بزرگ زبانی را آشکار می‌کند که می‌توان آن‌ها را بدون نیاز به آموزش وسیع و خاص برای هر وظیفه فعال کرد.
این یافته نشان می‌دهد که با مهندسی اعلان مناسب، این مدل‌ها می‌توانند طیف وسیعی از وظایف را به‌طور مؤثرتری انجام دهند و نیاز به دیتاست‌های بزرگ و برچسب‌گذاری شده و همچنین فرایندهای زمان‌بر تنظیم دقیق مدل‌ها را کاهش دهند. در شکل \ref{fig_zerocot} نمونه از حل مسئله دیتاست \lr{GSM8K} آورده شده است و جواب این روش با روش زنجیره تفکر مقایسه شده است.

\begin{figure}[!t]
	\centering
	\includegraphics[width=140mm]{images/zerocot}
	\caption{مقایسه روش استدلال بدون دیدن نمونه آموزشی و روش زنجیره تفکر}
	\label{fig_zerocot}
\end{figure}


\subsection{روش برنامه تفکر}
روش برنامه تفکر
\LTRfootnote{Program-of-Thought} \cite{PoT}
یک روش پیشرفته در مهندسی اعلان است که برای بهبود توانایی‌های استدلال عددی در مدل‌های زبانی بزرگ طراحی شده است. برخلاف تکنیک زنجیره تفکر که در آن مدل هم استدلال و هم محاسبات را در قالب متن تولید می‌کرد، در برنامه تفکر این دو فرآیند از یکدیگر جدا می‌شوند. در این روش، مدل استدلال خود را به صورت کد قابل اجرایی (معمولاً با زبان‌هایی مانند پایتون) بیان می‌کند و این کد توسط یک مفسر خارجی اجرا می‌شود تا پاسخ نهایی به دست آید. این جداسازی باعث می‌شود محاسبه و استدلال از یکدیگر تفکیک شوند.

از مزایای روش برنامه تفکر می\/توان به موارد زیر اشاره کرد:
\begin{itemize}
	\item 
	با واگذاری محاسبات به یک مفسر خارجی، احتمال بروز خطاهای عددی که ممکن است در صورت انجام محاسبه و استدلال توسط خود مدل رخ دهد، کاهش می‌یابد و دقت بالاتر می\/رود.
	\item 
	مطالعات تجربی نشان داده‌اند که برنامه تفکر می‌تواند عملکرد مدل را در وظایف عددی پیچیده به طور قابل توجهی افزایش دهد. برای مثال، آزمایش‌ها نشان داده‌اند که PoT نتایجی هم‌سطح با بهترین روش‌های موجود در دیتاست‌های مسائل ریاضی و نزدیک به بهترین عملکردها در دیتاست‌های مالی داشته است.
	\item 
	نمایش مراحل استدلال به صورت کد، شفافیت فرآیند حل مسئله را افزایش می‌دهد و بررسی و درک هر مرحله را ساده‌تر می‌کند.
\end{itemize}

در شکل \ref{fig_pot} حل مسئله دنباله فیبونانچی و پیداکردن 50\/امین عضو این دنباله، با استفاده از روش زنجیره تفکر و روش برنامه تفکر بررسی شده است. به دلیل طولانی بودن مراحل حل مسئله، روش زنجیره تفکر نتوانسته جواب درست را پیدا کند ولی روش برنامه تفکر ابتدا یک کد برای حل این مسئله ارائه داده است و پس از اجرای این کد با مفسر، به جواب صحیح رسیده است.

\begin{figure}[!t]
	\centering
	\includegraphics[width=140mm]{images/pot}
	\caption{مقایسه روش برنامه تفکر و روش زنجیره تفکر}
	\label{fig_pot}
\end{figure}

\subsection{روش بهینه سازی با اعلان}
همانطور که گفته شد مدل‌های زبانی بزرگ تا به اینجا برای وظایفی مانند تولید متن، ترجمه و خلاصه‌سازی به کار گرفته می‌شدند. با این حال، تحقیقات اخیر ظرفیت این مدل‌ها را به‌عنوان بهینه‌سازها مورد بررسی قرار داده‌اند و از قابلیت‌های استدلالی آن‌ها برای حل مسائل پیچیده بهینه‌سازی بهره برده‌اند.

یکی از رویکردهای برجسته در این زمینه، روش بهینه سازی با اعلان 
\LTRfootnote{Optimization by PROmpting} \cite{opro}
 است. این روش از مدل‌های زبانی بزرگ برای تولید راه‌حل‌های مسائل بهینه‌سازی بر اساس اعلان‌های زبان طبیعی استفاده می‌کند. این فرآیند شامل تکرارهای پی‌در‌پی است که در آن، اعلان جدید با راه‌حل‌ها و ارزیابی‌های قبلی طراحی می‌شود و در تکرارهای بعدی، راه‌حل‌های بهتری پیشنهاد می‌دهد. این روش در کاربردهای مختلفی همچون رگرسیون خطی
 \LTRfootnote{Linear Regression}
 ، مسئله فروشنده دوره‌گرد
 \LTRfootnote{Traveling Salesman Problem}
  و حتی بهینه‌سازی اعلان برای خود مدل‌های زبانی بزرگ مؤثر واقع شده است. جالب توجه اینکه این روش توانسته تا ۸٪ بهبود عملکرد در دیتاست \lr{GSM8K} و تا ۵۰٪ بهبود در وظایف دشوار
  \LTRfootnote{Big-Bench Hard}
   داشته باشد و حتی از اعلان‌های طراحی‌شده توسط انسان نیز پیشی بگیرد.

در روش بهینه سازی با اعلان، مسئله بهینه‌سازی با زبان طبیعی توصیف می‌شود تا مدل بتواند بافت و هدف مسئله را درک کند. همانطور که در شکل \ref{fig_opro} در هر مرحله از بهینه‌سازی، مدل با توجه به اعلان شامل راه‌حل‌ها و ارزیابی‌های قبلی، راه‌حل‌های جدیدی تولید می‌کند. این راه‌حل‌ها ارزیابی شده و سپس برای تکرار بعدی در اعلان گنجانده می‌شوند و این چرخه بهبود مستمر ادامه پیدا می‌کند.

\begin{figure}[!t]
	\centering
	\includegraphics[width=140mm]{images/opro}
	\caption{دیاگرام روش بهینه سازی با اعلان}
	\label{fig_opro}
\end{figure}


از مزایای استفاده از مدل‌های زبانی بزرگ به‌عنوان بهینه‌ساز می\/توان به موارد زیر اشاره کرد:

\begin{itemize}
	\item 
	مدل‌های زبانی بزرگ به دلیل توانایی درک و تولید متنی مشابه انسان، می‌توانند به انواع مختلف مسائل بهینه‌سازی که با زبان طبیعی توصیف می‌شوند، پاسخ دهند.
	\item 
	همچنین این مدل ها قادر به انجام بهینه‌سازی در مسائلی هستند که اطلاعات گرادیان در دسترس نیست یا به‌دست‌آوردن آن دشوار است و به همین دلیل رویکردی بدون نیاز به مشتق ارائه می‌دهند.
	\item 
	از طرفی این مدل ها می‌توانند اعلان‌های خود را بهینه کنند و بدون نیاز به آموزش مجدد، عملکرد خود را در وظایف خاص بهبود دهند.
\end{itemize}




\subsection{روش برنامه\/ریزی و حل}
روش برنامه\/ریزی و حل
\LTRfootnote{Plan-and-Solve (PS)}
یک تکنیک پیشرفته در مهندسی اعلان است که بهبود عملکرد مدل‌های زبانی بزرگ را در حل مسائل پیچیده مورد هدف قرار می‌دهد. در این روش، مدل ابتدا یک برنامه‌ریزی
\LTRfootnote{Planing}
 انجام داده و سپس بر اساس آن، به حل مسئله
\LTRfootnote{Solve}
  می‌پردازد.  

برخلاف روش‌های سنتی که مدل را مستقیماً درگیر حل مسئله می‌کنند، این روش باعث کاهش نرخ خطا شده و دقت پاسخ‌های مدل را افزایش می‌دهد. روش برنامه ریزی و حل به‌ویژه در مسائلی که نیاز به چندین مرحله استدلالی دارند، مانند حل مسائل ریاضی، تحلیل منطقی و برنامه‌ریزی وظایف، بسیار کارآمد است.  


این روش شامل دو مرحله اصلی است:  

\begin{enumerate}
	\item برنامه‌ریزی 
	\LTRfootnote{Plan Phase}:
	مدل یک برنامه کلی برای حل مسئله ارائه می‌دهد، شامل مراحل موردنیاز برای رسیدن به پاسخ.  
	\item حل مسئله
	\LTRfootnote{Solve Phase}:
	مدل بر اساس برنامه تولیدشده، گام‌به‌گام راه‌حل را پیاده‌سازی کرده و پاسخ نهایی را استخراج می‌کند.  
\end{enumerate}  

این تفکیک دو مرحله‌ای، عملکرد مدل را بهبود می‌بخشد زیرا ابتدا ساختار حل مسئله مشخص شده و سپس محاسبات انجام می‌شود.  
در ادامه یک مسئله و راه حل این روش برای آن مسئله را بررسی می\/کنیم.

مسئله:	سن علی ۳ برابر سن برادرش است. ۴ سال پیش، مجموع سن آن‌ها ۲۰ سال بوده است. سن هر یک را مشخص کنید. 

\begin{enumerate}
	\item اجرای مرحله برنامه‌ریزی:
	\begin{itemize}
		\item تعریف متغیرها: فرض کنیم سن برادر علی را $X$ در نظر بگیریم.  
		\item رابطه کنونی: سن علی برابر با $3X$ است.  
		\item رابطه در گذشته: ۴ سال پیش، سن برادر علی برابر $X-4$ و سن علی برابر $3X-4$ بوده است.  
		\item معادله کلی:  
		\[
		(X - 4) + (3X - 4) = 20
		\]
		\item حل معادله و یافتن مقدار $X$
	\end{itemize} 
	
	\item اجرای مرحله حل:
	\[
	X - 4 + 3X - 4 = 20
	\]  
	\[
	4X - 8 = 20
	\]  
	\[
	4X = 28
	\]  
	\[
	X = 7
	\]  
	در نتیجه :  
	\begin{itemize}
		\item سن برادر علی : $7$ سال  
		\item سن علی : $3 \times 7 = 21$ سال  
	\end{itemize}  
\end{enumerate}

با بررسی این مثال می\/توان متوجه افزایش دقت حل مسائل چند مرحله ای با استفاده از روش برنامه ریزی و حل شد و همچنین عملکرد برنامه ریزی و سپس عمل را مشاهده کرد که مشابه سیستم تفکر انسانی است.
در شکل \ref{fig_ps} یک مثال دیگر از این روش آورده شده است.

\begin{figure}[!t]
	\centering
	\includegraphics[width=160mm]{images/ps}
	\caption{یک مثال از روش برنامه\/ریزی و حل}
	\label{fig_ps}
\end{figure}


\section{بهینه‌سازی خودکار اعلان‌ها}
بهینه‌سازی خودکار اعلان‌ها یک رویکرد نوین در مهندسی اعلان است که با بهره‌گیری از الگوریتم‌های جستجو و بهینه‌سازی، به صورت سیستماتیک بهترین ورودی‌های متنی را برای مدل‌های زبانی شناسایی و تولید می‌کند. در این روش به جای تکیه بر شهود انسانی و روش‌های آزمون و خطا، از تکنیک‌هایی مانند الگوریتم‌های تکاملی، جستجوی تصادفی و یادگیری تقویتی استفاده می‌شود. این الگوریتم‌ها فضای اعلان‌ها را مورد بررسی قرار داده و بر اساس معیارهای مشخصی مانند دقت، انسجام و ارتباط معنایی، به انتخاب بهینه‌ترین اعلان‌ها می‌پردازند.

یکی از مزایای مهم بهینه‌سازی خودکار اعلان‌ها این است که در مقیاس‌های بزرگ و در زمان‌های کوتاه، قادر به دستیابی به نتایج بهینه می‌باشد. بهینه‌سازی خودکار پرامت ها با ارائه چارچوبی سیستماتیک، امکان ارزیابی سریع و انتخاب اعلان‌های بهینه را فراهم می‌آورد. این دو رویکرد می‌توانند مکمل یکدیگر عمل کرده و در کاربردهایی که نیاز به سرعت و دقت بالا دارند، مانند پردازش دسته‌جمعی داده‌های متنی یا تنظیم خودکار اعلان‌ها برای وظایف متنوع، عملکرد بهتری ارائه دهند.

\subsection{روش زنجیره تفکر خودکار}
از اولین جرقه های خودکارسازی مهندسی اعلان، میتوان به خودکارسازی تولید زنجیره‌های استدلالی در روش زنجیره تفکر اشاره کرد. روش زنجیره تفکر خودکار 
\LTRfootnote{AUTOMATIC CHAIN-OF-THOUGHT PROMPTING (Auto-CoT)} \cite{auto_cot}
 به‌طور خودکار زنجیره‌هایی از استدلال‌ها و سوالات را برای ساخت نسخهها ایجاد می‌کند. این روش شامل دو مرحله اصلی است. مرحله اول خوشه‌بندی سوالات و تقسیم سوالات یک مجموعه‌داده به چندین خوشه و مرحله دوم نمونه‌گیری از نسخهها و انتخاب یک سوال نماینده از هر خوشه و تولید زنجیره استدلال برای آن است. روند کلی این روش در شکل \ref{fig_autocot} نشان داده شده است.

در مرحله اول، از آنجایی که خوشه‌بندی مبتنی بر تنوع می‌تواند از گمراهی ناشی از شباهت جلوگیری کند، این روش تحلیل خوشه‌ای را برای مجموعه‌ای از سوالات $Q$ انجام می‌دهد. بدین صورت که ابتدا برای هر سوال در $Q$، یک بردار نمایشی با استفاده از \lr{Sentence-BERT} \cite{sentenceBert} محاسبه می\/شود و سپس این بردارهای متنی میانگین‌گیری شده و یک بردار با اندازه ثابت برای هر سوال ایجاد می‌شود. پس از آن، با استفاده از الگوریتم خوشه‌بندی k-means ، سوالات به $k$ خوشه تقسیم می‌شوند. در نهایت برای هر خوشه $i$، سوالات آن خوشه را به‌صورت یک لیست مرتب‌شده \lr{$q^{(i)} = [q^{(i)}_1, q^{(i)}_2, \ldots]$} بر اساس فاصله از مرکز خوشه به ترتیب صعودی مرتب می‌شوند.

در مرحله دوم، برای سوالات نمونه‌گیری‌شده، زنجیره‌های استدلال تولید می\/شوند و نسخههایی که با معیارهای انتخاب مطابقت دارند استخراج می\/شوند. به‌طور دقیق‌تر، برای هر خوشه $i$، یک نسخه به صورت \lr{$d^{(i)}$} (ترکیبی از سوال، استدلال و پاسخ) ساخته می\/شود. برای خوشه $i$، سوالات مرتب‌شده در لیست \lr{$q^{(i)} = [q^{(i)}_1, q^{(i)}_2, \ldots]$} تا زمانی که معیارهای انتخاب برآورده شوند، بررسی می‌شوند.

به عبارت دیگر، سوالی که به مرکز خوشه نزدیک‌تر است، زودتر بررسی می‌شود. فرض کنید سوال \lr{$q^{(i)}_j$} که $j$-اُمین سوال نزدیک به مرکز خوشه $i$ است، در حال بررسی باشد. یک ورودی به شکل زیر ساخته می‌شود:

\begin{center}
	\lr{[Q: $q^{(i)}_j$ A: [P]]}
\end{center}

سپس یک نسخه کاندید برای خوشه $i$ به صورت زیر ساخته می‌شود:

\begin{center}
	\lr{[Q: $q^{(i)}_j$ A: $r^{(i)}_j$ $a^{(i)}_j$]}
\end{center}

مشابه معیارهای مورد استفاده در نسخه\/های دستی روش زنجیره تفکر \cite{CoT}, معیار انتخاب این الگوریتم نیز بر قوانین ساده‌ای تکیه دارد تا سوالات و استدلال‌های ساده‌تر را انتخاب کند: نسخه \lr{$d^{(i)}$} به‌عنوان \lr{$d^{(i)}_j$} انتخاب می‌شود اگر سوال \lr{$q^{(i)}_j$} دارای کمتر از ۶۰ توکن و استدلال \lr{$r^{(i)}_j$} دارای کمتر از ۵ گام استدلالی باشد.

\begin{figure}[!t]
	\centering
	\includegraphics[width=140mm]{images/autocot}
	\caption{دیاگرام روش auto-CoT}
	\label{fig_autocot}
\end{figure}

\subsection{روش مهندس اعلان خودکار}
روش  مهندس اعلان خودکار
\LTRfootnote{Auto Prompt Engineer} \cite{APE}
 از خود مدل‌های زبانی بزرگ برای خودکارسازی فرآیند ایجاد و بهینه‌سازی اعلان‌ها استفاده می‌کند. در این روش، یک مدل زبانی، مجموعه‌ای از اعلان‌های مختلف تولید می‌کند و سپس این اعلان‌ها بر اساس کیفیت پاسخ‌هایی که از یک مدل دیگر دریافت می‌شود، ارزیابی می‌شوند. این فرآیند به‌صورت تکراری ادامه می‌یابد تا مؤثرترین اعلان‌ها شناسایی شوند. نتایج در ۲۴ وظیفه مختلف در حوزه پردازش زبان طبیعی نشان داده که اعلان‌های تولید شده توسط این روش در اغلب موارد عملکرد بهتری نسبت به روش‌های قبلی داشته‌اند و در ۱۹ مورد از ۲۴ وظیفه، کیفیتی معادل یا نزدیک به اعلان‌های طراحی شده توسط انسان ارائه کرده‌اند.

تأثیرات این روش قابل توجه است؛ این روش نیاز به مهندسان اعلان انسانی را کاهش می‌دهد، فرآیند توسعه را سریع‌تر می‌کند و سازگاری مدل‌های زبانی بزرگ با وظایف مختلف را افزایش می‌دهد. این پیشرفت نشان‌دهنده ظرفیت مدل‌های بزرگ زبانی برای نه‌تنها انجام وظایف، بلکه بهینه‌سازی دستورالعمل‌های خودشان نیز هست و گامی مهم به سوی سامانه‌های هوشمند خودمختارتر به‌شمار می‌رود.

\subsection{روش مولد اعلان}
روش مولد اعلان
\LTRfootnote{Promptbreeder} \cite{PromptBreeder}
 یک سیستم خودارجاعی
 \LTRfootnote{self-referential}
  و تکاملی برای بهبود خودکار اعلان‌های مورد استفاده در مدل‌های زبانی بزرگ است. این روش با الهام از الگوریتم‌های تکاملی و فرآیندهای خودبهبوددهی
  \LTRfootnote{self-improvment}
  ، به جای اتکا بر اعلان‌های دستی و مهندسی‌شده، به طور خودکار اعلان‌هایی را تولید و اصلاح می‌کند که می‌توانند عملکرد مدل را در حل مسائل مختلف بهبود بخشند. ویژگی کلیدی این روش این است که نه‌تنها اعلان‌های وظیفه
  \LTRfootnote{Task Prompts}
   را تکامل می‌دهد، بلکه اعلان‌های تغییر
   \LTRfootnote{Mutation Prompts}
    را نیز که برای تغییر اعلان‌های مورد جستجو استفاده می‌شوند، بهبود می‌بخشد.  

این الگوریتم از یک فرآیند تکاملی مبتنی بر جمعیت
\LTRfootnote{population base}
 استفاده می‌کند. همانطور که در شکل \ref{fig_promptbreeder} نشان داده شده است، در ابتدا یک مجموعه از اعلان‌های اولیه به همراه دستورات تغییر تولید می‌شود. سپس، در هر نسل از فرآیند تکامل: 
 \begin{enumerate}
 	\item ارزیابی سازگاری
 	\LTRfootnote{Fitness Evaluation}: اعلان‌ها بر اساس عملکردشان در پاسخ‌دهی به مجموعه‌ای از سؤالات آموزشی ارزیابی می‌شوند.
 	
 	\item انتخاب
 	\LTRfootnote{Selection}: دو اعلان به‌صورت تصادفی انتخاب شده و مقایسه می‌شوند؛ اعلانی که عملکرد بهتری داشته باشد، انتخاب می‌شود.
 	
 	\item اعمال تغییرات
 	\LTRfootnote{Mutation}: اعلان انتخاب‌شده با استفاده از عملگرهای تکاملی تغییر داده می‌شود. این تغییرات شامل تولید نسخه‌های جدید از اعلان، اصلاح بر اساس الگوهای موفق، و یا حتی بهبود خود دستور تغییر است.  
 	
 	\item جایگزینی
 	\LTRfootnote{Replacement}: نسخه‌ی بهبودیافته جایگزین اعلان با عملکرد ضعیف‌تر شده و این فرآیند در نسل‌های بعدی تکرار می‌شود.
 \end{enumerate} 

عملگرهای تکاملی شامل جهش مستقیم
\LTRfootnote{Direct Mutation}
، جهش مبتنی بر توزیع
\LTRfootnote{Estimation of Distribution (EDA) Mutation}
، ابرجهش
\LTRfootnote{Hypermutation}
، جهش لامارکین
\LTRfootnote{Lamarckian Mutation}
، و ترکیب اعلان‌ها
\LTRfootnote{Prompt Crossover and Context Shuffling}
 هستند که هر یک روش‌های متفاوتی را برای اصلاح و بهبود اعلان‌ها ارائه می‌دهند. با ادامه این فرآیند در طی چندین نسل، اعلان‌ها به تدریج بهینه شده و به عملکرد بهتری در مدل‌های زبانی منجر می‌شوند.  
 
\begin{figure}[!t]
	\centering
	\includegraphics[width=140mm]{images/promptbreeder}
	\caption{دیاگرام روش مولد اعلان}
	\label{fig_promptbreeder}
\end{figure}

روش مولد اعلان در آزمایش‌های مختلف نشان داده است که می‌تواند از روش‌های متداول مهندسی اعلان مانند CoT و PS عملکرد بهتری داشته باشد. همچنین، این سیستم در حوزه‌های مختلف مانند حل مسائل ریاضی، استدلال عمومی
\LTRfootnote{commonsense reasoning}
، و کلاس‌بندی گفتار نفرت‌آمیز
\LTRfootnote{hate speech classification}
 بهبود چشمگیری ایجاد کرده است. مهم‌ترین مزیت آن این است که به به‌روزرسانی پارامترهای مدل نیازی ندارد و تنها از طریق اصلاح اعلان‌ها، کارایی مدل را افزایش می‌دهد.  

این روش نشان‌دهنده‌ی قدرت بهینه‌سازی خودکار اعلان‌ها و حرکت به‌سوی سیستم‌های هوش مصنوعی خودبهبوددهنده و خودارجاعی است که می‌توانند با حداقل مداخله انسانی، عملکرد خود را در طی زمان بهبود بخشند.


\section{جمع‌بندی مباحث ارائه‌شده}
این فصل با هدف بررسی پیشینه تحقیق و تبیین مفاهیم بنیادی مرتبط با موضوع پژوهش تدوین گردید. در این فصل، تلاش شد تا ضمن ارائه چارچوب نظری جامع، زمینه مناسبی برای درک بهتر مسأله تحقیق و مسیر پژوهش فراهم شود. با توجه به گستره و عمق موضوعات مطرح‌شده، می‌توان این فصل را به‌عنوان شالوده‌ای برای تحلیل‌ها و مطالعات تخصصی‌تر در فصول بعدی قلمداد نمود.

در این فصل، نخست به معرفی و تحلیل مدل‌های زبان بزرگ پرداخته شد و روند تکاملی این مدل‌ها از آغاز تا وضعیت کنونی آن‌ها مورد بررسی قرار گرفت. در ادامه، به معماری‌های رایج و کاربردهای متنوع این مدل‌ها اشاره شد و اهمیت آن‌ها در پیشبرد حوزه‌های مختلف پردازش زبان طبیعی و سامانه‌های مبتنی بر هوش مصنوعی تبیین گردید.

سپس، به موضوع مهندسی اعلان و نقش آن در بهره‌برداری مؤثر از مدل‌های زبان بزرگ پرداخته شد. در این بخش، ابتدا روش‌های سنتی مهندسی اعلان، شامل استفاده از یادگیری درون متنی، روش زنجیره تفکر ، استدلال بدون نمونه آموزشی، برنامه تفکر، بهینه سازی با اعلان و روش برنامه\/ریزی و حل به تفصیل بررسی شدند. در ادامه، به معرفی و تحلیل رویکردهای نوین در بهینه‌سازی خودکار اعلان‌ها پرداخته شد. از جمله این روش‌ها می‌توان به زنجیره تفکر خودکار، مهندسی اعلان خودکار و روش مولد اعلان اشاره نمود که هریک با هدف افزایش دقت و توان استدلال مدل‌های زبانی توسعه یافته‌اند.


بر مبنای مباحث ارائه‌شده، آشکار شد که مهندسی اعلان و بهینه‌سازی آن از مهم‌ترین مؤلفه‌ها در افزایش کارایی مدل‌های زبان بزرگ به شمار می‌رود. این امر اهمیت طراحی الگوریتم‌های نوین و کارآمد برای بهبود ساختار و محتوای اعلان‌ها را دوچندان می‌سازد.

در فصل آتی، به تشریح کامل الگوریتم پیشنهادی پرداخته خواهد شد. این الگوریتم که مولد اعلان ساده نام دارد، یکی از رویکردهای نوین در زمینه بهینه‌سازی و تکامل خودکار اعلان‌ها محسوب می‌شود و مبتنی بر اصول الگوریتم‌های تکاملی و بهبود مستمر اعلان‌ها از طریق تولید، ارزیابی و گزینش نسل‌های مختلف اعلان‌ها طراحی شده است. در فصل آینده، به‌طور دقیق به ساختار، مراحل اجرایی و مزایای این الگوریتم در مقایسه با سایر روش‌ها پرداخته خواهد شد.
%فصل دوم
\cleardoublepage
% !TeX root=SBUKThesis-main.tex
\clearpage
\thispagestyle{empty}
\chapter{روش‌های پیشنهادی}\label{chap3}

\section{مقدمه}
همانطور که گفته شد، 
عملکرد مدل‌های زبانی بزرگ به‌شدت به مهندسی اعلان وابسته است، چرا که دستورات به‌طور مستقیم بر تفسیر مدل از وظایف و همچنین نحوه تولید خروجی تاثیر می\/گذارند. 
بنابراین، دستورات نقش حیاتی در اثربخشی مدل‌های زبانی بزرگ ایفا می‌کنند، فرآیندی که معمولاً از طریق آنچه به‌عنوان مهندسی دستور شناخته می‌شود، بهینه‌سازی می‌گردد.
استراتژی‌های پرکاربردی مانند
 زنجیره تفکر \cite{CoT} 
اثربخشی خود را در بهبود توانمندی‌های استدلالی مدل‌های زبانی با تجزیه مسائل پیچیده به گام‌های میانی نشان داده‌اند
\cite{selfconsistencyimproveschainthought}, \cite{TowardsUnderstandingCoTPrompting}, \cite{cotcollectionimprovingzeroshot}, \cite{revealingmysterychainthought}.
با این حال، استراتژی‌های دستی طراحی‌شده توسط انسان، به دلیل وابستگی به شهود انسانی، غالباً برای وظایف خاص دامنه محور، بهینه نبوده و نمی‌توانند به‌طور کامل از ظرفیت مدل‌های پایه بهره‌برداری کنند.
از سوی دیگر، با پیشرفت مداوم مدل‌های زبانی و تغییر قابلیت‌های آن‌ها، مؤثرترین دستورات نیز ممکن است دستخوش تغییر شوند.
به‌طور کلی، مهندسی اعلان که به‌صورت دستی انجام می‌شود، فرآیندی زمان‌بر بوده و نتایجی کمتر از حد بهینه به همراه دارد.
این محدودیت‌ها موجب شده است که تولید خودکار دستورات به‌عنوان یک حوزهٔ تحقیقاتی مهم در هوش مصنوعی مطرح گردد 
\cite{textpatternseffectivechain}, \cite{APE}.
همانطور که در فصل قبل دیدیم، الگوریتم‌های مختلفی برای بهینه‌سازی دستورات معرفی شده‌اند؛ از جمله
 بهینه\/سازی با اعلان \cite{opro}، مولد اعلان \cite{PromptBreeder} و سایر روش ها \cite{ExploringthePromptSpaceofLLMsthroughEvolutionarySampling}  \cite{epiccosteffectivesearchbasedprompt}.
در این میان، روش مولد اعلان با استفاده از یک چارچوب تکاملی و خودارجاعی، از طریق تکرار فرآیند جهش و ارزیابی یک جمعیت از دستورات وظیفه، به بهینه‌سازی دستورات می‌پردازد.
اگرچه نشان داده شده است که روش مود اعلان برای وظایف استدلالی و طبقه‌بندی مؤثر است، اما پیچیدگی طراحی آن موجب کاهش قابلیت تعمیم این روش می‌شود. افزون بر این، عدم تنوع کافی در دستورات تولید شده می‌تواند چالش‌زا باشد.
وابستگی به چندین لایه بهینه‌سازی، سازوکارهای خودارجاعی و حتی به‌کارگیری تکنیک‌هایی برای فریب مدل زبانی
\footnote{صفحه ۷ مقاله مولد اعلان : «توجه داشته باشید که ما به مدل زبانی ‘دروغ’ گفته‌ایم و بیان کرده‌ایم که ترتیب نزولی است.»}
 ممکن است موجب عدم موفقیت این روش هنگام به‌کارگیری بر روی مدل‌های زبانی جدید شود.

در این مطالعه، جایگزینی ساده‌تر و شفاف‌تر برای مولد اعلان تحت عنوان مولد اعلان ساده
\LTRfootnote{SimplePromptBreeder}
 پیشنهاد می‌شود که ضمن حفظ نقاط قوت اصلی، کاستی‌های آن را نیز برطرف می‌سازد.
با ساده‌سازی چارچوب تکاملی بهینه‌سازی دستورات، روش پیشنهادی، ضمن کاهش پیچیدگی، قابلیت تفسیر و مقیاس‌پذیری را نیز بهبود می‌بخشد (همان‌طور که در شکل~\ref{fig_teaser} نشان داده شده است).
این رویکرد در تلاش است تا تولید خودکار دستورات را کاربردی‌تر و در دسترس‌تر سازد.

 \begin{figure}[!t]
	\centering
	\includegraphics[width=100mm]{images/tf}
	\caption{مقایسه پیچیدگی محاسباتی روش مولد اعلان و روش مولد اعلان ساده}
	\label{fig_teaser}
\end{figure}

مولد اعلان ساده به‌صورت تکرارشونده دستورات را بررسی می‌کند و با بهره‌گیری مستقیم از یک مدل احتمالاتی به نام فرآیندهای نقطه\/ای دترمینانی، به بهینه‌سازی تنوع و کیفیت آن‌ها می‌پردازد. مدل های احتمالی فرآیندهای نقطه\/ای دترمینانی، به‌گونه‌ای طراحی شده‌اند که راهکارهای بهینه‌ای برای مسئله انتخاب در فضای دستورات ارائه دهند به گونه\/ای که هم با کیفیت و هم متنوع باشند.
 
فرضیهٔ تحقیق ما این است که استفاده از فرآیندهای نقطه\/ای دترمینانی می‌تواند تعداد دفعات فراخوانی مدل زبانی بزرگ را برای یافتن دستورات بهینه کاهش داده و در عین حال، سطح عملکرد رقابتی را حفظ نماید.
سادگی و ظرافت ریاضی این رویکرد همچنین می‌تواند موجب افزایش استحکام آن در برابر انواع مدل‌های زبانی شود که این روش بر روی آن‌ها اعمال می‌گردد.
از طرفی  ساختار ریاضی این روش، آن را به جایگزینی عملی و قابل تفسیر برای روش‌های پیچیده تبدیل می‌کند.




\section{روش مولد اعلان ساده}

برای استفاده از مدل‌های زبانی بزرگ با دستورالعمل
\LTRfootnote{instructed LLMs}
 مانند Mistral \cite{mistral}، لازم است اعلان‌ها در قالب خاصی سازمان‌دهی شوند تا حالت مکالمه‌ای
 \LTRfootnote{chat-mode}
  شبیه‌سازی شود.  
این قالب بر پایه‌ی ساختار کلید-مقدار
\LTRfootnote{key-value}
 استوار است، که در آن هر کلید، نقش
 \LTRfootnote{role}
  را مشخص می‌کند و مقدار متناظر، محتوا یا پیام مربوط به آن نقش را شامل می‌شود. سه نقش رایج در این قالب وجود دارد: سیستم، دستیار و کاربر. برای نقش سیستم، محتوا به عنوان دستورالعمل‌هایی برای مدل زبانی عمل می‌کند تا وظایف مشخصی را انجام داده و به اعلان‌های کاربر پاسخ دهد.  
این دستورالعمل‌ها با عنوان اعلان دستوری
\LTRfootnote{instruction prompt}
 شناخته می‌شوند و تأثیر بسزایی بر کیفیت پاسخ‌های تولید شده توسط مدل دارند.

برای کشف این نوع اعلان‌های دستوری، از یک الگوریتم تکاملی استفاده می‌کند که به طور کارآمد یک جستجوی محلی
\LTRfootnote{Local Search}
 را در فضای بی‌نهایت اعلان‌ها انجام می‌دهد.  
 \begin{figure}[!t]
 	\centering
 	\includegraphics[width=140mm]{images/flow}
 	\caption{دیاگرام روش مولد اعلان ساده}
 	\label{fig_flowchart}
 \end{figure}

 
همان‌طور که در شکل \ref{fig_flowchart} نشان داده شده است، مولد اعلان ساده یک چرخه سه مرحله‌ای را به صورت تکراری دنبال می‌کند:

\begin{itemize}[noitemsep]
	\item \textbf{نمونه‌گیری
		\LTRfootnote{Sampling}
		:} مولد اعلان ساده با تولید مجموعه متنوعی از اعلان‌های دستوری شروع می‌کند و اطلاعاتی مانند توصیف وظایف
		\LTRfootnote{task-description}
		، مثال‌ها یا تغییرات مبتنی بر اعلان‌های موفق قبلی را در نظر می‌گیرد.
	\item \textbf{انتخاب
		\LTRfootnote{Selection}
		:} از میان اعلان‌های تولید شده، مولد اعلان ساده زیرمجموعه‌ای متنوع از اعلان‌های با کیفیت را انتخاب می‌کند، به گونه‌ای که هم شباهت به اعلان‌های موفق قبلی حفظ شود و هم تنوع میان اعلان‌های انتخاب شده رعایت گردد.
	\item \textbf{ارزیابی
		\LTRfootnote{Evaluation}
		:} اعلان‌های انتخاب شده بر روی یک مجموعه‌ داده کوچک ارزیابی می‌شوند و بهترین‌ها به مجموعه نامزدها 
	\LTRfootnote{Candidate Set}
	افزوده می‌شوند. این مجموعه به صورت تکراری به‌روز شده و روند انتخاب در آینده را هدایت می‌کند.
\end{itemize}
در طول چندین تکرار
\LTRfootnote{epoch}
، مولد اعلان ساده اعلان‌های موجود در مجموعه نامزدها را پالایش می‌کند تا اطمینان حاصل شود که این اعلان‌ها مؤثر و متناسب با وظیفه موردنظر هستند.



\begin{algorithm}[h]
	\caption{مولد اعلان ساده}
\lr{	\begin{algorithmic}[0]
		\STATE \textbf{Input:}
		\STATE \quad \textbf{problem\_description:} A description of the problem domain.
		\STATE \quad \textbf{epochs:} Number of epochs to repeat the searching method.
		\STATE \quad \textbf{C\_size:} Maximum size of the candidate set $C$.
		\STATE \quad \textbf{DPP\_selection\_size:} Number of prompts selected by DPP.
		\STATE \quad \textbf{Dataset:} A dataset of Q\&A pairs relevant to the problem domain.
		\STATE \textbf{Output:} $C$ (a set of $k$ best evaluated instruction prompts).
	\end{algorithmic}
	\begin{algorithmic}[1]
		\STATE Initialize $C \gets \emptyset$ \COMMENT{Start with an empty candidate set}
		\FOR{epoch = 1 to epochs}
		\STATE Generate sample set $S$ from problem\_description and Dataset.training\_data
		\STATE Select set $P$ of prompts from $S$ by SPB-DPP , $|P|=  DPP\_selection\_size$ 
		\STATE Evaluate prompts in $P$
		\STATE Merge $P$ with $C$
		\STATE Select the $k$ best prompts from $C$ and set them as $C$ for the next epoch
		\ENDFOR
		\STATE \textbf{Return} $C$ \COMMENT{Set of $k$ best evaluated prompts}
	\end{algorithmic}}
	\label{alg_simple_prompt_breeder}
\end{algorithm}



\subsection{نمونه‌گیری}
مرحله نمونه‌گیری مشابه فرآیند ساختن جمعیت اولیه
\LTRfootnote{initialization}
 در الگوریتم‌های تکاملی عمل می‌کند.  
در این مرحله، مجموعه‌ای از دستورالعمل‌های اولیه بالقوه بر اساس توصیف مسئله، جفت‌های پرسش و پاسخ موجود در مجموعه داده به عنوان نمونه، و همچنین دستورالعمل‌های موفق استخراج‌شده از تکرارهای قبلی الگوریتم تولید می‌گردد.

هر مجموعه داده دارای توصیف مسئله مشخصی است که در جدول \ref{tab_des} آورده شده\/اند. 

\begin{table}[h!] 
	\centering
	\begin{tabular}{|>{\centering\arraybackslash}p{4cm}|p{10cm}|}
		\hline
		\textbf{مجموعه داده} & \centering\textbf{توصیف مسئله} \tabularnewline
		\hline
		\lr{SVAMP, SINGLEEQ, ADDSUB, GSM8K, MULTIARITH} & \begin{LTR}\raggedright \fontfamily{pcr}\selectfont Solve the math word problem, giving your answer as an arabic numeral.\end{LTR} \\
		\hline
		\lr{AQUA-RAT} & \begin{LTR}\raggedright \fontfamily{pcr}\selectfont Solve the multiple choice math word problem, choosing (A),(B),(C),(D) or (E).\end{LTR} \\
		\hline
		\lr{CSQA} & \begin{LTR}\raggedright \fontfamily{pcr}\selectfont Solve the multiple choice math word problem, choosing (A),(B),(C),(D) or (E).\end{LTR} \\
		\hline
		\lr{SQA} & \begin{LTR}\raggedright \fontfamily{pcr}\selectfont Work out an answer to the commonsense reasoning question above, and then answer yes or no.\end{LTR} \\
		\hline
	\end{tabular}
	\caption{توصیف مسئله برای مجموعه داده های مختلف} \label{tab_des}
\end{table}




نخستین رویکرد نمونه‌گیری برگرفته از روش یادگیری بدون نمونه \cite{ZSL} است که در آن نمونه‌ای صریح از مجموعه داده در متن اعلان ارائه نمی‌شود و تولید پاسخ صرفاً مبتنی بر توصیف مسئله انجام می‌پذیرد.  
به عنوان مثال، در مجموعه داده GSM8K، توصیف مسئله به صورت زیر بیان شده است:  
\textit{''مسئله کلمه‌ای ریاضی را حل کن و پاسخ را به صورت یک عدد عربی ارائه بده.``}
\LTRfootnote{Solve the math word problem, giving your answer as an arabic numeral.}

در رویکرد دوم نمونه‌گیری، علاوه بر توصیف مسئله، از چندین پرسش و پاسخ به عنوان مثال‌هایی از خروجی مورد انتظار استفاده می‌شود که در روش مولد اعلان ساده این تعداد 5 عدد است. این رویکرد مبتنی بر یادگیری با نمونه های کم \cite{FSL} بوده و با ارائه نمونه‌هایی از خروجی مورد انتظار، به مدل زبانی کمک می‌کند تا درک دقیق‌تری از ماهیت مسئله داشته باشد. به‌کارگیری این نمونه‌ها موجب تولید دستورالعمل‌های اولیه مرتبط‌تر و متناسب‌تر با مسئله خواهد شد.

در نهایت، در سومین رویکرد نمونه‌گیری، به مدل زبانی دستور داده می‌شود تا با استفاده از دستورالعمل‌های موفق و ارزیابی‌شده از تکرارهای پیشین، که به عنوان مجموعه کاندیدا شناخته می‌شوند، تغییرات و نسخه‌های جدیدی از این دستورالعمل‌ها تولید نماید.

در مجموع، بهره‌گیری از این سه رویکرد نمونه‌گیری منجر به ایجاد مجموعه‌ای متنوع و جامع از دستورالعمل‌های اولیه بالقوه می‌گردد. جزئیات بیشتر در خصوص این سه رویکرد نمونه‌گیری در ضمیمه B: دستورالعمل‌های اولیه برای سه رویکرد نمونه‌گیری ارائه شده است.

\subsection{انتخاب}

تعیین دقت یک دستورالعمل مستلزم محاسبات قابل توجهی است، چرا که برای ارزیابی، لازم است هر دستورالعمل بر روی مجموعه آزمون اجرا شود. 
از این رو، ضروری است تا زیرمجموعه‌ای از دستورالعمل‌های نمونه‌گیری‌شده از مرحله قبلی، که دارای بیشترین پتانسیل هستند، برای ارزیابی انتخاب شوند.
این زیرمجموعه باید متنوع باشد تا امکان کاوش بیشتر فراهم شود و در عین حال به دستورالعمل‌های موفق پیشین و مجموعه کاندید نزدیک باقی بماند.
یک روش ساده نظیر انتخاب صرفاً دستورالعمل‌هایی با بیشترین دقت، منجر به کاهش تنوع و همگرایی سریع به سمت بهینه محلی خواهد شد.
برای مقابله با این مشکل، روش مولد اعلان ساده از مدل احتمالاتی فرآیندهای نقطه\/ای دترمینانی بهره می‌گیرد.

\noindent\textbf{مدل فرآیندهای نقطه\/ای دترمینانی }
\cite{DPP}
 ، مدل‌های احتمالی هستند که برای انتخاب زیرمجموعه‌هایی متنوع و در عین حال مرتبط از یک مجموعه بزرگ‌تر طراحی شده‌اند. 
این مدل‌ها در ابتدا برای فیزیک کوانتوم توسعه یافته‌اند، اما بعدها در مسائل یادگیری ماشین نظیر خلاصه‌سازی اسناد، سیستم‌های توصیه‌گر، انتخاب ویژگی و انتخاب زیرمجموعه‌های داده به کار گرفته شده‌اند \cite{DPP_for_ML}.
نکته مشترک در این کاربردها، بهینه‌سازی همزمان دو معیار تنوع
\LTRfootnote{Diversity}
 و کیفیت
\LTRfootnote{Quality}
  است.
در فرآیندهای نقطه\/ای دترمینانی، یک ماتریس کرنل
\LTRfootnote{Kernel Matrix}
 نیمه مثبت معین
 \LTRfootnote{Positive semi-definit}
  \( L \) تعریف می‌شود، که هر درایه \( L_{ij} \) میزان شباهت
  \LTRfootnote{Similarity}
   بین عناصر \( i \) و \( j \) را نشان می‌دهد. احتمال انتخاب زیرمجموعه‌ای \( S \subseteq Y \)، که \( Y \) مجموعه اصلی است، متناسب با دترمینان زیرماتریس \( L_S \) خواهد بود، که متناظر با سطرها و ستون‌های شاخص‌گذاری‌شده توسط \( S \) است:
\begin{equation}
	P(S) \propto \det(L_S)
\end{equation}
در این روش، دترمینان به عنوان معیار تنوع عمل می‌کند و زیرمجموعه‌هایی با عناصر نامشابه‌تر را ترجیح می‌دهد. به‌طور شهودی، زیرمجموعه‌هایی با عناصر نامشابه‌تر دارای دترمینان بزرگ‌تری هستند، چرا که بردارهای متعامد
\LTRfootnote{orthogonal}
 یا تقریباً متعامد
 \LTRfootnote{near-orthogonal}
  در ماتریس کرنل، به دترمینان بزرگ‌تری منجر می‌شوند.

\noindent\textbf{مدل فرآیندهای نقطه\/ای دترمینانی شرطی}
 ، مدلی احتمالی است که به زیرمجموعه‌ای از آیتم‌ها، با توجه به ورودی یا قیودی مشخص، احتمال تخصیص می‌دهد.
در روش مولد اعلان ساده، فرآیندهای نقطه\/ای دترمینانی شرطی توازن میان کیفیت (برحسب شباهت به دستورالعمل‌های موفق) و تنوع (برحسب عدم شباهت بین اعضا زیرمجموعه) را برقرار می‌سازد.

به زبان ریاضی، احتمال انتخاب زیرمجموعه \( Y \) از مجموعه اصلی \( \mathcal{Y}(X) \) با توجه به ورودی \( X \) متناسب با دترمینان یک ماتریس کرنل نیمه مثبت معین \( L_Y(X) \) خواهد بود.

\noindent فرض کنید \( \mathcal{X} \) فضای ورودی (مثلاً مجموعه نمونه‌ها) و \( \mathcal{Y}(X) \) مجموعه آیتم‌ها با توجه به ورودی \( X \in \mathcal{X} \) (مثلاً مجموعه دستورالعمل‌های نمونه‌گیری‌شده) باشد.
تعریف رسمی به صورت زیر خواهد بود:

\noindent
\textbf{تعریف: } یک فرآیندهای نقطه\/ای دترمینانی شرطی به صورت \( P(\textbf{Y} = Y |X) \) مدلی احتمالاتی است که به هر زیرمجموعه ممکن \( Y \subseteq \mathcal{Y}(X) \) احتمال تخصیص می‌دهد و به صورت زیر تعریف می‌شود:
\begin{equation}
	P(\textbf{Y} = Y |X) \propto \det(L_{Y} (X)),
\end{equation} 
که در آن \( L_{Y}(X) \) ماتریس کرنل نیمه مثبت معینی با ابعاد \( \left|\mathcal{Y}(X) \right| \times \left|\mathcal{Y}(X)\right| \) و وابسته به \( X \) است.

ثابت نرمال‌سازی فرآیندهای نقطه\/ای دترمینانی شرطی به صورت کارآمد با رابطه \( \det(L(X)+I) \) محاسبه می‌شود. اگر تنوع برابر \(\phi\) و کیفیت برابر $q$ باشد، با استفاده از تجزیه کیفیت-تنوع، خواهیم داشت:

\begin{equation}
	L_{ij} (X) = q_{i}(X) \phi_{i}(X)^{T} \phi_{j}(X)q_{j}(X) 
\end{equation} 
که در آن \( q_{i}(X) \in  \mathbb{R}^{+} \) و \( \phi_{i}(X) \in \mathbb{R}^{D} \) با شرط \( \|\phi_{i}(X)\|=1 \) تعریف شده‌اند و هر دو به \( X \) وابسته هستند.


روش مولد اعلان ساده ، تجزیه‌ای از فرآیندهای نقطه\/ای دترمینانی شرطی به کار گرفته است که به‌طور صریح توازن میان تنوع و کیفیت آیتم‌ها را نشان می‌دهد.
برای این منظور، ساخت ماتریس کرنل بر مبنای دو مؤلفه کیفیت و تنوع انجام می‌شود.
پارامتر تنوع از طریق شباهت زوجی میان دستورالعمل‌های موجود در مجموعه نمونه
\LTRfootnote{Sample set}
 با ماتریسی با نام \( \Phi \) از ابعاد \( n \times n \) سنجیده می‌شود.
برای محاسبه \( \Phi \)، ابتدا یک ماتریس صفر از ابعاد \( n \times n \) ایجاد می‌شود و سپس برای هر جفت از دستورالعمل‌ها \( (p_i, p_j) \) در مجموعه \( S = \{p_1, p_2, \dots, p_n\} \)، بردار‌های عددی \( e_i \) و \( e_j \) با استفاده از یک مدل sentence transformer استخراج می‌شوند.
سپس شباهت کسینوسی بین \( e_i \) و \( e_j \) مطابق رابطه زیر محاسبه می‌شود:
\begin{equation}
	\label{eq:similarity}
	\text{similarity}(e_i, e_j) = \frac{\sum_{k=1}^{d} e_i^k \cdot e_j^k}{\sqrt{\sum_{k=1}^{d} (e_i^k)^2} \cdot \sqrt{\sum_{k=1}^{d} (e_j^k)^2}}
\end{equation}
که در آن \( d \) ابعاد بردارهای عددی است. 
در نهایت:
\begin{equation}
	\Phi[i][j] = \text{similarity}(e_i, e_j)
\end{equation}

\noindent از طرف دیگر، ماتریس
 \( Q_d \) 
 که نشان دهنده  پارامتر کیفیت است، از ابعاد \( n \times 1 \) است که فاصله هر دستورالعمل \( S = \{p_1, p_2, \dots, p_n\} \) را از مجموعه دستورالعمل‌های موفق \( S_{success} \) می‌سنجد.
فرآیند محاسبه \( Q_d \) با محاسبه شباهت کسینوسی بین هر \( p_i \in S \) و هر \( p_j \in S_{success} \) با استفاده از رابطه \ref{eq:similarity} آغاز می‌شود.
سپس:
\begin{equation}
	\mu_i = \frac{1}{m} \sum_{k=1}^{m} s_{ik}
\end{equation}
و انحرافات مثبت از میانگین محاسبه می‌گردد:
\begin{equation}
	s'_{ik} = \max(0, s_{ik} - \mu_i)
\end{equation}
که سپس با تابع softmax نرمال‌سازی می‌شود:
\begin{equation}
	w_{ik} = \frac{\exp(s'_{ik})}{\sum_{j=1}^{m} \exp(s'_{ij})}.
\end{equation}
وزن نهایی هر دستورالعمل \( p_i \) به صورت زیر محاسبه می‌شود:
\begin{equation}
	W_i = \sum_{k=1}^{m} w_{ik}
\end{equation}
و در درایه \( i \)ام قطر اصلی \( Q_d \) قرار می‌گیرد.

\noindent در نهایت، این روش ماتریس کرنل را به صورت زیر می‌سازد:
\begin{equation}
	L = Q_{d} \otimes
	\Phi \otimes
	Q_{d}
\end{equation}
که در آن نماد \( \otimes \) به معنای ضرب عنصر به عنصر
\LTRfootnote{element-wise product}
 است.
سپس، از این ماتریس در فرآیندهای نقطه\/ای دترمینانی برای انتخاب زیرمجموعه بهینه از نظر تنوع و کیفیت استفاده می‌شود.
در این فرآیند، زیرمجموعه‌هایی که نقاط انتخابی آنها پراکندگی بیشتری دارند احتمال بالاتری خواهند داشت.




\subsection{ارزیابی}
برای ارزیابی
\LTRfootnote{Evaluation}
از معیار دقت
\LTRfootnote{Accuracy}
 استفاده شده است. برای بدست آوردن دقت اعلان های زیرمجموعهٔ انتخاب‌شده توسط فرآیندهای نقطه\/ای دترمینانی ، زیرمجموعه‌ای شامل ۱۰۰ جفت پرسش و پاسخ از هر مجموعه داده به‌عنوان مجموعهٔ اعتبارسنجی
 \LTRfootnote{Validation set}
  مورد استفاده قرار می‌گیرد. 
هر دستور آموزشی در زیرمجموعهٔ انتخاب‌شده برای هدایت مدل زبانی بزرگ به کار می‌رود. سپس این مدل زبانی هدایت‌شده، برای هر پرسش در مجموعهٔ اعتبارسنجی، پاسخ‌هایی تولید می‌کند که به آن‌ها پاسخ های پیش\/بینی\/شده گفته می‌شود.
این پاسخ‌های پیش‌بینی‌شده سپس نیاز به اعتبارسنجی دارند. برای اعتبارسنجی، از رویکردی تحت عنوان مدل زبانی به‌عنوان داور
\LTRfootnote{LLM-as-a-Judge}
 استفاده می‌کنیم که در آن، پاسخ پیش‌بینی‌شده با پاسخ واقعی مقایسه می‌شود. ما از مدل Mistral به‌عنوان داور استفاده می‌کنیم.

\noindent جریان کلی این فرآیند در شکل
 \ref{fig_evaluation} با استفاده از نمونه\/ای از مجموعه‌دادهٔ GSM8K  نشان داده شده است.

\noindent همانطور که دیده می\/شود، پس از ارزیابی دستورات آموزشی انتخاب‌شده، آن‌ها با دستورات آموزشی موفق از آخرین دورهٔ آموزشی ادغام می‌شوند.
علاوه بر این، بهترین دستورات آموزشی برتر (\(k\) مورد بر اساس دقت آزمایشی) به‌عنوان دستورات آموزشی موفق جدید انتخاب می‌شوند تا در تکرار های بعدی الگوریتم مورد استفاده قرار گیرند. مقدار \(k\) در روش مولد اعلان ساده برابر 10 است، این مقدار با آزمون و خطای مختلف بدست آمده است.

 \begin{figure}[!t]
	\centering
	\includegraphics[width=90mm]{images/evaluation}
	\caption{دیاگرام مرحله ارزیابی روش مولد اعلان ساده}
	\label{fig_evaluation}
\end{figure}
























%فصل سوم
\cleardoublepage
% !TeX root=SBUKThesis-main.tex
\chapter{نتایج و بحث}\label{chap4}
\section{مجموعه داده های مورد استفاده}

روش مولد اعلان ساده را در مجموعه‌ای گسترده از وظایف مرتبط با استدلال حسابی
\LTRfootnote{arithmetic reasoning}
 و استدلال مبتنی بر درک مسئله
 \LTRfootnote{commonsense reasoning}
  مورد ارزیابی قرار داده\/ایم. برای این منظور از هشت مجموعه داده‌ی عمومی، شامل 
  GSM8K \cite{GSM8k}
   ،
  SVAMP \cite{SVAMP}
   ،
  MultiArith \cite{MultiArith}
   ،
  AddSub \cite{AddSub}
   ،
  AQuA-RAT \cite{AquaRat}
   و
  SingleEq \cite{SingleEQ}
  
   به همراه CommonsenseQA \cite{CSQA} و StrategyQA \cite{SQA} استفاده شده است.
   
همانطور که در جدول \ref{tab_Results} می\/بینید، روش مولد اعلان ساده با روش‌های زنجیره تفکر، برنامه\/ریزی و حل با اعلان بهبود یافته و بدون آن، مهندس اعلان خودکار و بهینه\/سازی با اعلان ها مورد مقایسه قرار گرفته است. بالاترین دقت برای هر مجموعه داده به صورت فونت برجسته مشخص شده است. نتایج مدل زبان text-davinci-003 مستقیماً از مقاله‌ی برنامه\/ریزی و حل \cite{PS} استخراج شده و نتایج مدل زبان PaLM 2-L نیز به طور مستقیم از مقاله‌ی مولد اعلان \cite{PromptBreeder} گرفته شده‌اند. جهت تضمین مقایسه‌ای عادلانه، تمامی روش ها با مدل زبانی Mistral ارزیابی شده\/اند و سپس نتایج مقایسه شده است.

\begin{table}[h!]
		\centering
		{\resizebox{\textwidth}{!}{%
			\begin{LTR} % Begin left-to-right environment
			\begin{tabular}{c|ll|ccccccccc}
				\textbf{Results from} &\textbf{Method}  &\textbf{LLM}  & MultiArith  &SingleEq  &AddSub  &SVAMP  &SQA  &CSQA  &AQuA-RAT &GSM8K \\\hline
				\multirow{4}{*}{PS \cite{PS}}
				&COT  &text-davinci &\lr{83.8}  &\lr{88.1}  &\lr{85.3}  &\lr{69.9}  &\lr{63.8}  &\lr{65.2}  &\lr{38.9}  &\lr{56.4} \\
				&POT &text-davinci &\lr{92.2}  &\lr{91.7}  &\lr{85.1}  &\lr{70.8}  &-  &-  &\lr{43.9}  &\lr{57.0} \\
				&PS &text-davinci &\lr{87.2}  &\lr{89.2}  &\lr{88.1}  &\lr{72.0}  &-  &-  &\lr{42.5}  &\lr{58.2} \\
				&PS+ &text-davinci &\lr{91.8}  &\lr{94.7}  &\lr{92.2}  &\lr{75.7}  &\lr{65.4}  &\lr{71.9}  &\lr{46.0}  &\lr{59.3} \\\hline
				\multirow{5}{*}{PB\cite{PromptBreeder}}
				&PS &PaLM 2-L  &\lr{97.7}  &\lr{90.6}  &\lr{72.4}  &\lr{83.8}  &\lr{50.0}  &\lr{77.9}  &\lr{40.2}  &\lr{59.0} \\
				&PS+ &PaLM 2-L  &\lr{92.5}  &\lr{94.7}  &\lr{74.4}  &\lr{86.3}  &\lr{50.1}  &\lr{73.3}  &\lr{39.4}  &\lr{60.5} \\
				&APE &PaLM 2-L  &\lr{95.8}  &\lr{82.2}  &\lr{72.2}  &\lr{73.0}  &\lr{38.4}  &\lr{67.3}  &\lr{45.7}  &\lr{77.9} \\
				&OPRO &PaLM 2-L  &-  &-  &-  &-  &-  &-  &-  &80.2 \\
				&PB &PaLM 2-L  &\textbf{\lr{99.7}}  &\textbf{\lr{96.4}}  &\lr{87.8}  &\lr{90.2}  &\lr{71.8}  &\lr{85.4}  &\lr{62.2}  &\lr{83.9} \\\hline
				\multirow{6}{*}{\text{مولد اعلان ساده(us)}}
				&CoT &Mistral  &\lr{79.0}  &\lr{85.0}  &\lr{78.0}  &\lr{64.0}  &\lr{54.0}  &\lr{51.0}  &\lr{33.0}  &\lr{72.0} \\
				&PS &Mistral  &\lr{68.0}  &\lr{91.0}  &\lr{78.0}  &\lr{68.0}  &\lr{61.0}  &\lr{43.0}  &\lr{36.0}  &\lr{66.0} \\
				&PS+ &Mistral  &\lr{82.0}  &\lr{84.0}  &\lr{80.0}  &\lr{62.0}  &\lr{44.0}  &\lr{27.0}  &\lr{26.0}  &\lr{60.0} \\
				&APE &Mistral  &\lr{78.0}  &\lr{86.0}  &\lr{84.0}  &\lr{69.0}  &\lr{54.0 } &\lr{51.0}  &\lr{36.0}  &\lr{66.0} \\
				&OPRO &Mistral  &\lr{85.0}  &\lr{84.0}  &\lr{78.0}  &\lr{61.0}  &\lr{53.0}  &\lr{35.0}  &\lr{34.0}  &\lr{67.0} \\
				&مولد اعلان ساده(ours) &Mistral  &\cellcolor{mybluecolor!20}\lr{92.0 } &\cellcolor{mybluecolor!60}\lr{94.0 } &\cellcolor{mybluecolor!90}\textbf{\lr{89.0}}  &\cellcolor{mybluecolor!90}\textbf{\lr{93.0}}  &\cellcolor{mybluecolor!90}\textbf{\lr{92.0}}  &\cellcolor{mybluecolor!90}\textbf{\lr{91.0}}  &\cellcolor{mybluecolor!90}\textbf{\lr{93.0}}  &\cellcolor{mybluecolor!90}\textbf{\lr{84.0}} \\\hline
			\end{tabular}
		\end{LTR}
	}
		}
		\caption{مقایسه دقت روش مولد اعلان ساده با سایر روش های موجود} 
		\label{tab_Results}
	 % End left-to-right environment
\end{table}

\section{متغیرهای روش مولد اعلان ساده}
 از آن‌جایی که مولد اعلان ساده یک الگوریتم تکاملی است، این الگوریتم را در بازه ۲۰ تا ۳۰ تکرار
 \LTRfootnote{Epoch}
  اجرا می‌کنیم؛ تا جایی ‌که در آخرین تکرار‌ها، مجموعه کاندیدا تغییر نخواهد کرد. در هر دوره، جمعیتی متشکل از ۱۲۰ تا ۱۵۰ اعلان دستوری، بسته به مجموعه کاندیدا، تولید می‌شود. همانطور که در فصل قبل گفته شد، این اعلان‌ها بر اساس توضیحات مسئله و یک مجموعه آموزشی شامل ۲۰ سوال و پاسخ از مجموعه داده، نمونه‌گیری می‌شوند. از میان این جمعیت، زیرمجموعه‌ای شامل ۱۰ اعلان دستوری انتخاب شده و بر روی یک مجموعه اعتبارسنجی که شامل ۲۰ سوال و پاسخ متمایز از مجموعه آموزشی است، ارزیابی می‌گردد.

\noindent پس از اتمام تمامی دوره‌ها، اعلان‌های دستوری با بهترین عملکرد بر روی مجموعه‌ای شامل ۱۰۰  سوال و پاسخ که به‌طور تصادفی از مجموعه داده انتخاب شده و از هر دو مجموعه آموزشی و اعتبارسنجی متمایز می‌باشند، مورد آزمون قرار می‌گیرند.

\noindent
 پارامتر توضیحات مسئله به مجموعه داده بستگی دارد که در جدول \ref{tab_des} توضیح متناظر با هر مجموعه داده آورده شده است.
 
 در این پژوهش، از مدل زبانی Mistral-7B \cite{mistral} برای تولید و آزمون اعلان‌های دستوری استفاده می‌شود. همچنین، به منظور تبدیل اعلان‌های دستوری به بردار عددی، از مدل تبدیل جمله‌ی
 \LTRfootnote{Sentence Embedding}
 \lr{bert-base-nli-mean-tokens}
   موجود در سایت HuggingFace بهره گرفته شده است.
   
   
\section{روش های مرجع}
    نتایج روش مولد اعلان ساده با استفاده از پیشرفته‌ترین روش‌های مهندسی که در فصل های قبلی به تفصیل توضیح داده شده اند، مورد مقایسه قرار گرفته‌اند. به‌منظور تضمین مقایسه‌ای عادلانه، در صورت امکان، این روش‌ها با استفاده از مدل پایه Mistral مجدداً اجرا شده‌اند زیرا امکان اجرای روش مولد اعلان ساده با مدل های text-davinci-003 و PaLM 2-L  وجود ندارد چرا که این مدل‌ها به صورت عمومی در دسترس نمی‌باشند و روش های دسترسی به آن‌ها منسوخ شده است. اعلان‌های دستوری این روش‌ها به‌طور مستقیم از مقاله مولد اعلان استخراج شده‌اند؛ اما اعلان‌های تولید شده توسط مولد اعلان مورد آزمون قرار نگرفته‌اند زیرا آن‌ها به‌گونه‌ای برای PaLM 2 بهینه‌سازی شده‌اند که استفاده از آن‌ها در مدل Mistral مقایسه‌ای منصفانه ارائه نخواهد داد.
   
لازم به یادآوری است که برای هر مجموعه داده و هر روش، از دقت به عنوان معیار ارزیابی استفاده شده است.
     
\section{نمایش دوبعدی اعلان ها}
برای درک بهتر عملکرد مدل احتمالاتی فرآیندهای نقطه\/ای دترمینانی در طول تکرارها، در شکل \ref{fig_visualize-dpp} توزیع دو بعدی بردار های عددی اعلان ها را نمایش داده‌ایم.
برای کاهش ابعاد این  بردار های عددی، ابتدا از روش کاهش بعد PCA \cite{pca} برای کاهش ابعاد به ۵۰ و سپس از روش t-SNE \cite{tsne} برای کاهش 50 بعد به ۲ بعد استفاده کردیم.
همانطور که دیده می\/شود، توزیع یکنواخت دستورالعمل‌های انتخاب‌شده در نواحی مختلف فضای جستجو نمایش داده شده است که بیانگر توازن میان تنوع و کیفیت در این روش است.

\begin{figure}[!ht]
	\centering
	\includegraphics[width=140mm]{images/eval}
	\caption{نمایش دو بعدی بردار اعلان ها برای ارزیابی انتخاب در روش مولد اعلان ساده}
	\label{fig_visualize-dpp}
\end{figure}

     
\section{نتایج}
نتایج اجرای روش مولد اعلان ساده بر روی مجموعه داده‌های مذکور نشان می‌دهد که این روش از پیشرفته‌ترین روش‌های مهندسی برتری دارد. همانطور که در جدول \ref{tab_Results} مشاهده می‌شود، مولد اعلان ساده در تمامی مجموعه داده‌ها از سایر روش ها برتر است.

ابتدا عملکرد هر روش را در میان مدل‌های زبانی مختلف ارزیابی کردیم. مشاهده می‌شود که اکثر روش‌ها دقت مشابهی در میان مدل‌های مختلف دارند، اگرچه بهبودهایی در عملکرد هنگام تغییر از 
\lr{text-davinci-003 }
به 
\lr{PaLM 2-L} 
و از 
\lr{PaLM 2-L}
 به 
\lr{Mistral}
  دیده می‌شود؛ این بهبودها را می‌توان به پیشرفت‌های موجود در دقت و ساختار مدل‌های زبانی نسبت داد. 

برای تضمین صحت مقایسه، از روش برنامه\/ریزی و حل به عنوان نقطه مرجع استفاده شده و بهبود نسبی مولد اعلان و مولد اعلان ساده محاسبه گردیده است. در روش مولد اعلان، با استفاده از فرمول زیر به بهبود نسبی
 \lr{23.4 ٪}
  نسبت به روش برنامه\/ریزی و حل در میان مجموعه داده‌ها دست یافتیم:

\[
\begin{split}
	\text{بهبود نسبی مولد اعلان} = \frac{1}{8} \Bigg[
	& \frac{99.7}{97.7} + \frac{96.4}{90.6} + \frac{87.8}{72.4}  + \frac{90.2}{83.8} \\
	& + \frac{71.8}{50.0} + \frac{85.4}{77.9} + \frac{62.2}{40.2} + \frac{83.9}{59.0}
	\Bigg],
\end{split}
\]
در همین حال، برای مولد اعلان ساده با فرمول زیر، به بهبود نسبی
\lr{54.7 ٪} 
 نسبت به روش برنامه\/ریزی و حل در میان مجموعه داده‌ها دست یافتیم:

     
\[
 \begin{split}
   	\text{بهبود نسبی مولد اعلان ساده} = \frac{1}{8} \Bigg[
   	& \frac{92.0}{68.0} + \frac{94.0}{91.0} + \frac{89.0}{78.0} + \frac{93.0}{68.0}\\
   	& + \frac{92.0}{61.0} + \frac{91.0}{43.0}  + \frac{93.0}{36.0} + \frac{84.0}{66.0}
   	\Bigg],
 \end{split}
\]
     
     
\section{سربار محاسباتی}
همانطور که در بخش پارامتر های روش مولد اعلان ساده توضیح داده شد، مولد اعلان ساده به‌طور مشابه با مولد اعلان در بازه ۲۰ تا ۳۰ تکرار اجرا می‌شود. در هر تکرار، مولد اعلان ساده جمعیتی متشکل از ۱۰۰ تا ۱۲۰ اعلان دستوری تولید می‌کند که برای هر یک، یک فراخوانی مدل زبانی مورد نیاز است. در مقابل، مولد اعلان با جمعیتی متشکل از ۵۰ واحد عمل می‌کند؛ هر واحد شامل یک جفت اعلان است که منجر به ۱۰۰ فراخوانی مدل زبانی برای تولید آن‌ها می‌شود. در گام بعدی، برای انتخاب و جهش، مولد اعلان از رویکرد ترکیبی جهش و ابرجهش استفاده می‌کند که برای هر آیتم از واحدهای جمعیت، حداقل دو فراخوانی مدل زبانی نیاز دارد. بنابراین، تعداد کل فراخوانی‌های مدل زبانی در این مرحله برای مولد اعلان به صورت زیر محاسبه می‌شود:
\[
50 \times 2 \times 2 = 200.
\]
در مقابل، مولد اعلان ساده این محاسبات را با استفاده از محاسبات روش DPP انجام می‌دهد و از فراخوانی‌های اضافی مدل زبانی در این گام صرف نظر می‌کند.

در انتها، در گام ارزیابی، مولد اعلان کل جمعیت را مورد ارزیابی قرار داده و به هر واحد نمره می‌دهد. از آنجا که هر واحد در جمعیت مولد اعلان شامل یک جفت اعلان است، ارزیابی ۵۰ واحد نیازمند:
\[
50 \times 2 = 100
\]
فراخوانی مدل زبانی می‌باشد. اما مولد اعلان ساده تنها ۱۰ اعلان دستوری از میان ۱۰۰ اعلان برای ارزیابی انتخاب می‌کند و هر اعلان یک فراخوانی مدل زبانی نیاز دارد؛ بنابراین:
\[
10 \times 1 = 10.
\]
از سوی دیگر، اندازه‌ی مجموعه داده اعتبارسنجی برای مولد اعلان شامل ۱۰۰ جفت سوال و پاسخ است، در حالی که برای مولد اعلان ساده به ۲۰ کاهش یافته است. علاوه بر این، مولد اعلان ساده پس از شناسایی اعلان‌های با عملکرد برتر، گام آزمون را به منظور تضمین مقایسه‌ای منصفانه اضافه می‌کند؛ این گام اعلان‌های انتخاب‌شده را بر روی مجموعه داده‌ای شامل ۱۰۰ جفت سوال و پاسخ مانند مولد اعلان ارزیابی می‌کند.

\noindent تعداد کل فراخوانی‌های مدل زبانی برای مولد اعلان و مولد اعلان ساده در هر دوره به صورت زیر بیان می‌شود:
\[
\text{کل فراخوانی‌های مدل زبانی برای مولد اعلان} = 100 + 200 + 100 = 400,
\]
\[
\text{کل فراخوانی‌های مدل زبانی برای مولد اعلان ساده} = 100 \text{ تا } 120 + 10 = 110 \text{ تا } 130.
\]
بنابراین، مولد اعلان ساده سربار محاسباتی را تقریباً به میزان ۳ برابر کاهش می‌دهد.
    
    
\section{خروجی ها}
با بررسی خروجی های روش مولد اعلان ساده ، مشاهده شد که اعلان‌های دستوری با بهترین عملکرد برای هر مجموعه داده، در قالب چارچوبی صریح ساختاربندی شده‌اند. این چارچوب شامل چند بخش است: بخش اول، توضیح مفصل وظیفه می‌باشد که با دقت به تشریح آن می‌پردازد و در برخی موارد شامل دستورالعمل‌هایی است که جنبه‌های مختلف وظیفه را توضیح می‌دهند. بخش دوم این اعلان، مسیر حل مسئله با توضیحات گام به گام جهت رفع مشکلات موجود در حوزه مربوطه است. در مرحله بعد، تعدادی مثال ارائه شده است که به عنوان نمایش نمونه از مجموعه آموزشی با ارائه راه‌حل کامل و گام به گام، جهت هدایت روند تفکر مدل زبانی به کار رفته‌اند. 

برای مثال، اعلان دستوری با بهترین عملکرد برای مجموعه داده MultiArith، در ابتدا وظیفه را به‌طور کامل تعریف می‌کند، سپس پنج مثال به همراه راه‌حل کامل آن‌ها ارائه می‌دهد و در نهایت، دستورالعمل‌هایی برای حل مسائل در این حوزه بیان می‌نماید. متن کامل این اعلان دستوری در کادر \ref{p_MutiArith} آورده شده است.



\begin{tcolorbox}[breakable,colframe=mybluecolor!100, colback=mybluecolor!20, title=بهترین اعلان دستوری تولید شده برای مجموعه داده MultiArith] \label{p_MutiArith}
	\begin{LTR}
	\textbf{Task:} Analyze the given math problem, break down the information, and compute the answer step by step. Present the final answer as an Arabic number.
	
	\textbf{Examples:}
	\begin{enumerate}
		\item \lr{Understand that April makes \$9 for every rose she sells. Estimate how many roses April started with, then calculate how many roses she ended with, in order to figure out the revenue she generated.}
		\begin{itemize}
			\item Compute the difference between the number of roses April started with and the number of roses she ended with to find the quantity of roses April sold.
			\item Multiply the number of roses sold by the cost per rose to determine the earnings incurred.
		\end{itemize}
		
		\item \lr{Understand the scenario in which Megan prepared 68 cupcakes, and her brother, Todd, consumed 32 of them. Find out how many packages Megan can assemble with 6 cupcakes per package.}
		\begin{itemize}
			\item Subtract the number of cupcakes Todd consumed from the total number of cupcakes Megan prepared to determine the remaining amount of uneaten cupcakes.
			\item Divide the quantity of uneaten cupcakes by the number of cupcakes needed to make one package to find the number of packages Megan can assemble.
		\end{itemize}
		
		\item \lr{Identify the situation in which Katie organized and counted 9 albums with a consistent number of pictures in each album. Determine the number of pictures in each album.}
		\begin{itemize}
			\item Sum the total number of pictures Katie has to obtain the grand total.
			\item Divide the grand total by the number of albums to figure out the number of pictures in each album.
		\end{itemize}
		
		\item \lr{Grasp the details that Tiffany earns 6 points for each treasure she discovers. Figure out the point score Tiffany received from finding treasures on the first and second levels.}
		\begin{itemize}
			\item Add the number of treasures Tiffany discovered on the first and second levels to get the sum of treasures found.
			\item Multiply the total number of treasures by the points per treasure to learn the point score Tiffany attained.
		\end{itemize}
		
		\item \lr{Recall the situation in which a waiter manages 33 customers. Estimate how many customers the waiter is still serving after some prospective diners left and others joined the restaurant.}
		\begin{itemize}
			\item Subtract the number of customers who left from the current number of customers to determine the number of customers still at the restaurant.
			\item Add the number of new customers who arrived to find the updated number of customers the waiter is currently serving.
		\end{itemize}
	\end{enumerate}
	
	\textbf{Instructions:}
	\begin{enumerate}[label=\arabic*.]
		\item Explore the given problem to recognize the necessary computations needed to derive the solution.
		\item Break down the problem into more manageable parts for better organization and clarity.
		\item Calculate the solutions step by step.
		\item Deliver the numerical solution.
	\end{enumerate}
	\end{LTR}

\end{tcolorbox}
     
     
     
     
     
     
     
     
     
     
     
     %فصل چهارم
\cleardoublepage
% !TeX root=SBUKThesis-main.tex
\clearpage
\thispagestyle{empty}
\chapter{نتیجه گیری و پیشنهادات آتی}\label{chap5}

\section{نتیجه‌گیری}
در این پژوهش، یک مدل چندوجهی مبتنی بر ترنسفورمر به نام MAGNET برای تشخیص بدافزار اندروید پیشنهاد شد. این مدل از سه ماژول اصلی تشکیل شده است: EnhancedTabTransformer برای پردازش ویژگی‌های جدولی، GraphTransformer برای تحلیل گراف فراخوانی، و SequenceTransformer برای پردازش توالی‌های API. برای بهینه‌سازی پارامترها از الگوریتم‌های Adam و CosineAnnealingWarmRestarts استفاده شد. همچنین، بهینه‌سازی با روش‌های بهینه‌سازی (۴۷۶ آزمایش) و Optuna (۱۳ آزمایش) پیاده‌سازی شد. دیتاست DREBIN \cite{Drebin} با ۶،۰۹۲ نمونه (۴،۶۴۱ برای آموزش و ۱،۴۵۱ برای تست) برای ارزیابی مدل به‌کار گرفته شد. نتایج نشان داد که مدل MAGNET با دقت ۹۷.۲۴٪، F1 Score ۰.۹۸۲۳ و AUC ۰.۹۹۳۲ عملکرد برتری نسبت به روش‌های پایه دارد.

مقایسه با روش‌های دیگر نشان داد که MAGNET از روش چندوجهی \cite{Alsaleh2023} با دقت ۸۹.۲٪ و روش مبتنی بر ترنسفورمر \cite{TransformerMalware} با دقت ۹۵.۸٪ بهتر عمل می‌کند. همچنین، عملکرد مدل از روش‌های سنتی مانند SVM \cite{ZhangNix2017} و CNN \cite{Vinayakumar2019} به طور قابل توجهی بهتر بود. با این حال، تفاوت‌های جزئی با روش مبتنی بر ترنسفورمر مشاهده شد که می‌تواند به دلیل تفاوت در معماری و پارامترهای مدل باشد.

برای بهبود بیشتر مدل MAGNET، پیشنهاد می‌شود که تعادل کلاس‌ها در دیتاست بهبود یابد و معماری مدل برای دیتاست‌های بزرگ‌تر و متنوع‌تر گسترش یابد. همچنین، آزمایش مدل با داده‌های پویا (مانند الگوهای زمان‌بندی API) و بررسی مقاومت آن در برابر حملات گریز می‌تواند موضوع تحقیقات آینده باشد.

\section{پیشنهادات آتی}
\subsection{پژوهش‌های تکمیلی}
بررسی تأثیر افزایش تعداد لایه‌های ترنسفورمر (\lr{num\_layers} از \lr{1} به \lr{2} یا \lr{3}) در مدل MAGNET با توجه به نتایج بهینه‌سازی و \lr{Optuna} \cite{Optuna2019} که تنها یک لایه را بهینه یافتند، برای بهبود عملکرد در دیتاست‌های بزرگ‌تر و متنوع‌تر پیشنهاد می‌شود. همچنین، آزمایش مدل با داده‌های پویا (مانند الگوهای زمان‌بندی فراخوانی \lr{API}) که در این تحقیق محدود بود، توصیه می‌شود.

\subsection{پیشنهادات اجرایی}
پیاده‌سازی مدل MAGNET در یک سیستم امنیتی واقعی برای اندروید، با ادغام داده‌های پویا (مانند فعالیت شبکه و دسترسی به فایل‌ها) که در دیتاست فعلی به‌صورت محدود استفاده شدند، به‌منظور افزایش دقت تشخیص در محیط‌های عملیاتی پیشنهاد می‌شود. این سیستم می‌تواند به‌عنوان افزونه‌ای برای \lr{Google Play Protect} \cite{GooglePlayProtect} توسعه یابد.

\subsection{تولید داده‌های جدید}
جمع‌آوری دیتاستی با تعادل بیشتر بین کلاس‌ها (افزایش نمونه‌های کلاس \lr{0} به حداقل \lr{1,000} نمونه برای نزدیک شدن به \lr{1,124} نمونه کلاس \lr{1}) و افزودن ویژگی‌های جدید (مانند الگوهای رفتاری کاربران) برای کاهش تأثیر عدم تعادل و ارزیابی جامع‌تر مدل MAGNET توصیه می‌شود.

در نهایت، نتایج این پژوهش زمینه‌ساز ارائه یک چارچوب توسعه‌پذیر برای بهینه‌سازی اعلان‌ها در مدل‌های زبانی بزرگ بوده و می‌تواند بستر مناسبی برای تحقیقات و کاربردهای آینده در حوزه مهندسی اعلان فراهم آورد.
%فصل پنجم
\cleardoublepage
% تنظیمات برای چپ‌چین کردن مراجع
\begin{latin}
\printbibliography[heading=bibintoc, title={مراجع}]
\end{latin}

\newpage
\thispagestyle{empty}


%فهرست مراجع
\cleardoublepage
\setlength{\parindent}{0pt}
\cleardoublepage
\appendix
\addcontentsline{toc}{chapter}{پیوست}
\chapter*{پیوست‌ها}\label{peyvast}

% پیوست A: کدهای پیاده‌سازی مدل MAGNET
\section{پیوست A: کدهای پیاده‌سازی مدل MAGNET}

\subsection{کد معماری مدل MAGNET}
این بخش کد اصلی معماری مدل MAGNET را ارائه می‌دهد که در فصل 3 به‌صورت شبه‌کد توصیف شد. این کد با استفاده از PyTorch پیاده‌سازی شده است.

\begin{LTR}
\begin{lstlisting}[language=Python, caption={کد معماری مدل MAGNET}, label={lst:magnet_architecture}, basicstyle=\scriptsize\ttfamily]
import torch
import torch.nn as nn

class MAGNET(nn.Module):
    def __init__(self, embedding_dim=64, lstm_num_layers=1, dropout=0.2):
        super(MAGNET, self).__init__()        
        self.embedding_dim = embedding_dim
        # Transformation layers for different data modalities
        self.tab_to_emb = nn.Linear(430, embedding_dim)
        self.graph_to_emb = nn.Linear(embedding_dim, embedding_dim)
        self.sequence_processor = nn.LSTM(embedding_dim, embedding_dim, 
                                         num_layers=lstm_num_layers, batch_first=True)
        # Fusion and classification layers
        self.fusion_layer = nn.Linear(3 * embedding_dim, embedding_dim)
        self.classifier = nn.Linear(embedding_dim, 1)
        self.dropout_layer = nn.Dropout(dropout)
    def forward(self, tab_data, graph_data, seq_data):
        # tab_data: (batch_size, 430)
        # graph_data: (batch_size, embedding_dim)
        # seq_data: (batch_size, seq_len, embedding_dim)
        # Transform different data types to embedding vectors
        tab_emb = torch.relu(self.tab_to_emb(tab_data))
        graph_emb = torch.relu(self.graph_to_emb(graph_data))
        
        # Process sequential data with LSTM
        lstm_out, (hn, cn) = self.sequence_processor(seq_data)
        # Use the last output vector from LSTM for each sample in batch
        seq_emb = lstm_out[:, -1, :]
        # Concatenate embedding vectors
        combined_embeddings = torch.cat((tab_emb, graph_emb, seq_emb), dim=-1)
        # Apply fusion layer and activation
        fused_representation = self.fusion_layer(combined_embeddings)
        fused_representation = torch.relu(fused_representation)
        fused_representation = self.dropout_layer(fused_representation)
        # Final classification
        output = torch.sigmoid(self.classifier(fused_representation))
        return output
\end{lstlisting}
\end{LTR}

\subsection{کد بهینه‌سازی با PIRATES}
این بخش قسمت اصلی کد بهینه‌سازی PIRATES را نشان می‌دهد که برای تنظیم ابرپارامترها استفاده شد.

\begin{LTR}
\begin{lstlisting}[language=Python, caption={کد بهینه‌سازی با PIRATES}, label={lst:pirates_optimization}, basicstyle=\scriptsize\ttfamily]
import numpy as np

class Pirates():
    def __init__(self, func, fmax=(), fmin=(), hr=0.2, ms=3, max_r=1, 
                 num_ships=5, dimensions=2, max_iter=10, max_wind=1, c={},
                 top_ships=10, sailing_radius=0.3, plundering_radius=0.1):        
        # Main algorithm parameters
        self.num_ships = num_ships
        self.num_top_ships = top_ships
        self.max_iter = max_iter
        # Objective function parameters
        self.func_obj = func
        self.cost_func = self.func_obj.func
        self.fmin = fmin
        self.fmax = fmax
        self.dimensions = dimensions
        # Weight parameters
        default_c = {
            'leader': 0.5,
            'private_map': 0.5,
            'map': 0.5,
            'top_ships': 0.5
        }
        self.c = {**default_c, **c}
        # Movement parameters
        self.sailing_radius = sailing_radius
        self.plundering_radius = plundering_radius
        # Leader and map variables
        self.leader_index = None
        self.hr = 1 - hr
        self.r = None
        self.max_r = max_r
        self.ms = ms
        self.map = None
        # Problem type
        self.problem = 'min'
        # Chart variables
        self.bsf_position = None
        self.bsf_list = []
        # Initialization
        self.random_init()
        self.iter = 0
    def search(self):
        """
        Run optimization algorithm and return best results
        Returns:
        --------
        tuple
            (best position, best cost, best metrics)
        """
        # Run algorithm
        self.start()
        # Get results
        result = self.cal_costs()
        if result is not None:
            best_cost, best_metrics = result
        else:
            best_cost = self.costs[self.leader_index]
            best_metrics = {'f1': 0.0, 'accuracy': 0.0, 
                          'precision': 0.0, 'recall': 0.0}
        return self.ships[self.leader_index], best_cost, best_metrics
\end{lstlisting}
\end{LTR}

% پیوست B: داده‌های خام و پیش‌پردازش
\section{پیوست \lr{B}: داده‌های خام و پیش‌پردازش}

\subsection{نمونه داده‌های خام DREBIN}
این جدول نمونه‌ای از داده‌های خام دیتاست DREBIN را نشان می‌دهد که برای آموزش مدل استفاده شد.

\begin{table}[ht]
    \centering
    \caption{نمونه‌ای از داده‌های خام دیتاست DREBIN}
    \label{tab:raw_data_sample}
    \begin{tabular}{|l|c|c|c|}
        \hline
        \textbf{شناسه نمونه} & \textbf{تعداد مجوزها} & \textbf{فراخوانی‌های API} & \textbf{برچسب} \\ 
        \hline
        001 & 15 & \lr{["read\_contacts", "send\_sms"]} & 1 \\ 
        \hline
        002 & 8 & \lr{["get\_accounts"]} & 0 \\ 
        \hline
        003 & 12 & \lr{["read\_phone\_state", "write\_external\_storage"]} & 1 \\ 
        \hline
    \end{tabular}
\end{table}

\subsection{توضیحات پیش‌پردازش}
داده‌ها پیش‌پردازش شدند تا برای مدل مناسب شوند:

\begin{itemize}
    \item تنظیم ابعاد ویژگی‌ها از (16, 32) به (16, 430)
    \item نرمال‌سازی با استفاده از استانداردسازی \lr{z-score}
    \item تبدیل داده‌های متنی به بردارهای باینری
\end{itemize}

% پیوست C: جزئیات سخت‌افزاری و نرم‌افزاری
\section{پیوست \lr{C}: جزئیات سخت‌افزاری و نرم‌افزاری}

\subsection{مشخصات سخت‌افزاری}
آزمایش‌ها با استفاده از زیرساخت زیر اجرا شدند:
\begin{itemize}
    \item \lr{GPU}: \lr{NVIDIA RTX 3090} با 24 گیگابایت \lr{VRAM}
    \item \lr{CPU}: \lr{Intel Xeon E5-2690 v4} با 32 هسته
    \item \lr{RAM}: 128 گیگابایت
\end{itemize}

\subsection{مشخصات نرم‌افزاری}
محیط نرم‌افزاری شامل موارد زیر بود:
\begin{itemize}
    \item زبان برنامه‌نویسی: \lr{Python 3.8.5}
    \item کتابخانه‌ها: 
    \begin{itemize}
        \item \lr{PyTorch 1.9.0}
        \item \lr{PyTorch Geometric 1.7.0}
        \item \lr{Optuna 2.10.0}
    \end{itemize}
    \item سیستم‌عامل: \lr{Ubuntu 20.04 LTS}
\end{itemize}

% پیوست D: نتایج اضافی و ماتریس‌های کامل
\section{پیوست \lr{D}: نتایج اضافی و ماتریس‌های کامل}

\subsection{ماتریس درهم‌ریختگی کامل}
این جدول ماتریس درهم‌ریختگی را برای مجموعه تست با 1,451 نمونه نشان می‌دهد.

\begin{table}[ht]
    \centering
    \caption{ماتریس درهم‌ریختگی برای مجموعه تست}
    \label{tab:confusion_matrix}
    \begin{tabular}{|l|c|c|}
        \hline
        \textbf{پیش‌بینی/واقعیت} & \textbf{کلاس 0} & \textbf{کلاس 1} \\ 
        \hline
        \multicolumn{1}{|c|}{\textbf{کلاس 0}} & \lr{304 (TN)} & \lr{23 (FP)} \\ 
        \hline
        \multicolumn{1}{|c|}{\textbf{کلاس 1}} & \lr{17 (FN)} & \lr{1107 (TP)} \\ 
        \hline
    \end{tabular}
\end{table}

\subsection{گزارش طبقه‌بندی برای هر دسته}
این جدول نتایج هر دسته در اعتبارسنجی متقاطع 5-تایی را نشان می‌دهد.

\begin{table}[ht]
    \centering
    \caption{گزارش طبقه‌بندی برای هر دسته در اعتبارسنجی متقاطع}
    \label{tab:fold_results}
    \begin{tabular}{|l|c|c|c|c|}
        \hline
        \textbf{دسته} & \textbf{\lr{F1 Score}} & \textbf{دقت} & \textbf{\lr{AUC}} & \textbf{زیان} \\ 
        \hline
        دسته 1 & \lr{0.9858} & \lr{0.9785} & \lr{0.9950} & \lr{0.0786} \\ 
        \hline
        دسته 2 & \lr{0.9846} & \lr{0.9763} & \lr{0.9955} & \lr{0.0735} \\ 
        \hline
        دسته 3 & \lr{0.9839} & \lr{0.9752} & \lr{0.9945} & \lr{0.0839} \\ 
        \hline
        دسته 4 & \lr{0.9742} & \lr{0.9601} & \lr{0.9861} & \lr{0.1199} \\ 
        \hline
        دسته 5 & \lr{0.9808} & \lr{0.9709} & \lr{0.9946} & \lr{0.0864} \\ 
        \hline
    \end{tabular}
\end{table}

%پیوست
% !TeX root=SBUKThesis-main.tex

\chapter*{توضیحات تکمیلی}
\addcontentsline{toc}{chapter}{توضیحات تکمیلی}
\section*{مقدمه}
\addcontentsline{toc}{section}{مقدمه}
در پیوست در ابتدا، اعلان های دستوری که برای تنظیم کردن مدل زبانی استفاده شده اند ، آورده شده اند که شامل سه بخش نمونه\/گیری روش مولد اعلان ساده ، ارزیابی روش مولد اعلان ساده و اعلان های دستوری سایر روش ها می\/باشد. سپس اعلان های دستوری تولید شده توسط مولد اعلان ساده برای هر مجموعه داده آورده شده اند.

\section*{اعلان های دستوری برای نمونه\/گیری روش مولد اعلان ساده}
\addcontentsline{toc}{section}{اعلان های دستوری برای نمونه\/گیری روش مولد اعلان ساده}

در روش مولد اعلان ساده از سه رویکرد برای نمونه\/گیری اعلان های دستوری جدید استفاده کردیم، رویکرد اول نمونه\/گیری براساس توضیح مسئله مربوط به مجموعه داده بود. اعلان دستوری برای این رویکرد در کادر \ref{p_s1} آورده شده است.

\begin{tcolorbox}[breakable,colframe=mybluecolor!100, colback=mybluecolor!20, title=اعلان دستوری برای نمونه\/گیری براساس توضیح مسئله] \label{p_s1}
	\begin{LTR}
	Given a task description, produce a detailed system prompt to guide a language model in completing the task effectively.
	
	\textbf{Guidelines:}
	\begin{itemize}
		\item \textbf{Understand the Task:} Grasp the core objective, goals, and expected output of the problem as described in the problem description. Identify any implicit requirements or constraints.
		\item \textbf{Minimal Changes:} Since this is a zero-shot approach, use only the information available in the problem description without assuming any additional context or knowledge. Clarify instructions where needed, but avoid adding new elements unless absolutely necessary for comprehension.
		\item \textbf{Reasoning Before Conclusions:} Guide the model to break down the problem step by step before arriving at any conclusions. Structure the prompt to ensure that reasoning is fully explored before the final solution is given.
		\begin{itemize}
			\item Reverse the order if reasoning is provided after conclusions in any sample content. Always start with the reasoning.
		\end{itemize}
		\item \textbf{Clarity and Conciseness:} Make sure that the prompt uses clear, specific language. The instructions should avoid unnecessary complexity or ambiguity.
		\item \textbf{Examples:} Since no examples are provided in a zero-shot context, ensure the problem is fully explained with placeholders for any variables or specifics that may vary.
		\item \textbf{Formatting:} Use markdown for readability. Present steps clearly and in order.
		\item \textbf{Preserve User Content:} Focus entirely on the problem description without bringing in external examples, but structure it logically.
	\end{itemize}
	
	\textbf{Steps:}
	\begin{enumerate}
		\item Parse the problem description.
		\item Identify key variables or constraints.
		\item Guide the model to explore reasoning steps (list if applicable).
		\item Ensure any assumptions or logical pathways are clearly outlined.
	\end{enumerate}
	
	\textbf{Output Format:}  
	The output should be structured as detailed paragraphs or step-by-step instructions, depending on the problem complexity.
	
	\textbf{Notes:}  
	Edge cases: Ensure prompts remain flexible for a variety of inputs, even though no examples are provided.
\end{LTR}
	
\end{tcolorbox}

در رویکرد دوم از مدل زبانی خواسته می\/شد که براساس توضیح مسئله و چند نمونه مثال به همراه جواب از مجموعه داده اقدام به تولید اعلان های دستوری مناسب کند، اعلان دستوری برای این امر در کادر \ref{p_s2} آورده شده است.
\begin{tcolorbox}[breakable,colframe=mybluecolor!100, colback=mybluecolor!20, title=اعلان دستوری برای نمونه\/گیری براساس توضیح مسئله و چند نمونه مثال از مجموعه داده] \label{p_s2}
	\begin{LTR}
		
	
	\textbf{Given a problem description and two example Q\&A pairs, produce a detailed system prompt to guide a language model in completing the task effectively.}
	
	\textbf{Guidelines}
	
	\begin{itemize}
		\item \textbf{Understand the Task:} Use the problem description to understand the overall goal. Supplement this understanding with the two provided examples to clarify the problem’s scope.
		\item \textbf{Minimal Changes:} Incorporate key elements from the examples into the prompt, while maintaining the structure of the problem description. Only adjust where clarity or better instruction flow is necessary.
		\item \textbf{Reasoning Before Conclusions:} Guide the model to analyze the examples and reasoning patterns within the example Q\&As. Ensure that prompts encourage reasoning steps before arriving at final answers.
		\begin{itemize}
			\item Reverse the reasoning order if necessary to ensure it starts with analysis.
		\end{itemize}
		\item \textbf{Examples:} Highlight the key learning points or steps from each of the two examples. Use placeholders [in brackets] to allow flexibility for future examples.
		\item \textbf{Clarity and Conciseness:} Be specific in what needs to be done, and avoid vague or generalized instructions. Ensure the combination of the problem description and examples provides enough guidance.
		\item \textbf{Formatting:} Use markdown for clear structure, with sections for example-based learning, reasoning, and solution paths.
		\item \textbf{Preserve User Content:} Include both the problem description and examples faithfully, without losing important context.
	\end{itemize}
	
	\textbf{Steps}
	
	\begin{enumerate}
		\item Analyze the problem description.
		\item Examine the example Q\&As for patterns in reasoning and solutions.
		\item Synthesize the information to produce a prompt that mirrors the examples while remaining flexible for new problems.
	\end{enumerate}
	
	\textbf{Output Format}  
	
	The output should be a structured set of instructions, with examples embedded for illustration. Use a mix of bullet points and paragraphs for clarity.
	
	\textbf{Examples}
	
	Provide example reasoning paths based on the given Q\&A pairs.
	
	\textbf{Notes}  
	
	Edge cases: Address how prompts should handle examples that deviate from common patterns found in the provided examples.
	\end{LTR}
\end{tcolorbox}

در رویکرد سوم نمونه\/گیری، از مدل زبانی خواسته شد که با الهام از اعلان های دستوری موفق، اقدام به تولید اعلان های دستوری جدید و مشابه با پراپت های موفق کند، اعلان دستوری مربوط به این رویکرد در کادر \ref{p_s3} آورده شده است.

\begin{tcolorbox}[breakable,colframe=mybluecolor!100, colback=mybluecolor!20, title=اعلان دستوری برای نمونه\/گیری براساس اعلان های دستوری موفق] \label{p_s3}
	\begin{LTR}
	Given a successful prompt, produce variations of the prompt while maintaining the original task's goals and structure.
	
	\textbf{Guidelines}
	\begin{itemize}
		\item \textbf{Understand the Task:} Start by identifying the core objective of the successful prompt. Determine why it was effective in completing the task and maintain this focus.
		\item \textbf{Minimal Changes:} Focus on slight variations in wording, structure, or approach to maintain effectiveness. Do not change the task’s essence or main steps unless necessary for clarity.
		\item \textbf{Reasoning Before Conclusions:} Ensure that all variations continue to follow reasoning-first structures. If the original prompt placed conclusions before reasoning, reverse the order for variations.
		\item \textbf{Clarity and Conciseness:} Variations should remain clear and to the point, without introducing ambiguity or confusion.
		\item \textbf{Examples:} Highlight variations with slight changes in phrasing, while retaining the core elements of the original prompt.
		\item \textbf{Formatting:} Keep formatting consistent across variations. Use bullet points or markdown headings to segment the variations clearly.
		\item \textbf{Preserve User Content:} Maintain the overall flow and details of the successful prompts, making variations in small increments.
	\end{itemize}
	
	\textbf{Steps}
	\begin{enumerate}
		\item Analyze the successful prompt to identify key elements that make it work.
		\item Create multiple variations by adjusting wording, step order, or clarity points.
		\item Ensure each variation follows the same reasoning and solution path, with slight differences in phrasing or structure.
	\end{enumerate}
	
	\textbf{Output Format}
	
	Output only one variation in the given prompt form, with minor changes to structure, wording, or instruction flow.
	
	\textbf{Notes}
	
	Edge cases: Test how different variations might perform across a range of inputs. Identify possible weaknesses in certain phrasing and adjust accordingly.
	
	\end{LTR}  
\end{tcolorbox}









\section*{ارزیابی}
\addcontentsline{toc}{section}{ارزیابی}

همانطور که در فصل 3 بخش ارزیابی توضیح داده شد، هر اعلان دستوری تولید شده روی مجموعه داده ارزیابی می\/شود و برای سوال موجود در آن مجموعه داده، پاسخی تولید می\/کند. سپس نیاز است که این پاسخ با پاسخ واقعی مقایسه شود. این مقایسه و ارزیابی توسط مدل زبانی از طریق اعلان دستوری \ref{p_v} به عنوان دستورالعمل انجام می\/شود.

\begin{tcolorbox}[breakable,colframe=mybluecolor!100, colback=mybluecolor!20, title=اعلان دستوری برای ارزیابی جواب تولید شده] \label{p_v}
	\begin{LTR}
	By the provided information,  
	Task:  
	\begin{enumerate}
		\item Compare the Predicted Answer with the Actual Answer.
		\item Determine if the Predicted Answer matches the Actual Answer exactly.
		\item Output "CORRECT" if the answers match, otherwise output "INCORRECT".
	\end{enumerate}
	
	Response Format:  
	
	\{CORRECT or INCORRECT\} 
	\end{LTR} 
\end{tcolorbox}

در ادامه یک سوال و جواب از مجموعه داده GSM8K به عنوان مثال و همچنین جواب تولید شده توسط بهترین اعلان دستوری برای بررسی نحوه پاسخدهی مدل زبانی آورده شده است.

\begin{tcolorbox}[breakable,colframe=mybluecolor!100, colback=mybluecolor!20, title=یک نمونه سوال و جواب از مجموعه داده GSM8K به همراه جواب تولید شده توسط روش مولد اعلان ساده ] \label{p_ex}
	\begin{LTR}
	\textbf{Question:}  
	\lr{Henry made two stops during his 60-mile bike trip. He first stopped after 20 miles. His second stop was 15 miles before the end of the trip. How many miles did he travel between his first and second stops?}
	\\ \\
	\textbf{Answer:}  
	\lr{Henry traveled 20 miles + 15 miles = \( 20 + 15 = 35 \) miles, not counting the distance between stops.}
	
	\lr{Henry traveled 60 miles - 35 miles = \( 60 - 35 = 25 \) miles between his first and second stop.}
	
	\lr{Final Answer : 25}
	\\ \\
	\textbf{Predicted-Answer:}
	
	\text{Analyze the Problem Description:}
	
	\lr{The problem involves finding the distance traveled between two stops during a bike trip. The initial stop was after 20 miles, and the second stop was 15 miles before the end of the trip.}
	
	\text{Example-Based Learning:}
	\begin{itemize}
		\item Identify the key components of the problem: initial stop distance, final stop distance, and total trip distance.
		\item Understand the order of operations: find the distance between the two stops first, then subtract this value from the total trip distance to find the distance traveled between the stops.
	\end{itemize}
	
	\text{Reasoning:}
	\begin{itemize}
		\item Find the distance between the two stops: $60 - (20 + 15) = 60 - 35 = 25$ miles.
		\item \lr{The distance traveled between the first and second stops is 25 miles.}
	\end{itemize}
	
	\text{Solution Paths:}
	\begin{itemize}
		\item Find the distance between the two stops by subtracting the sum of the initial and final stop distances from the total trip distance.
		\item The distance traveled between the first and second stops is the calculated distance between the two stops.
	\end{itemize}
	
	\text{\lr{Example Q\&A Pair 1 :} } \
	\lr{Ken created a care package to send to his brother, who was away at boarding school. Ken placed a box on a scale, and then he poured into the box enough jelly beans to bring the weight to 2 pounds. Then, he added enough brownies to cause the weight to triple. Next, he added another 2 pounds of jelly beans. And finally, he added enough gummy worms to double the weight once again. What was the final weight of the box of goodies, in pounds?}
	\\ \\
	\text{\lr{Example Q\&A Pair 2 : }} \
	\lr{James creates a media empire. He creates a movie for \$2000. Each DVD costs \$6 to make. He sells it for 2.5 times that much. He sells 500 movies a day for 5 days a week. How much profit does he make in 20 weeks?}
	\\ \\
	\text{Prompt and Example:}
	\begin{quote}
		\text{Prompt:}  
		Given a problem that involves finding the distance traveled between two stops during a trip, solve it by finding the distance between the two stops first, then subtracting this value from the total trip distance to find the distance traveled between the stops.
		
		\text{Example:}  
		\lr{Henry made two stops during his 60-mile bike trip. He first stopped after 20 miles. His second stop was 15 miles before the end of the trip. How many miles did he travel between his first and second stops?}
		
		\text{Solution:}  
		\begin{itemize}
			\item Find the distance between the two stops: $60 - (20 + 15) = 60 - 35 = 25$ miles.
			\item \lr{The distance traveled between the first and second stops is 25 miles.}
		\end{itemize}
	\end{quote}
	\end{LTR}
\end{tcolorbox}


\section*{سایر روش ها}
\addcontentsline{toc}{section}{سایر روش ها}
در جدول \ref{tab_prompts_arithmetic}، اعلان‌های دستوری به‌کاررفته برای هدایت مدل زبانی Mistral به‌منظور ایفای نقش به‌عنوان روش‌های زنجیره تفکر ، برنامه\/ریزی و حل، برنامه\/ریزی و حل پیشرفته، مهندس اعلان خودکار و بهینه سازی با اعلان ارائه شده است. این اعلان‌ها برگرفته از مقاله مولد اعلان \cite{PromptBreeder} می‌باشند.
\begin{table}[h!]
	\centering
	\begin{LTR}
	\begin{tabular}{lp{13cm}}
		\hline
		\textbf{Method} & \textbf{Instruction Prompt} \\ \hline
		CoT   & \lr {“Let’s think step by step.”} \\ 
		PS    & \lr {“Let’s first understand the problem and devise a plan to solve the problem. Then, let’s carry out the plan and solve the problem step by step.”} \\ 
		PS+   & \lr {“Let’s first understand the problem, extract relevant variables and their corresponding numerals, and make a plan. Then, let’s carry out the plan, calculate intermediate variables (pay attention to correct numerical calculation and commonsense), solve the problem step by step, and show the answer.”} \\ 
		APE   & \lr {“Let’s work this out in a step by step way to be sure we have the right answer.”} \\ 
		OPRO  & \lr{“Take a deep breath and work on this problem step-by-step.”} \\ \hline
	\end{tabular}
	\end{LTR}
	\caption{اعلان های دستوری برای سایر روش ها جهت مقایسه نتایج }
	\label{tab_prompts_arithmetic}
\end{table}









%% !TeX root=SBUKThesis-main.tex

\chapter*{پیوست دوم}
%%%%%%%%%%%%%%%%%%%%%%%%%%%
\newgeometry{left=1cm,right=3cm,top=3cm,bottom=3cm}
\renewcommand{\thefootnote}{\fnsymbol{footnote}}
\thispagestyle{empty}
\begin{landscape}
\begin{center}
\includegraphics[width=2cm]{logo}
\vskip -3mm
{\bfseries \fontsize{11}{12}\selectfont
	بسمه تعالی} \\
\vskip -2mm
{\bfseries \fontsize{11}{12}\selectfont
فرم تایید اطلاعات تولیدات علمی
$^\star$
مستخرج از پایان‌نامه دانشجويان کارشناسی ارشد
}\\
\end{center}
\begin{center}
\fontsize{12}{13}\selectfont

 \textbf{نام و نام خانوادگي دانشجو: }
علیرضا ایرانمنش
\textbf{شماره دانشجويي:}
401155015
\textbf{نام دانشکده:}
فنی مهندسی\\
\textbf{رشته و گرايش:}
مهندسی کامپیوتر-هوش مصنوعی
\textbf{نام استاد راهنما:}
دکتر حمید میروزیری\\
\end{center}
\begin{center}
	{\small{}}
	\textbf{عنوان پایان\/نامه:}
	تشخیص قدرتمند بدافزار‌های اندروید با استفاده از شبکه‌های عصبی ترنسفورمر

	\scalebox{.91}{
		\begin{tabular}{|c|c|c|c|}
			\hline
			\multicolumn{4}{|>{\centering}m{22cm}|}{\scriptsize{{
						مشخصات تولیدات علمی}}}
			\\ \hline
			\multicolumn{1}{|c|}{\scriptsize{{ردیف}}}&
			\multicolumn{1}{>{\centering}m{6cm}|}{\scriptsize{{عنوان تولیدات علمی}}}&
			\multicolumn{1}{>{\centering}m{6cm}|}{\scriptsize{{ مرجع تایید کننده/نام کنفرانس/ نام مجله}}}&
			\multicolumn{1}{>{\centering}m{4cm}|}{\scriptsize{{توضیحات}}}
		\\ \hline
		\end{tabular}}
	\end{center}
\fontsize{9}{10}\selectfont
\begin{center}
تولیدات علمی فوق با نمره (عدد) ۵.۱ (حروف) یک و پنج دهم (حداکثر 2 نمره) در ارزیابی پایان نامه مورد تایید قرار گرفت و نمره نهایی پایان‌نامه فوق با احتساب نمره تولیدات علمی   (عدد)            (حروف)  می‌باشد.

\end{center}
\begin{minipage}{.4\textwidth}
\begin{center}\small
نام و نام خانوادگی استاد / استادان راهنما:\\
تاریخ:\\
امضاء:
\end{center}
\end{minipage}
\hspace{1cm}
	\begin{minipage}{.4\textwidth}
		\begin{center}\small
		نام و نام خانوادگی نماینده هیأت داوران:\\
			تاریخ:\\
			امضاء:
		\end{center}
	\end{minipage}
\begin{minipage}{.4\textwidth}
\begin{center}\small
نام و نام خانوادگی مدیر گروه/ رییس بخش:\\
تاریخ:\\
امضاء:
\end{center}
\end{minipage}
\begin{center}
\bfseries \fontsize{8}{9}\selectfont
$^\star$
تولیدات علمی شامل مقاله، اختراع، ساخت دستگاه، اکتشاف، ثبت اثر بدیع هنری می‌باشد که از پایان‌نامه استخراج شده باشد.

\end{center}
\end{landscape}
\newgeometry{top=30mm, bottom=30mm, left=30mm, right=30mm}%صفحه استخراج مقاله
\cleardoublepage
%%%%%%%%%%%%%%%%%%%%%%
\titleformat{\chapter}[display]
{}{}{1ex}{\filcenter}[]
%% !TeX root=SBUKThesis-main.tex
\chapter*{\vspace{-2.5cm}\centering\bfseries\fontsize{15}{16}\selectfont واژه‌نامه انگلیسی به فارسی
\vspace{0.75cm}\hrule height 1.5pt \vspace{-1.5cm}}


\persiangloss{الگوریتم‌های حافظ کران}{bound-conserving algorithms}
\persiangloss{کران‌های بدبینانه}{pessimistic}
\persiangloss{زمان‌بر}{time-consuming}%واژگان انگلیسی به فارسی
\cleardoublepage
%% !TeX root=SBUKThesis-main.tex
\chapter*{\vspace{-2.5cm}\centering\bfseries\fontsize{15}{16}\selectfont واژه‌نامه فارسی به انگلیسی
\vspace{0.75cm}\hrule height 1.5pt \vspace{-1.5cm}}


\englishgloss{global steepest descent algorithm}{الگوریتم تندترین کاهش سراسری}
\englishgloss{global minimal residual descent algorithm}{الگوریتم کاهش باقیمانده کمین سراسری}
\englishgloss{convergence analysis}{آنالیز همگرایی}


%واژگان فارسی به انگلیسی
\cleardoublepage
%%%%%%%%%%%%%%%%%%%%%%%
\pagestyle{empty}
\newgeometry{left=4cm,right=3cm,top=3cm,bottom=3cm}
\titleformat{\chapter}[display]
{}{}{1ex}{\bfseries\raggedright\fontsize{15}{16}\selectfont}[]
\begin{latin}
% !TeX root=SBUKThesis-main.tex
\chapter*{\vspace{-3cm}\fontsize{14}{15}\selectfont Abstract}
\thispagestyle{empty}
\vspace{-1.5cm}\setlength{\parindent}{20pt}\fontsize{12}{13}\selectfont
The performance of large language models (LLMs) relies heavily on prompt engineering. Manual methods such as programming and problem solving have improved the reasoning process to some extent, but they often fall short in ensuring diversity in the generated prompts and limit overall effectiveness. 
On the other hand, prompt generation methods have overcome this limitation by introducing a self-reflective improvement mechanism. By leveraging a genetic algorithm with a binary tournament selection strategy, they gradually evolve instructional prompts. This algorithm enables the prompt generator to iteratively explore the prompt space while optimizing for both diversity and performance simultaneously.
Despite the significant advancements made by prompt generation methods in creating optimal prompts, a new challenge has emerged — the increased computational burden and complexity of these approaches, which makes their practical application difficult in many real-world scenarios.
To address these issues, we propose SimplePromptBreeder that utilizes a local search strategy to optimize prompts. This approach employs a probabilistic model called Determinantal Point Processes (DPPs) to select high-quality and diverse prompts, directly balancing performance and diversity without relying on complex self-referential mechanisms.

We evaluated the optimal prompt generation method on eight benchmark datasets: 
\lr{MultiArith}, 
\lr{SingleEq}, 
\lr{AddSub}, 
\lr{SVAMP}, 
\lr{SQA}, 
\lr{CSQA}, 
\lr{AQuA-RAT}, and 
\lr{GSM8K}, 
achieving a relative improvement of \lr{23.4\%} over the problem-and-solve approach and a relative improvement of \lr{54.7\%} over the PromptBreeder method.

\par\vspace{.5cm}\setlength{\parindent}{0pt}
\textbf{Keywords:} Large Language Models, Prompt Engineering, Instructional Prompts, Determinantal Point Processes, Diversity, Quality




 \par\vspace{.5cm}\setlength{\parindent}{0pt}{\bfseries \fontsize{12}{13}\selectfont Keywords: Large Language Models, Prompt Engineering, Instructional Prompts, Determinantal Point Processes, Diversity, Quality }

%چکیده انگلیسی
% !TeX root=SBUKThesis-main.tex


\setlength{\parindent}{0pt}
\begin{center}
%\thispagestyle{empty}
\includegraphics[height=2cm]{logo} \\
{\fontsize{14}{15}\selectfont \textbf{Shahid Bahonar University of Kerman}} \\
{\fontsize{14}{15}\selectfont\textbf{Faculty of  Engineering }} \\
{\fontsize{14}{15}\selectfont\textbf{Department of Computer Engineering}} \\
\vskip 1cm
\hrule height 2pt
%\vskip .3mm
%\hrule height 1.15pt
\par
\begin{center}
\fontsize{16}{17}\bfseries\selectfont
Robust Android Malware Detection using Transformer Neural Networks
\end{center}
\par
\hrule height 2pt
%\vskip .3mm
%\hrule height 1.15pt
\par
\vskip 2cm
{\fontsize{16}{17}\selectfont \bfseries
	 Prepared by:}
\\
{\fontsize{14}{15}\selectfont \bfseries
	Alireza Iranmanesh}
	 \par \vskip 1cm
{\fontsize{16}{17}\selectfont \bfseries
	 Supervisor:}
	 \\
{\fontsize{14}{15}\selectfont \bfseries
	 Dr. Hamid Mirvaziri}
% 	\par \vskip 1cm
% {\fontsize{16}{17}\selectfont \bfseries
% 	Advisor:}
% \\
% {\fontsize{14}{15}\selectfont \bfseries
% 	Dr. Sahar Vahdati}
% 	 \par \vskip 1cm


% 	 \par

 \vskip 1cm
{\fontsize{14}{15}\selectfont \bfseries
A Thesis Submitted as a Partial Fulfillment of the Requirements for the Degree of Master of Science in Computer Engineering (M. Sc.)}
\vskip15mm
{\fontsize{14}{15}\selectfont \bfseries
April 2025}
\end{center} 
%صفحه عنوان انگلیسی
%در صورتی که می خواهید دو صفحه خالی در انتهای پایان نامه باشد، دو خط زیر فعال شود
%\cleardoublepage
%%% !TeX root=GUATThesis-main.tex
%% !TEX TS-program = XeLaTeX
%\newpage\null\thispagestyle{empty}\newpage
\end{latin}
\end{document}
%% End of file `SBUKThesis-main.tex'.
%% End of file `SBUKThesis-main.tex'.