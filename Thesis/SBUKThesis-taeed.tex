{\nastaliq\bfseries \fontsize{10}{11}\selectfont
\begin{center}
به نام خدا
	\vskip 2mm
منشور اخلاق پژوهش\\
	\vskip 2mm
با استعانت از خدای سبحان و اعتقاد راسخ به این که عالم محضر خداست و او همواره ناظر بر اعمال ماست و به منظور انجام شایسته‌ی پژوهش‌های اصیل، تولید دانش جدید و بهسازی زندگانی بشر، ما دانشجویان و اعضای هیأت علمی دانشگاه‌ها و پژوهشگاه‌های کشور:
\end{center}
\begin{itemize}
\item[$\square$]
تمام تلاش خود را برای کشف حقیقت و فقط حقیقت به کار خواهیم بست و از هر گونه جعل و تحریف در فعالیت های علمی پرهیز می‌کنیم.
\item[$\square$]
حقوق پژوهشگران، پژوهیدگان (انسان، حیوان، نبات و اشیاء)، سازمان‌ها و سایر صاحبان حق را به رسمیت می شناسیم و در حفظ آن می‌کوشیم.
\item[$\square$]
به مالکیت مادي و معنوي آثار پژوهشی ارج می نهیم، برای انجام پژوهشی اصیل اهتمام ورزیده و از سرقت علمی و ارجاع نامناسب اجتناب می کنیم.
\item[$\square$]
ضمن پایبندی به انصاف و اجتباب از هر گونه تبعیض و تعصب در کلیه فعالیت‌های پژوهشی، رهیافتی نقادانه اتخاذ خواهیم کرد.
\item[$\square$]
ضمن امانت‌داری، از منابع و امکانات اقتصادی، انسانی و فنی موجود، استفاده بهره‌ورانه خواهیم کرد.
\item[$\square$]
از انتشار غیراخلاقی نتایج پژوهش، نظیر انتشار موازی، همپوشان و چندگانه (تکه‌ای) پرهیز می‌کنیم.
\item[$\square$]
اصل محرمانه بودن و رازداری را محور تمام فعالیت‌های پژوهشی خود قرار می‌دهیم.
\item[$\square$]
در همه فعالیت‌های پژوهشی به منافع ملی توجه کرده و برای تحقق آن می‌کوشیم.
\item[$\square$]
خویش را ملزم به رعایت کلیه هنجارهای علمی رشته خود، قوانین و مقررات، سیاست‌های حرفه‌ای، سازمانی، دولتی و راهبردهای ملی در همه مراحل پژوهش می‌دانیم.
\item[$\square$]
رعایت اصول اخلاق در پژوهش را اقدامی فرهنگی می‌دانیم و به منظور بالندگی این فرهنگ، به ترویج و اشاعه آن در جامعه اهتمام می‌ورزیم.
\end{itemize}
}