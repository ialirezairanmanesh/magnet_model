% !TeX root=SBUKThesis-main.tex
\chapter*{\vspace{-2.38cm}\fontsize{15}{16}\selectfont چکیده:}
\vspace{-1.5cm}\setlength{\parindent}{20pt}
با افزایش روزافزون تهدیدات سایبری، تشخیص بدافزارهای اندرویدی به یکی از چالش‌های اساسی در حوزه امنیت اطلاعات تبدیل شده است. روش‌های سنتی، به‌ویژه آن‌هایی که صرفاً بر تحلیل ویژگی‌های تک‌وجهی تکیه دارند، اغلب قادر به پردازش داده‌های پیچیده چندوجهی نبوده و در مواجهه با تهدیدات جدید، از تعمیم‌پذیری مناسبی برخوردار نیستند. این محدودیت‌ها، ضرورت توسعه رویکردهای نوین و کارآمد را آشکار می‌سازد. در این پژوهش، مدلی چندوجهی با عنوان «تبدیل‌گر چندوجهی مبتنی بر جاسازی گراف دینامیک با توجه پویا (MAGNET)» توسعه داده شد که با ترکیب داده‌های جدولی، گراف و ترتیبی، از جمله توالی فراخوانی‌های API، به شناسایی بدافزارهای اندرویدی پرداخت. هدف اصلی، ارتقای دقت و پایداری تشخیص با بهره‌گیری از معماری پیشرفته مبتنی بر یادگیری عمیق و ترنسفورمر بود. در این راستا، بهینه‌سازی هایپرپارامترها با استفاده از الگوریتم‌های پیشرفته‌ای مانند PIRATES و Optuna انجام گرفت و مدل با مجموعه داده‌ای شامل ۴۶۴۱ نمونه آموزشی و ۱۴۵۱ نمونه آزمایشی، همراه با اعتبارسنجی متقاطع پنج‌تایی، آموزش داده شد. ویژگی‌های مورد استفاده شامل ویژگی‌های ایستا نظیر مجوزها، فراخوانی‌های API، مقاصد و نام مؤلفه‌ها و همچنین ویژگی‌های پویا مانند فعالیت شبکه و دسترسی به فایل‌ها بود. داده‌ها به صورت بردارهای عددی باینری یا نرمال‌سازی‌شده آماده‌سازی شدند و پس از پیش‌پردازش، ابعاد ویژگی‌ها به ۴۳۰ ویژگی تنظیم گردید. ابزارهای مورد استفاده شامل کتابخانه‌های یادگیری عمیق مانند PyTorch، تکنیک‌های استانداردسازی و نرمال‌سازی داده‌ها و ساختارهای داده‌ای گرافی بود. نتایج نشان داد که مدل پیشنهادی عملکردی برجسته با دقت بالا، پایداری قابل توجه و قابلیت تعمیم‌پذیری مطلوب ارائه کرد و نسبت به روش‌های پیشین بهبود قابل ملاحظه‌ای داشت. این یافته‌ها، پتانسیل کاربرد مدل در سیستم‌های امنیتی واقعی را برجسته می‌سازد. پیشنهاد می‌شود در تحقیقات آینده، افزایش حجم داده‌ها، ادغام روش‌های خودنظارتی پیشرفته، آزمایش مدل در محیط‌های متنوع‌تر و بهینه‌سازی زمان اجرا مورد توجه قرار گیرد تا کارایی مدل در سناریوهای پیچیده‌تر و واقعی‌تر ارتقا یابد. همچنین، بررسی تأثیر ترکیب داده‌های جدید و توسعه الگوریتم‌های مقاوم در برابر حملات مخرب می‌تواند مسیرهای نوینی برای پژوهش‌های بعدی فراهم آورد.

\par\vspace{.5cm}\setlength{\parindent}{0pt}
{\bf
واژگان کلیدی: تشخیص بدافزار، ترنسفورمر، یادگیری عمیق، داده‌های چندوجهی، امنیت اندروید.

} 

 
