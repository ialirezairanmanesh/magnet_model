% !TeX root=SBUKThesis-main.tex
\chapter*{\vspace{-2.38cm}\fontsize{15}{16}\selectfont چکیده:}
\vspace{-1.5cm}\setlength{\parindent}{20pt}
تشخیص بدافزارهای اندرویدی با افزایش روزافزون تهدیدات سایبری، یکی از چالش‌های اصلی در حوزه امنیت اطلاعات به شمار می‌رود. روش‌های سنتی، به‌ویژه آن‌هایی که صرفاً بر تحلیل ویژگی‌های تک‌وجهی تکیه دارند، اغلب با محدودیت‌هایی نظیر ناتوانی در پردازش داده‌های پیچیده چندوجهی و تعمیم‌پذیری ضعیف در برابر تهدیدات جدید مواجه‌اند. این کاستی‌ها، نیاز به توسعه رویکردهای نوین و کارآمد را بیش از پیش آشکار می‌سازد. این پژوهش مدلی چندوجهی با عنوان تبدیل‌گر چندوجهی مبتنی بر جاسازی گراف دینامیک با توجه پویا (MAGNET) توسعه داد که از ترکیب داده‌های جدولی، گراف و ترتیبی، نظیر توالی فراخوانی‌های ،API برای شناسایی بدافزارهای اندرویدی بهره برد. هدف اصلی، بهبود دقت و پایداری تشخیص با استفاده از معماری پیشرفته مبتنی بر یادگیری عمیق و ترنسفورمر بود. روش تحقیق شامل بهینه‌سازی هایپرپارامترها با الگوریتم‌های پیشرفته مانند PIRATES و Optuna، آموزش مدل با مجموعه داده‌ای شامل 4641 نمونه آموزشی و 1451 نمونه آزمایشی، و اعتبارسنجی متقاطع 5-تایی شد. ویژگی‌های مورد استفاده شامل ویژگی‌های ایستا مانند مجوزها، فراخوانی‌های API، مقاصد، و نام‌های مؤلفه و ویژگی‌های پویا مانند فعالیت شبکه و دسترسی به فایل‌ها بود. داده‌ها به صورت بردارهای عددی باینری یا نرمال‌سازی شده بودند. ابعاد ویژگی‌ها پس از پیش‌پردازش به 430 ویژگی تنظیم شد. ابزارهای مورد استفاده شامل کتابخانه‌های یادگیری عمیق مانند PyTorch، تکنیک‌های پیش‌پردازش داده‌ها نظیر استانداردسازی و نرمال‌سازی، و ساختارهای داده‌ای گرافی بودند. مواد اولیه شامل داده‌های واقعی از رفتار اپلیکیشن‌های اندرویدی، شامل ویژگی‌های ایستا و پویا، بود که با دقت آماده‌سازی شدند. نتایج نشان داد که مدل پیشنهادی عملکردی برجسته با دقت بالا، پایداری قابل‌توجه و قابلیت تعمیم‌پذیری خوب ارائه کرد و نسبت به روش‌های پیشین بهبود قابل‌ملاحظه‌ای داشت. این دستاوردها پتانسیل کاربرد مدل در سیستم‌های امنیتی واقعی را برجسته ساخت. پیشنهاد می‌شود تحقیقات آینده بر افزایش حجم داده‌ها، ادغام روش‌های خودنظارتی پیشرفته، آزمایش مدل در محیط‌های متنوع‌تر و بهینه‌سازی زمان اجرای آن متمرکز شوند تا کارایی مدل در سناریوهای پیچیده‌تر و واقعی‌تر ارتقا یابد. همچنین، بررسی تأثیر ترکیب داده‌های جدیدتر و توسعه الگوریتم‌های مقاوم در برابر حملات مخرب می‌تواند مسیرهای نوینی برای تحقیقات بعدی گشوده کند. این پژوهش، گامی مؤثر در راستای ارتقای سیستم‌های تشخیص خودکار بدافزارها برداشت و پایه‌ای محکم برای توسعه راه‌حل‌های امنیتی پیشرفته‌تر فراهم آورد.

\par\vspace{.5cm}\setlength{\parindent}{0pt}
{\bf
واژگان کلیدی: تشخیص بدافزار، یادگیری عمیق، داده‌های چندوجهی، امنیت اندروید، ترنسفورمر.

} 

 
