% !TeX root=SBUKThesis-main.tex
\chapter*{\vspace{-2.38cm}\fontsize{15}{16}\selectfont چکیده:}
\vspace{-1.5cm}\setlength{\parindent}{20pt}
با افزایش روزافزون تهدیدات سایبری، تشخیص بدافزارهای اندرویدی به یکی از چالش‌های اساسی در حوزه امنیت اطلاعات تبدیل می‌شود. روش‌های سنتی، به‌ویژه آن‌هایی که صرفاً بر تحلیل ویژگی‌های تک‌وجهی تکیه دارند، اغلب در پردازش داده‌های پیچیده چندوجهی ناتوان هستند و در مواجهه با تهدیدات جدید، از تعمیم‌پذیری مناسبی برخوردار نیستند. این محدودیت‌ها، ضرورت توسعه رویکردهای نوین و کارآمد را آشکار می‌سازند. در این پژوهش، مدلی چندوجهی با عنوان «تبدیل‌گر چندوجهی مبتنی بر جاسازی گراف دینامیک با توجه پویا (MAGNET)» توسعه می‌یابد که با ترکیب داده‌های جدولی، گراف و ترتیبی، از جمله توالی فراخوانی‌های API، به شناسایی بدافزارهای اندرویدی می‌پردازد. هدف اصلی، ارتقای دقت و پایداری تشخیص با بهره‌گیری از معماری پیشرفته مبتنی بر یادگیری عمیق و ترنسفورمر است. در این راستا، بهینه‌سازی هایپرپارامترها با استفاده از الگوریتم‌های پیشرفته‌ای مانند PIRATES و Optuna انجام می‌گیرد و مدل با مجموعه داده‌ای شامل ۴۶۴۱ نمونه آموزشی و ۱۴۵۱ نمونه آزمایشی، همراه با اعتبارسنجی متقاطع پنج‌تایی، آموزش می‌بیند. ویژگی‌های مورد استفاده شامل ویژگی‌های ایستا نظیر مجوزها، فراخوانی‌های API، مقاصد و نام مؤلفه‌ها و همچنین ویژگی‌های پویا مانند فعالیت شبکه و دسترسی به فایل‌ها هستند. داده‌ها به صورت بردارهای عددی باینری یا نرمال‌سازی‌شده آماده‌سازی می‌شوند و پس از پیش‌پردازش، ابعاد ویژگی‌ها به ۴۳۰ ویژگی تنظیم می‌گردند. ابزارهای مورد استفاده شامل کتابخانه‌های یادگیری عمیق مانند PyTorch، تکنیک‌های استانداردسازی و نرمال‌سازی داده‌ها و ساختارهای داده‌ای گرافی هستند. نتایج نشان می‌دهند که مدل پیشنهادی عملکردی برجسته با دقت بالا، پایداری قابل توجه و قابلیت تعمیم‌پذیری مطلوب ارائه می‌دهد و نسبت به روش‌های پیشین بهبود قابل ملاحظه‌ای دارد. این یافته‌ها، پتانسیل کاربرد مدل در سیستم‌های امنیتی واقعی را برجسته می‌سازند.

\par\vspace{.5cm}\setlength{\parindent}{0pt}
{\bf
واژگان کلیدی: تشخیص بدافزار، ترنسفورمر، یادگیری عمیق، داده‌های چندوجهی، امنیت اندروید.
}
