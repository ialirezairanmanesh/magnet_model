% !TeX root=SBUKThesis-main.tex
\clearpage
\thispagestyle{empty}
\chapter{نتیجه گیری و پیشنهادات آتی}\label{chap5}
\section{نتیجه‌گیری}
در این پژوهش، با هدف ارتقای فرآیند تولید و بهینه‌سازی اعلان‌ها برای مدل‌های زبانی بزرگ، روشی نوین و کم‌هزینه با عنوان مولد اعلان شاده \LTRfootnote{Simple Prompt Breeder} طراحی و پیاده‌سازی گردید. دلیل اصلی انجام این تحقیق، چالش‌های محاسباتی موجود در روش‌های پیشین نظیر مولد اعلان \cite{PromptBreeder} بود که به سبب پیچیدگی‌های ذاتی خود، در جستجوی فضای وسیع اعلان‌ها با محدودیت‌های جدی مواجه بودند. به همین منظور، تلاش شد تا با ارائه رویکردی مبتنی بر جستجوی محلی، کارایی و بهره‌وری فرآیند بهینه‌سازی اعلان‌ها به شکل محسوسی افزایش یابد.

نتایج حاصل از آزمایش‌ها نشان می‌دهد که روش مولد اعلان شاده، ضمن حفظ دقت مطلوب، توانسته است به میزان قابل توجهی هزینه‌های محاسباتی را در مقایسه با الگوریتم‌های تکاملی مشابه کاهش دهد. این دستاورد، مؤید اثربخشی روش پیشنهادی در تسهیل و تسریع فرآیند مهندسی اعلان در زمینه‌های پژوهشی و کاربردی می‌باشد. به‌طور کلی، روش مولد اعلان شاده به عنوان ابزاری کارآمد، قادر است به نیازهای پروژه‌ها و سامانه‌هایی که با محدودیت منابع محاسباتی روبرو هستند، پاسخ موثری ارائه دهد.

\section{پیشنهادات آتی}
با توجه به نتایج مثبت حاصل از این پژوهش و همچنین چالش‌های موجود در زمینه بهینه‌سازی اعلان‌ها برای مدل‌های زبانی، پیشنهاد می‌شود مسیرهای زیر در مطالعات آتی مورد توجه قرار گیرد:

\begin{enumerate}
	\item \textbf{گسترش مطالعات بر روی داده‌های گسترده‌تر و متنوع‌تر:} بررسی عملکرد الگوریتم مولد اعلان شاده بر روی مجموعه داده‌هایی با ابعاد و تنوع بیشتر، می‌تواند میزان پایداری و تعمیم‌پذیری این الگوریتم را در محیط‌های عملیاتی واقعی مورد ارزیابی قرار دهد.
	
	\item \textbf{افزودن شاخص‌های تکمیلی برای سنجش کیفیت و تنوع:} در ادامه پژوهش، می‌توان با افزودن معیارهای جدید جهت سنجش تنوع و کیفیت اعلان‌ها، الگوریتم را در برابر خطر همگرایی زودهنگام و تولید اعلان‌های یکنواخت مقاوم‌تر ساخت و توازن مطلوبی میان اکتشاف و بهره‌برداری برقرار نمود.
	
	\item \textbf{سازگاری با معماری‌های نوین مدل‌های زبانی:} با توجه به ظهور معماری‌های نوین در حوزه مدل‌های زبانی نظیر معماری‌های مبتنی بر \lr{Sparse Attention}
	 و یا مدل‌های چندوجهی
	 \LTRfootnote{multimodal}
	  ، توسعه نسخه‌های بهینه‌شده از مولد اعلان شاده متناسب با این معماری‌ها می‌تواند گامی مؤثر در راستای افزایش انعطاف‌پذیری و قابلیت‌های این الگوریتم باشد.
	
	\item \textbf{بررسی و پیاده‌سازی سایر روش‌های جستجوی محلی:} از آنجا که مولد اعلان شاده ماهیتاً مشابه الگوریتم \lr{Hill-Climbing} عمل می‌نماید، پیشنهاد می‌شود در مطالعات آتی از رویکردهای جایگزین نظیر \lr{Simulated Annealing} یا \lr{Tabu Search} نیز استفاده گردد تا ضمن افزایش تنوع در فرآیند جستجو، به بهبود کارایی الگوریتم در مسائل پیچیده‌تر منجر شود.
	
	\item \textbf{ادغام با رویکردهای مبتنی بر یادگیری تقویتی:} به منظور بهبود تدریجی و هدفمند سیاست‌های جستجو، می‌توان مولد اعلان شاده را با الگوریتم‌های یادگیری تقویتی ترکیب نمود و از این طریق، فرآیند بهینه‌سازی اعلان‌ها را با دریافت بازخوردهای پویا از محیط و مدل هدف، هوشمندانه‌تر و اثربخش‌تر ساخت.
\end{enumerate}

در نهایت، نتایج این پژوهش زمینه‌ساز ارائه یک چارچوب توسعه‌پذیر برای بهینه‌سازی اعلان‌ها در مدل‌های زبانی بزرگ بوده و می‌تواند بستر مناسبی برای تحقیقات و کاربردهای آینده در حوزه مهندسی اعلان فراهم آورد.
