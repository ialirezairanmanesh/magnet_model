% !TeX root=SBUKThesis-main.tex
\clearpage
\thispagestyle{empty}
\chapter{نتیجه گیری و پیشنهادات آتی}\label{chap5}

\section{نتیجه‌گیری}
در این پژوهش، یک مدل چندوجهی مبتنی بر ترنسفورمر به نام MAGNET برای تشخیص بدافزار اندروید پیشنهاد شد. این مدل از سه ماژول اصلی تشکیل شده است: EnhancedTabTransformer برای پردازش ویژگی‌های جدولی، GraphTransformer برای تحلیل گراف فراخوانی، و SequenceTransformer برای پردازش توالی‌های API. برای بهینه‌سازی پارامترها از الگوریتم‌های Adam و CosineAnnealingWarmRestarts استفاده شد. همچنین، بهینه‌سازی با روش‌های بهینه‌سازی (۴۷۶ آزمایش) و Optuna (۱۳ آزمایش) پیاده‌سازی شد. دیتاست DREBIN \cite{Drebin} با ۶،۰۹۲ نمونه (۴،۶۴۱ برای آموزش و ۱،۴۵۱ برای تست) برای ارزیابی مدل به‌کار گرفته شد. نتایج نشان داد که مدل MAGNET با دقت ۹۷.۲۴٪، F1 Score ۰.۹۸۲۳ و AUC ۰.۹۹۳۲ عملکرد برتری نسبت به روش‌های پایه دارد.

مقایسه با روش‌های دیگر نشان داد که MAGNET از روش چندوجهی \cite{Alsaleh2023} با دقت ۸۹.۲٪ و روش مبتنی بر ترنسفورمر \cite{TransformerMalware} با دقت ۹۵.۸٪ بهتر عمل می‌کند. همچنین، عملکرد مدل از روش‌های سنتی مانند SVM \cite{ZhangNix2017} و CNN \cite{Vinayakumar2019} به طور قابل توجهی بهتر بود. با این حال، تفاوت‌های جزئی با روش مبتنی بر ترنسفورمر مشاهده شد که می‌تواند به دلیل تفاوت در معماری و پارامترهای مدل باشد.

برای بهبود بیشتر مدل MAGNET، پیشنهاد می‌شود که تعادل کلاس‌ها در دیتاست بهبود یابد و معماری مدل برای دیتاست‌های بزرگ‌تر و متنوع‌تر گسترش یابد. همچنین، آزمایش مدل با داده‌های پویا (مانند الگوهای زمان‌بندی API) و بررسی مقاومت آن در برابر حملات گریز می‌تواند موضوع تحقیقات آینده باشد.

\section{پیشنهادات آتی}
\subsection{پژوهش‌های تکمیلی}
بررسی تأثیر افزایش تعداد لایه‌های ترنسفورمر (\lr{num\_layers} از \lr{1} به \lr{2} یا \lr{3}) در مدل MAGNET با توجه به نتایج بهینه‌سازی و \lr{Optuna} \cite{Optuna2019} که تنها یک لایه را بهینه یافتند، برای بهبود عملکرد در دیتاست‌های بزرگ‌تر و متنوع‌تر پیشنهاد می‌شود. همچنین، آزمایش مدل با داده‌های پویا (مانند الگوهای زمان‌بندی فراخوانی \lr{API}) که در این تحقیق محدود بود، توصیه می‌شود.

\subsection{پیشنهادات اجرایی}
پیاده‌سازی مدل MAGNET در یک سیستم امنیتی واقعی برای اندروید، با ادغام داده‌های پویا (مانند فعالیت شبکه و دسترسی به فایل‌ها) که در دیتاست فعلی به‌صورت محدود استفاده شدند، به‌منظور افزایش دقت تشخیص در محیط‌های عملیاتی پیشنهاد می‌شود. این سیستم می‌تواند به‌عنوان افزونه‌ای برای \lr{Google Play Protect} \cite{GooglePlayProtect} توسعه یابد.

\subsection{تولید داده‌های جدید}
جمع‌آوری دیتاستی با تعادل بیشتر بین کلاس‌ها (افزایش نمونه‌های کلاس \lr{0} به حداقل \lr{1,000} نمونه برای نزدیک شدن به \lr{1,124} نمونه کلاس \lr{1}) و افزودن ویژگی‌های جدید (مانند الگوهای رفتاری کاربران) برای کاهش تأثیر عدم تعادل و ارزیابی جامع‌تر مدل MAGNET توصیه می‌شود.

در نهایت، نتایج این پژوهش زمینه‌ساز ارائه یک چارچوب توسعه‌پذیر برای بهینه‌سازی اعلان‌ها در مدل‌های زبانی بزرگ بوده و می‌تواند بستر مناسبی برای تحقیقات و کاربردهای آینده در حوزه مهندسی اعلان فراهم آورد.
