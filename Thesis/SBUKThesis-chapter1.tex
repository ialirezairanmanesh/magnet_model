% !TeX root=SBUKThesis-main.tex
\clearpage
\thispagestyle{empty}
\chapter{کلیات پژوهش}\label{chap1}
\section{مقدمه و بیان مسئله}\label{intro}
در سال\/های اخیر، پیشرفت\/های چشمگیر در حوزه مدل\/های بزرگ زبانی 
\LTRfootnote{Large Language Models (LLMs)}
 موجب شده تا این مدل\/ها به عنوان ابزارهایی قدرتمند در پردازش زبان طبیعی
 \LTRfootnote{Natural Language Processing}
  و هوش مصنوعی شناخته شوند. اما عملکرد بهینه این مدل\/ها تا حد زیادی وابسته به کیفیت اعلان\/ها
  \LTRfootnote{Prompts}
   یا دستورات ورودی است؛ چرا که با توجه به ساختار این\/گونه مدل\/ها
   \LTRfootnote{Transformers}
    نحوه بیان اعلان\/ها به طور مستقیم بر تفسیر وظایف و تولید خروجی\/های مدل تأثیر می\/گذارد. از این رو، مهندسی اعلان
    \LTRfootnote{Prompt Engineering}
     به عنوان فرآیندی حیاتی برای بهبود عملکرد مدل\/ها مطرح شده است.

\subsection{طراحی دستی اعلان}
روش\/های دستی طراحی اعلان مانند
 زنجیره تفکر \LTRfootnote{ Chain-of-Thought (CoT) }\cite{CoT}
 ، 
 برنامه تفکر \LTRfootnote{Program-of-Thoughts (PoT)}\cite{PoT} 
 و
 برنامه\/ریزی و حل \LTRfootnote{Plan-and-Solve}\cite{PS}
 توانسته\/اند با تقسیم حل مسائل پیچیده به مراحل میانی، قدرت استدلال مدل\/ها را تقویت کنند؛ اما این روش\/ها به دلایل زیادی از جمله وابستگی به حدس و گمان انسانی، پوشش ناکافی فضای گسترده جستجوی اعلان\/هاو همچنین تغییرات سریع در قابلیت\/های مدل\/های زبانی، غالباً برای وظایف دامنه\/محور بهینه عمل نمی\/کنند. قابل ذکر است که فرآیند دستی طراحی و تنظیم اعلان\/ها زمان\/بر بوده و به علت پیچیدگی\/های موجود، نیاز به خودکارسازی این فرآیند بوجود آمد.

\subsection{روش\/های بهینه\/سازی اعلان و خودکارسازی مهندسی اعلان}
جهت رفع محدودیت\/های روش\/های دستی، رویکردهای خودکارسازی مهندسی اعلان به کمک الگوریتم\/های بهینه\/سازی پیشنهاد شده\/اند. در این راستا، چندین رویکرد مطرح شده\/اند:
\begin{itemize}
	\item الگوریتم\/های تکاملی
	\LTRfootnote{Evolutionary Algorithms}: 
	روش\/هایی مانند مولد اعلان
	\LTRfootnote{PromptBreeder (PB)}\cite{PromptBreeder}
	 از یک چارچوب تکاملی برپایه الگوریتم\/های ژنتیک استفاده می\/کنند. در این روش، اعلان\/های اولیه به صورت جمعیتی تولید می\/شوند و از طریق مراحل نمونه\/گیری، انتخاب و ارزیابی، نمونه\/های بهینه\/تر شناسایی و تقویت می\/شوند. این فرآیند که بر مبنای تغییرات تدریجی و ارزیابی مکرر استوار است، تلاش می\/کند تا بهترین نمونه\/های اعلان را برای وظایف مشخص پیدا کند؛ اما پیچیدگی چندلایه و وابستگی به تنظیمات دقیق، می\/تواند مانعی بر سر راه تعمیم آن در کاربردهای مختلف باشد.
	\item الگوریتم\/های مبتنی بر چارچوب\/های ریاضی:
	استفاده از روش\/هایی مانند فرآیندهای نقطه‌ای دترمینانی
	\LTRfootnote{Determinantal Point Processes (DPPs)}\cite{DPP_for_ML}\cite{DPP}
	 به انتخاب زیرمجموعه\/ای از اعلان\/های متنوع و با کیفیت کمک می\/کند. این رویکرد به گونه\/ای طراحی شده که بتواند تنوع و کیفیت اعلان\/ها را به صورت همزمان بهینه کند و در نتیجه، تعداد فراخوانی\/های مدل\/های زبانی را کاهش داده و از لحاظ محاسباتی کارآمدتر باشد.
	 
	\item الگوریتم\/های مبتنی بر جستجوی تصادفی و بهینه\/سازی گرادیان:
	برخی روش\/ها سعی در استفاده از تکنیک\/های جستجوی تصادفی و بهینه\/سازی گرادیان برای تنظیم خودکار اعلان\/ها دارند. اگرچه این روش\/ها نیز نویدبخش هستند، اما معمولاً به دلیل فضای جستجوی بسیار بزرگ و پیچیدگی\/های محاسباتی، نیاز به تنظیمات دقیق و منابع محاسباتی بالایی دارند.
\end{itemize}

\subsection{مجموعه داده\/های مربوطه}
در حوزه مسائل ریاضی، \/مجموعه داده\/های متنوعی برای ارزیابی عملکرد اعلان\/های بهینه\/شده مورد استفاده قرار گرفته\/است. از جمله این مجموعه داده\/ها می\/توان به موارد زیر اشاره کرد:
\begin{itemize}
	\item مجموعه داده های
	 MultiArith \cite{MultiArith}
	  و
	 SingleEq \cite{SingleEQ} 
	 :
	شامل مسائل ریاضی ساده و چند مرحله\/ای که توانایی مدل\/ها در حل مسائل محاسباتی پایه و استنتاج منطقی را مورد ارزیابی قرار می\/دهند.
	\item مجموعه داده های
	 AddSub \cite{AddSub}
	 و
	 SVAMP \cite{SVAMP} 
	 :
	مسائل جمع و تفریق و همچنین مسائل ترکیبی را شامل می\/شوند که دقت و سرعت حل مسائل را به چالش می\/کشند.
	\item مجموعه داده های
	 SQA \cite{SQA} 
	 ،
	 CSQA \cite{CSQA} 
	 ، 
	 AQuA-RAT \cite{AquaRat}
	 و
	 GSM8K \cite{GSM8k} 
	 :
	شامل مسائل با سطوح مختلف دشواری، از جمله مسائل ریاضی پیچیده\/تر می\/باشند که علاوه بر محاسبات صحیح، نیاز به توانایی استدلال و توضیح روند حل مسئله نیز دارند.
\end{itemize}

این مجموعه داده\/ها به عنوان شاخص\/های استاندارد امکان ارزیابی دقیق و جامع عملکرد الگوریتم\/های بهینه\/سازی اعلان را فراهم می\/کنند و نقش مهمی در اثبات قابلیت تعمیم و کارایی روش\/های خودکارسازی مهندسی اعلان دارند.

\section{ ضرورت تحقیق و اهداف}\label{import}
همانطور که گفته شد، یکی از چالش\/های اساسی مهندسی اعلان بهبود و بهینه\/سازی اعلان\/ها به گونه\/ای است که توان مدل در تفسیر وظایف و تولید خروجی\/های دقیق تقویت شود. از چالش های موجود در اعلان\/سازی دستی میتوان به موارد زیر اشاره کرد:

\begin{itemize}
	\item روش\/های دستی مبتنی بر تجربه و درک انسانی  از مسئله طراحی شده\/اند و به تخصص و حدس انسانی وابسته \/اند. این وابستگی موجب عدم توانایی در پوشش فضای گسترده اعلان\/های ممکن شده است و اغلب به نتایج بهینه نمی\/رسند.
	\item فرآیند تنظیم و آزمایش دستی اعلان\/ها نیازمند تلاش\/های تکراری و زمان\/بر است که می\/تواند منابع محاسباتی و انسانی را به شدت مصرف کند.
	\item روش\/های دستی معمولاً برای یک دامنه یا وظیفه خاص طراحی می\/شوند و در مواجهه با تغییرات دامنه یا وظایف جدید، عملکرد مناسبی از خود نشان نمی\/دهند.
\end{itemize}

در این راستا، ضرورت تحقیق و توسعه رویکردهای خودکار در مهندسی اعلان به دلایل زیر برجسته می\/شود:

\begin{itemize}
	\item استفاده از الگوریتم\/های بهینه\/سازی خودکار می\/تواند نیاز به تنظیم دستی و آزمون\/های متعدد را کاهش دهد و در نتیجه مصرف منابع را بهبود بخشد.
	\item یک سیستم خودکار قادر است به\/طور همزمان به دنبال اعلان\/های با کیفیت و متنوع بگردد. این امر باعث می\/شود که مدل\/های زبانی بتوانند در وظایف دامنه\/محور (به ویژه در حل مسائل ریاضی) عملکرد بهتری داشته باشند؛ چرا که از تنوع مثال\/ها و راه\/حل\/های ارائه شده بهره\/مند می\/شوند.
\end{itemize}


در زمینه مسائل ریاضی، که معمولاً شامل مجموعه داده\/هایی می\/شوند که در بخش قبل به تفصیل توضیح داده شدند، چالش\/های خاصی از جمله نیاز به استدلال دقیق و توان حل مسائل چند مرحله\/ای مطرح است. بهبود اعلان\/ها در این حوزه می\/تواند منجر به افزایش دقت حل مسائل ریاضی، بهبود توان استدلال و توضیح روند حل مسئله و همچنین فراهم آوردن یک چارچوب ارزیابی استاندارد برای مقایسه روش\/های مختلف مهندسی اعلان شود.



با توجه به چالش\/ها و نیازهای مطرح شده، اهداف اصلی این تحقیق بدین شرح می باشد:
\begin{itemize}
	\item توسعه یک چارچوب ساده و کارآمد و معرفی رویکردی نوین به نام تولید بهینه اعلان
	\LTRfootnote{SimplePromptBreeder (SPB)}
	 که از الگوریتم\/های تکاملی و مدل احتمالاتی فرآیندهای نقطه\/ای دترمینانی بهره می\/گیرد تا اعلان\/های با کیفیت و متنوع تولید کند.
	\item با خودکارسازی فرآیند مهندسی اعلان، تلاش می\/شود تا نیاز به تخصص و حدس انسانی کاهش یابد و نتایج به صورت سیستماتیک و قابل تکرار حاصل شود.
	\item 
	رویکرد تولید بهینه اعلان دقت و کارایی مدل های زبانی را در حل مسائل دامنه محور از جمله مسائل ریاضی که از نظر استدلال و محاسبات چالش برانگیز هستند، بهبود می\/بخشد.
	\item 
	روش تولید بهینه اعلان با کاهش تعداد فراخوانی\/های مدل\/های زبانی در فرآیند بهینه\/سازی، از منابع محاسباتی به شکل بهینه\/تری استفاده می\/کند.
	\item  این چارچوب به سادگی قابل تعمیم به وظایف و دامنه\/های مختلف است و در مواجهه با تغییرات ساختاری مدل\/های زبانی، عملکرد مناسبی ارائه می\/دهد.
\end{itemize}

در مجموع، این تحقیق با هدف ارائه راهکاری نوین و کارآمد در حوزه مهندسی اعلان، به دنبال ایجاد توازنی بین کیفیت، تنوع و کارایی محاسباتی است که بتواند نیازهای رو به رشد دنیای هوش مصنوعی و پردازش زبان طبیعی را برآورده سازد.

\section{سازماندهی پایان نامه}\label{organiz}
در این پایان\/نامه، ساختار مطالب به گونه\/ای تدوین شده که مسیر پژوهش از مبانی نظری و معرفی مسئله تا ارائه نتایج تجربی به صورت پیوسته و منطقی دنبال شود. به عبارت دیگر، هدف از سازماندهی مطالب این است که خواننده بتواند به راحتی با مباحث پایه، چالش\/ها، روش\/های موجود، و نوآوری\/های پیشنهادی آشنا شود و در نهایت به درک جامع از دستاوردهای تحقیق دست یابد. ساختار کلی پایان\/نامه  به شرح زیر است:

\begin{itemize}
	\item فصل دوم – پیشینه تحقیق و مفاهیم پایه:
	
	در این فصل، ابتدا به بررسی کلی مدل\/های بزرگ زبانی و اهمیت مهندسی اعلان پرداخته می\/شود. سپس، چالش\/ها و محدودیت\/های روش\/های دستی بیان شده و مسئله تحقیق به تفصیل معرفی می\/شود. هدف این فصل ایجاد زمینه نظری مناسب برای درک اهمیت بهینه\/سازی خودکار اعلان\/هاست.
	
	در ادامه به بررسی جامع مطالعات پیشین در حوزه بهینه\/سازی اعلان\/ها پرداخته می\/شود. در این بخش، رویکردهای مختلف از جمله الگوریتم\/های تکاملی و سایر روش\/های خودکارسازی مهندسی اعلان مورد تحلیل قرار می\/گیرند. نقاط قوت و ضعف هر یک از این رویکردها همراه با چالش\/های موجود در هر کدام به تفصیل بررسی می\/شود.
	
	\item فصل سوم – روش پیشنهادی (تولید بهینه اعلان) :
	
	در این فصل، رویکرد نوین پیشنهادی با عنوان تولید بهینه اعلان به صورت کامل تشریح می\/شود. ابتدا مفاهیم نظری و چارچوب ریاضی مورد استفاده، به ویژه استفاده از مدل احتمالاتی فرآیندهای نقطه\/ای دترمینانی برای تضمین تنوع و کیفیت اعلان\/ها، توضیح داده می\/شود. سپس، مراحل گام به گام الگوریتم شامل نمونه\/گیری، انتخاب و ارزیابی اعلان\/ها به تفصیل بیان شده و نوآوری\/های اصلی این روش نسبت به سایر روش\/ها برجسته می\/شود.
	
	\item فصل چهارم – نتایج و بحث:
	
	این فصل به ارائه نتایج آزمایش\/های انجام شده بر روی چندین مجموعه داده معتبر اختصاص دارد. مجموعه داده\/های به کار رفته شامل مجموعه\/های مسائل ریاضی می\/باشند که هر یک از آن\/ها چالش\/های خاص خود در زمینه استدلال و حل مسئله را به نمایش می\/گذارند. در این فصل، عملکرد روش تولید بهینه اعلان از نظر دقت، کارایی و صرفه\/جویی در منابع محاسباتی مورد مقایسه قرار گرفته و نتایج به دست آمده تحلیل می\/شوند.
	
	\item فصل پنجم – نتیجه\/گیری و پیشنهادات آتی:
	
	در فصل نهایی، یافته\/های اصلی تحقیق به طور خلاصه ارائه شده و به نتیجه\/گیری کلی از دستاوردهای پژوهش پرداخته می\/شود. در این بخش، چالش\/های باقی\/مانده، محدودیت\/های تحقیق و نیز پیشنهاداتی جهت تحقیقات آتی و بهبود رویکرد ارائه می\/شود.
\end{itemize}



