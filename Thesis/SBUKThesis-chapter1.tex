% !TeX root=SBUKThesis-main.tex
\clearpage
\thispagestyle{empty}
\chapter{کلیات پژوهش}\label{chap1}
\section{مقدمه و بیان مسئله}\label{intro}
امروزه، تلفن‌های همراه به بخشی جدایی‌ناپذیر از زندگی روزمره تبدیل شده‌اند و به‌طور گسترده‌ای برای اهداف گوناگون مورد استفاده قرار می‌گیرند. این دستگاه‌ها، که نمادی از آخرین تحولات فناوری در قرن بیست و یکم به شمار می‌روند، زندگی روزمره و ارتباطات اجتماعی را به‌سوی وابستگی بیشتر به این ابزارهای محبوب سوق داده‌اند. رشد سریع این فناوری، آن را به حوزه‌های تجاری و نظامی راه برده و به دلیل فراگیری بین گروه‌های سنی مختلف در سراسر جهان، محبوبیت بی‌نظیری یافته است. تلفن‌های هوشمند، به‌ویژه در پلتفرم‌های متنوعی مانند اندروید، که یکی از پرکاربردترین و شناخته‌شده‌ترین سیستم‌عامل‌هاست، با استقبال گسترده مواجه شده‌اند. شهرت پلتفرم اندروید، بازار آن را پررونق کرده، اما این محبوبیت، محیط‌هایی خطرناک را نیز به همراه داشته است که از آن سوءاستفاده می‌کنند. برای نمونه، عرضه نرم‌افزارهای ظاهراً رایگان از منابع غیرمعتبر، اغلب با تهدیداتی مانند ویروس‌ها و بدافزارها همراه است. از این رو، برنامه‌های اندرویدی باید از هرگونه تهدید امنیتی، چه فعال و چه غیرفعال، مانند حملات انسداد سرویس (DoS)، ایمن بمانند و این موضوع نیازمند بررسی دقیق است.

تحقیقات متعددی در سطح جهانی برای حفظ امنیت این سیستم‌ها انجام شده که رویکردهای متفاوتی را شامل می‌شوند، از جمله روش‌های پیشگیری و تشخیص بدافزارهای موبایلی. برخی از این روش‌ها بر پایه تطابق امضا عمل می‌کنند و تا حدی در تشخیص موفق‌اند، در حالی که روش‌های دیگر با استفاده از تحلیل کد منبع و استنتاج، نتیجه‌گیری می‌کنند که آیا یک برنامه بدافزار است یا خیر. این رویکردها و تحقیقات، تفاوت‌هایی در قدرت تشخیص و دقت دارند و طی سال‌ها با پژوهش‌های گسترده، بهبودهای قابل‌توجهی یافته‌اند. با این حال، این پژوهش با توسعه مدل تبدیل‌گر چندوجهی مبتنی بر جاسازی گراف با توجه پویا (MAGNET)، که از داده‌های تبلوار، گرافی و ترتیبی بهره می‌برد، به دنبال ارتقای دقت و تعمیم‌پذیری تشخیص بدافزارهای اندرویدی است. این مدل با بهینه‌سازی‌های پیشرفته و اعتبارسنجی متقاطع، راه‌حلی نوین برای مقابله با تهدیدات نوظهور ارائه می‌دهد.

در سال‌های اخیر، با افزایش تهدیدات سایبری، تشخیص بدافزارهای اندرویدی به یکی از چالش‌های اصلی در حوزه امنیت اطلاعات تبدیل شده است. روش‌های سنتی، به‌ویژه آن‌هایی که صرفاً بر تحلیل ویژگی‌های تک‌وجهی تکیه دارند، اغلب با محدودیت‌هایی نظیر ناتوانی در پردازش داده‌های پیچیده چندوجهی و تعمیم‌پذیری ضعیف در برابر تهدیدات جدید مواجه‌اند. این کاستی‌ها، نیاز به توسعه رویکردهای نوین و کارآمد را بیش از پیش آشکار می‌سازد. این پژوهش مدلی چندوجهی با عنوان MAGNET توسعه داد که از ترکیب داده‌های جدولی، گراف و ترتیبی، نظیر توالی فراخوانی‌های API، برای شناسایی بدافزارهای اندرویدی بهره برد. هدف اصلی، بهبود دقت و پایداری تشخیص با استفاده از معماری پیشرفته مبتنی بر یادگیری عمیق و ترنسفورمر بود. روش تحقیق شامل بهینه‌سازی هایپرپارامترها با الگوریتم‌های پیشرفته مانند PIRATES و Optuna، آموزش مدل با مجموعه داده‌ای شامل 4641 نمونه آموزشی و 1451 نمونه آزمایشی، و اعتبارسنجی متقاطع 5-تایی شد. ویژگی‌های مورد استفاده شامل ویژگی‌های ایستا مانند مجوزها، فراخوانی‌های API، مقاصد، و نام‌های مؤلفه و ویژگی‌های پویا مانند فعالیت شبکه و دسترسی به فایل‌ها بود. داده‌ها به صورت بردارهای عددی باینری یا نرمال‌سازی شده بودند. ابعاد ویژگی‌ها پس از پیش‌پردازش به 430 ویژگی تنظیم شد. این پژوهش گامی مؤثر در راستای ارتقای سیستم‌های تشخیص خودکار بدافزارها برداشت و پایه‌ای محکم برای توسعه راه‌حل‌های امنیتی پیشرفته‌تر فراهم آورد.

\subsection{طراحی دستی اعلان}
روش\/های دستی طراحی اعلان مانند
 زنجیره تفکر \LTRfootnote{ Chain-of-Thought (CoT) }\cite{CoT}
 ، 
 برنامه تفکر \LTRfootnote{Program-of-Thoughts (PoT)}\cite{PoT} 
 و
 برنامه\/ریزی و حل \LTRfootnote{Plan-and-Solve}\cite{PS}
 توانسته\/اند با تقسیم حل مسائل پیچیده به مراحل میانی، قدرت استدلال مدل\/ها را تقویت کنند؛ اما این روش\/ها به دلایل زیادی از جمله وابستگی به حدس و گمان انسانی، پوشش ناکافی فضای گسترده جستجوی اعلان\/هاو همچنین تغییرات سریع در قابلیت\/های مدل\/های زبانی، غالباً برای وظایف دامنه\/محور بهینه عمل نمی\/کنند. قابل ذکر است که فرآیند دستی طراحی و تنظیم اعلان\/ها زمان\/بر بوده و به علت پیچیدگی\/های موجود، نیاز به خودکارسازی این فرآیند بوجود آمد.

\subsection{روش\/های بهینه\/سازی اعلان و خودکارسازی مهندسی اعلان}
جهت رفع محدودیت\/های روش\/های دستی، رویکردهای خودکارسازی مهندسی اعلان به کمک الگوریتم\/های بهینه\/سازی پیشنهاد شده\/اند. در این راستا، چندین رویکرد مطرح شده\/اند:
\begin{itemize}
	\item الگوریتم\/های تکاملی
	\LTRfootnote{Evolutionary Algorithms}: 
	روش\/هایی مانند مولد اعلان
	\LTRfootnote{PromptBreeder (PB)}\cite{PromptBreeder}
	 از یک چارچوب تکاملی برپایه الگوریتم\/های ژنتیک استفاده می\/کنند. در این روش، اعلان\/های اولیه به صورت جمعیتی تولید می\/شوند و از طریق مراحل نمونه\/گیری، انتخاب و ارزیابی، نمونه\/های بهینه\/تر شناسایی و تقویت می\/شوند. این فرآیند که بر مبنای تغییرات تدریجی و ارزیابی مکرر استوار است، تلاش می\/کند تا بهترین نمونه\/های اعلان را برای وظایف مشخص پیدا کند؛ اما پیچیدگی چندلایه و وابستگی به تنظیمات دقیق، می\/تواند مانعی بر سر راه تعمیم آن در کاربردهای مختلف باشد.
	\item الگوریتم\/های مبتنی بر چارچوب\/های ریاضی:
	استفاده از روش\/هایی مانند فرآیندهای نقطه‌ای دترمینانی
	\LTRfootnote{Determinantal Point Processes (DPPs)}\cite{DPP_for_ML}\cite{DPP}
	 به انتخاب زیرمجموعه\/ای از اعلان\/های متنوع و با کیفیت کمک می\/کند. این رویکرد به گونه\/ای طراحی شده که بتواند تنوع و کیفیت اعلان\/ها را به صورت همزمان بهینه کند و در نتیجه، تعداد فراخوانی\/های مدل\/های زبانی را کاهش داده و از لحاظ محاسباتی کارآمدتر باشد.
	 
	\item الگوریتم\/های مبتنی بر جستجوی تصادفی و بهینه\/سازی گرادیان:
	برخی روش\/ها سعی در استفاده از تکنیک\/های جستجوی تصادفی و بهینه\/سازی گرادیان برای تنظیم خودکار اعلان\/ها دارند. اگرچه این روش\/ها نیز نویدبخش هستند، اما معمولاً به دلیل فضای جستجوی بسیار بزرگ و پیچیدگی\/های محاسباتی، نیاز به تنظیمات دقیق و منابع محاسباتی بالایی دارند.
\end{itemize}

\subsection{مجموعه داده\/های مربوطه}
در حوزه مسائل ریاضی، \/مجموعه داده\/های متنوعی برای ارزیابی عملکرد اعلان\/های بهینه\/شده مورد استفاده قرار گرفته\/است. از جمله این مجموعه داده\/ها می\/توان به موارد زیر اشاره کرد:
\begin{itemize}
	\item مجموعه داده های
	 MultiArith \cite{MultiArith}
	  و
	 SingleEq \cite{SingleEQ} 
	 :
	شامل مسائل ریاضی ساده و چند مرحله\/ای که توانایی مدل\/ها در حل مسائل محاسباتی پایه و استنتاج منطقی را مورد ارزیابی قرار می\/دهند.
	\item مجموعه داده های
	 AddSub \cite{AddSub}
	 و
	 SVAMP \cite{SVAMP} 
	 :
	مسائل جمع و تفریق و همچنین مسائل ترکیبی را شامل می\/شوند که دقت و سرعت حل مسائل را به چالش می\/کشند.
	\item مجموعه داده های
	 SQA \cite{SQA} 
	 ،
	 CSQA \cite{CSQA} 
	 ، 
	 AQuA-RAT \cite{AquaRat}
	 و
	 GSM8K \cite{GSM8k} 
	 :
	شامل مسائل با سطوح مختلف دشواری، از جمله مسائل ریاضی پیچیده\/تر می\/باشند که علاوه بر محاسبات صحیح، نیاز به توانایی استدلال و توضیح روند حل مسئله نیز دارند.
\end{itemize}

این مجموعه داده\/ها به عنوان شاخص\/های استاندارد امکان ارزیابی دقیق و جامع عملکرد الگوریتم\/های بهینه\/سازی اعلان را فراهم می\/کنند و نقش مهمی در اثبات قابلیت تعمیم و کارایی روش\/های خودکارسازی مهندسی اعلان دارند.

\section{ ضرورت تحقیق و اهداف}\label{import}
همانطور که گفته شد، یکی از چالش\/های اساسی مهندسی اعلان بهبود و بهینه\/سازی اعلان\/ها به گونه\/ای است که توان مدل در تفسیر وظایف و تولید خروجی\/های دقیق تقویت شود. از چالش های موجود در اعلان\/سازی دستی میتوان به موارد زیر اشاره کرد:

\begin{itemize}
	\item روش\/های دستی مبتنی بر تجربه و درک انسانی  از مسئله طراحی شده\/اند و به تخصص و حدس انسانی وابسته \/اند. این وابستگی موجب عدم توانایی در پوشش فضای گسترده اعلان\/های ممکن شده است و اغلب به نتایج بهینه نمی\/رسند.
	\item فرآیند تنظیم و آزمایش دستی اعلان\/ها نیازمند تلاش\/های تکراری و زمان\/بر است که می\/تواند منابع محاسباتی و انسانی را به شدت مصرف کند.
	\item روش\/های دستی معمولاً برای یک دامنه یا وظیفه خاص طراحی می\/شوند و در مواجهه با تغییرات دامنه یا وظایف جدید، عملکرد مناسبی از خود نشان نمی\/دهند.
\end{itemize}

در این راستا، ضرورت تحقیق و توسعه رویکردهای خودکار در مهندسی اعلان به دلایل زیر برجسته می\/شود:

\begin{itemize}
	\item استفاده از الگوریتم\/های بهینه\/سازی خودکار می\/تواند نیاز به تنظیم دستی و آزمون\/های متعدد را کاهش دهد و در نتیجه مصرف منابع را بهبود بخشد.
	\item یک سیستم خودکار قادر است به\/طور همزمان به دنبال اعلان\/های با کیفیت و متنوع بگردد. این امر باعث می\/شود که مدل\/های زبانی بتوانند در وظایف دامنه\/محور (به ویژه در حل مسائل ریاضی) عملکرد بهتری داشته باشند؛ چرا که از تنوع مثال\/ها و راه\/حل\/های ارائه شده بهره\/مند می\/شوند.
\end{itemize}

در زمینه مسائل ریاضی، که معمولاً شامل مجموعه داده\/هایی می\/شوند که در بخش قبل به تفصیل توضیح داده شدند، چالش\/های خاصی از جمله نیاز به استدلال دقیق و توان حل مسائل چند مرحله\/ای مطرح است. بهبود اعلان\/ها در این حوزه می\/تواند منجر به افزایش دقت حل مسائل ریاضی، بهبود توان استدلال و توضیح روند حل مسئله و همچنین فراهم آوردن یک چارچوب ارزیابی استاندارد برای مقایسه روش\/های مختلف مهندسی اعلان شود.

پلتفرم اندروید، به دلیل محبوبیت گسترده و سهم عظیمش از بازار جهانی، به هدف اصلی بدافزارها و حملات امنیتی تبدیل شده است. این سیستم‌عامل، که بیش از ۷۰ درصد دستگاه‌های هوشمند را پشتیبانی می‌کند، به دلیل ساختار باز و دسترسی‌پذیری بالا، با تهدیدات پیشرفته‌ای مواجه است. بدافزارهای اندرویدی، از جمله تروجان‌ها، جاسوس‌افزارها و باج‌افزارها، با روش‌های پیچیده‌ای طراحی شده‌اند و پیشرفت‌های چشمگیری داشته‌اند. این تهدیدات، از سرقت اطلاعات حساس گرفته تا ایجاد اختلال در عملکرد دستگاه‌ها، چالش‌های امنیتی جدی ایجاد کرده‌اند. از این رو، نیاز به سیستمی قدرتمند و کارآمد برای تشخیص بدافزارهای اندرویدی بیش از پیش احساس می‌شود. هدف اصلی این پژوهش، تمرکز بر شناسایی بدافزارهای ناشناخته و نادیده (Zero-Day) است که تا کنون شناسایی نشده‌اند و می‌توانند تهدیداتی پنهان برای کاربران ایجاد کنند.

تشخیص بدافزارهای اندرویدی به‌طور کلی از دو روش پویا و ایستا انجام می‌شود. در روش پویا، رفتار اپلیکیشن در زمان اجرا بررسی می‌شود؛ برای نمونه، میزان مصرف باتری، پردازنده یا ترافیک شبکه تحلیل می‌گردد تا الگوهای غیرعادی شناسایی شوند. با این حال، این روش به‌تنهایی کافی نیست و ممکن است برخی تهدیدات پنهان را نادیده بگیرد. از سوی دیگر، روش ایستا با تحلیل ساختار و کد اپلیکیشن، مانند بررسی فراخوانی‌های API و مجوزها، اطلاعات ارزشمندی ارائه می‌دهد که می‌تواند در تشخیص دقیق‌تر کمک کند. این روش، به دلیل توانایی در تحلیل عمیق‌تر بدون نیاز به اجرا، همچنان رویکردی مهم و کاربردی محسوب می‌شود، اما دقت آن به کیفیت تحلیل بستگی دارد. پژوهش‌های پیشین نشان داده‌اند که هر یک از این روش‌ها به‌تنهایی محدودیت‌هایی دارند و نمی‌توانند به‌طور کامل با تهدیدات پیشرفته مقابله کنند. سؤالی که اینجا مطرح می‌شود این است که چگونه می‌توان روشی ترکیبی طراحی کرد که با بهره‌گیری از مزایای روش‌های ایستا و پویا، و در عین حال استفاده از تکنیک‌های پیشرفته یادگیری عمیق، دقتی بالا و قابل‌اعتماد در تشخیص بدافزارها ارائه دهد و با اطمینان مشخص کند که یک نرم‌افزار تهدید امنیتی است یا خیر.

این پژوهش با معرفی مدل تبدیل‌گر چندوجهی مبتنی بر جاسازی گراف با توجه پویا (MAGNET)، به دنبال پاسخی برای این سؤال است. این مدل با ترکیب داده‌های جدولی، گرافی و ترتیبی، و بهره‌گیری از الگوریتم‌های بهینه‌سازی پیشرفته، رویکردی جامع برای تشخیص بدافزارهای اندرویدی ارائه می‌دهد.

با توجه به چالش\/ها و نیازهای مطرح شده، اهداف اصلی این تحقیق بدین شرح می باشد:
\begin{itemize}
	\item توسعه یک چارچوب ساده و کارآمد و معرفی رویکردی نوین به نام تولید بهینه اعلان
	\LTRfootnote{SimplePromptBreeder (SPB)}
	 که از الگوریتم\/های تکاملی و مدل احتمالاتی فرآیندهای نقطه\/ای دترمینانی بهره می\/گیرد تا اعلان\/های با کیفیت و متنوع تولید کند.
	\item با خودکارسازی فرآیند مهندسی اعلان، تلاش می\/شود تا نیاز به تخصص و حدس انسانی کاهش یابد و نتایج به صورت سیستماتیک و قابل تکرار حاصل شود.
	\item 
	رویکرد تولید بهینه اعلان دقت و کارایی مدل های زبانی را در حل مسائل دامنه محور از جمله مسائل ریاضی که از نظر استدلال و محاسبات چالش برانگیز هستند، بهبود می\/بخشد.
	\item 
	روش تولید بهینه اعلان با کاهش تعداد فراخوانی\/های مدل\/های زبانی در فرآیند بهینه\/سازی، از منابع محاسباتی به شکل بهینه\/تری استفاده می\/کند.
	\item  این چارچوب به سادگی قابل تعمیم به وظایف و دامنه\/های مختلف است و در مواجهه با تغییرات ساختاری مدل\/های زبانی، عملکرد مناسبی ارائه می\/دهد.
\end{itemize}

در مجموع، این تحقیق با هدف ارائه راهکاری نوین و کارآمد در حوزه مهندسی اعلان، به دنبال ایجاد توازنی بین کیفیت، تنوع و کارایی محاسباتی است که بتواند نیازهای رو به رشد دنیای هوش مصنوعی و پردازش زبان طبیعی را برآورده سازد.

\section{سازماندهی پایان نامه}\label{organiz}
در این پایان\/نامه، ساختار مطالب به گونه\/ای تدوین شده که مسیر پژوهش از مبانی نظری و معرفی مسئله تا ارائه نتایج تجربی به صورت پیوسته و منطقی دنبال شود. به عبارت دیگر، هدف از سازماندهی مطالب این است که خواننده بتواند به راحتی با مباحث پایه، چالش\/ها، روش\/های موجود، و نوآوری\/های پیشنهادی آشنا شود و در نهایت به درک جامع از دستاوردهای تحقیق دست یابد. ساختار کلی پایان\/نامه  به شرح زیر است:

\begin{itemize}
	\item فصل دوم – پیشینه تحقیق و مفاهیم پایه:
	
	در این فصل، ابتدا به بررسی کلی مدل\/های بزرگ زبانی و اهمیت مهندسی اعلان پرداخته می\/شود. سپس، چالش\/ها و محدودیت\/های روش\/های دستی بیان شده و مسئله تحقیق به تفصیل معرفی می\/شود. هدف این فصل ایجاد زمینه نظری مناسب برای درک اهمیت بهینه\/سازی خودکار اعلان\/هاست.
	
	در ادامه به بررسی جامع مطالعات پیشین در حوزه بهینه\/سازی اعلان\/ها پرداخته می\/شود. در این بخش، رویکردهای مختلف از جمله الگوریتم\/های تکاملی و سایر روش\/های خودکارسازی مهندسی اعلان مورد تحلیل قرار می\/گیرند. نقاط قوت و ضعف هر یک از این رویکردها همراه با چالش\/های موجود در هر کدام به تفصیل بررسی می\/شود.
	
	\item فصل سوم – روش پیشنهادی (تولید بهینه اعلان) :
	
	در این فصل، رویکرد نوین پیشنهادی با عنوان تولید بهینه اعلان به صورت کامل تشریح می\/شود. ابتدا مفاهیم نظری و چارچوب ریاضی مورد استفاده، به ویژه استفاده از مدل احتمالاتی فرآیندهای نقطه\/ای دترمینانی برای تضمین تنوع و کیفیت اعلان\/ها، توضیح داده می\/شود. سپس، مراحل گام به گام الگوریتم شامل نمونه\/گیری، انتخاب و ارزیابی اعلان\/ها به تفصیل بیان شده و نوآوری\/های اصلی این روش نسبت به سایر روش\/ها برجسته می\/شود.
	
	\item فصل چهارم – نتایج و بحث:
	
	این فصل به ارائه نتایج آزمایش\/های انجام شده بر روی چندین مجموعه داده معتبر اختصاص دارد. مجموعه داده\/های به کار رفته شامل مجموعه\/های مسائل ریاضی می\/باشند که هر یک از آن\/ها چالش\/های خاص خود در زمینه استدلال و حل مسئله را به نمایش می\/گذارند. در این فصل، عملکرد روش تولید بهینه اعلان از نظر دقت، کارایی و صرفه\/جویی در منابع محاسباتی مورد مقایسه قرار گرفته و نتایج به دست آمده تحلیل می\/شوند.
	
	\item فصل پنجم – نتیجه\/گیری و پیشنهادات آتی:
	
	در فصل نهایی، یافته\/های اصلی تحقیق به طور خلاصه ارائه شده و به نتیجه\/گیری کلی از دستاوردهای پژوهش پرداخته می\/شود. در این بخش، چالش\/های باقی\/مانده، محدودیت\/های تحقیق و نیز پیشنهاداتی جهت تحقیقات آتی و بهبود رویکرد ارائه می\/شود.
\end{itemize}



